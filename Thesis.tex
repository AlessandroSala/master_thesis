% A LaTeX template for MSc Thesis submissions to 
% Politecnico di Milano (PoliMi) - School of Industrial and Information Engineering
%
% S. Bonetti, A. Gruttadauria, G. Mescolini, A. Zingaro
% e-mail: template-tesi-ingind@polimi.it
%
% Last Revision: October 2021
%
% Copyright 2021 Politecnico di Milano, Italy. NC-BY

\documentclass{Configuration_Files/PoliMi3i_thesis}

%------------------------------------------------------------------------------
%	REQUIRED PACKAGES AND  CONFIGURATIONS
%------------------------------------------------------------------------------

% CONFIGURATIONS
\usepackage{parskip} % For paragraph layout
\usepackage{setspace} % For using single or double spacing
\usepackage{emptypage} % To insert empty pages
\usepackage{multicol} % To write in multiple columns (executive summary)
\setlength\columnsep{15pt} % Column separation in executive summary
\setlength\parindent{0pt} % Indentation
\raggedbottom  

% PACKAGES FOR TITLES
\usepackage{titlesec}
% \titlespacing{\section}{left spacing}{before spacing}{after spacing}
\titlespacing{\section}{0pt}{3.3ex}{2ex}
\titlespacing{\subsection}{0pt}{3.3ex}{1.65ex}
\titlespacing{\subsubsection}{0pt}{3.3ex}{1ex}
\usepackage{color}

% PACKAGES FOR LANGUAGE AND FONT
\usepackage[english]{babel} % The document is in English  
\usepackage[utf8]{inputenc} % UTF8 encoding
\usepackage[T1]{fontenc} % Font encoding
\usepackage[11pt]{moresize} % Big fonts

% PACKAGES FOR IMAGES
\usepackage{graphicx}
\usepackage{transparent} % Enables transparent images
\usepackage{eso-pic} % For the background picture on the title page
\usepackage{subfig} % Numbered and caption subfigures using \subfloat.
\usepackage{tikz} % A package for high-quality hand-made figures.
\usetikzlibrary{}
\graphicspath{{./Images/}} % Directory of the images
\usepackage{caption} % Coloured captions
\usepackage{xcolor} % Coloured captions
\usepackage{amsthm,thmtools,xcolor} % Coloured "Theorem"
\usepackage{float}
\usepackage{physics}

% STANDARD MATH PACKAGES
\usepackage{amsmath}
\usepackage{amsthm}
\usepackage{amssymb}
\usepackage{amsfonts}
\usepackage{bm}
\usepackage[overload]{empheq} % For braced-style systems of equations.
\usepackage{fix-cm} % To override original LaTeX restrictions on sizes

% PACKAGES FOR TABLES
\usepackage{tabularx}
\usepackage{longtable} % Tables that can span several pages
\usepackage{colortbl}

% PACKAGES FOR ALGORITHMS (PSEUDO-CODE)
\usepackage{algorithm}
\usepackage{algorithmic}

% PACKAGES FOR REFERENCES & BIBLIOGRAPHY
\usepackage[colorlinks=true,linkcolor=black,anchorcolor=black,citecolor=black,filecolor=black,menucolor=black,runcolor=black,urlcolor=black]{hyperref} % Adds clickable links at references
\usepackage{cleveref}
\usepackage[square, numbers, sort&compress]{natbib} % Square brackets, citing references with numbers, citations sorted by appearance in the text and compressed
\bibliographystyle{abbrvnat} % You may use a different style adapted to your field

% OTHER PACKAGES
\usepackage{pdfpages} % To include a pdf file
\usepackage{afterpage}
\usepackage{lipsum} % DUMMY PACKAGE
\usepackage{fancyhdr} % For the headers
\fancyhf{}

% Input of configuration file. Do not change config.tex file unless you really know what you are doing. 
% Define blue color typical of polimi
\definecolor{bluepoli}{cmyk}{0.4,0.1,0,0.4}

% Custom theorem environments
\declaretheoremstyle[
  headfont=\color{bluepoli}\normalfont\bfseries,
  bodyfont=\color{black}\normalfont\itshape,
]{colored}

% Set-up caption colors
\captionsetup[figure]{labelfont={color=bluepoli}} % Set colour of the captions
\captionsetup[table]{labelfont={color=bluepoli}} % Set colour of the captions
\captionsetup[algorithm]{labelfont={color=bluepoli}} % Set colour of the captions

\theoremstyle{colored}
\newtheorem{theorem}{Theorem}[chapter]
\newtheorem{proposition}{Proposition}[chapter]

% Enhances the features of the standard "table" and "tabular" environments.
\newcommand\T{\rule{0pt}{2.6ex}}
\newcommand\B{\rule[-1.2ex]{0pt}{0pt}}

% Pseudo-code algorithm descriptions.
\newcounter{algsubstate}
\renewcommand{\thealgsubstate}{\alph{algsubstate}}
\newenvironment{algsubstates}
  {\setcounter{algsubstate}{0}%
   \renewcommand{\STATE}{%
     \stepcounter{algsubstate}%
     \Statex {\small\thealgsubstate:}\space}}
  {}

% New font size
\newcommand\numfontsize{\@setfontsize\Huge{200}{60}}

% Title format: chapter
\titleformat{\chapter}[hang]{
\fontsize{50}{20}\selectfont\bfseries\filright}{\textcolor{bluepoli} \thechapter\hsp\hspace{2mm}\textcolor{bluepoli}{|   }\hsp}{0pt}{\huge\bfseries \textcolor{bluepoli}
}

% Title format: section
\titleformat{\section}
{\color{bluepoli}\normalfont\Large\bfseries}
{\color{bluepoli}\thesection.}{1em}{}

% Title format: subsection
\titleformat{\subsection}
{\color{bluepoli}\normalfont\large\bfseries}
{\color{bluepoli}\thesubsection.}{1em}{}

% Title format: subsubsection
\titleformat{\subsubsection}
{\color{bluepoli}\normalfont\large\bfseries}
{\color{bluepoli}\thesubsubsection.}{1em}{}

% Shortening for setting no horizontal-spacing
\newcommand{\hsp}{\hspace{0pt}}

\makeatletter
% Renewcommand: cleardoublepage including the background pic
\renewcommand*\cleardoublepage{%
  \clearpage\if@twoside\ifodd\c@page\else
  \null
  \AddToShipoutPicture*{\BackgroundPic}
  \thispagestyle{empty}%
  \newpage
  \if@twocolumn\hbox{}\newpage\fi\fi\fi}
\makeatother

%For correctly numbering algorithms
\numberwithin{algorithm}{chapter}

%----------------------------------------------------------------------------
%	NEW COMMANDS DEFINED
%----------------------------------------------------------------------------

% EXAMPLES OF NEW COMMANDS
\newcommand{\bea}{\begin{eqnarray}} % Shortcut for equation arrays
\newcommand{\eea}{\end{eqnarray}}
\newcommand{\e}[1]{\times 10^{#1}}  % Powers of 10 notation

%----------------------------------------------------------------------------
%	ADD YOUR PACKAGES (be careful of package interaction)
%----------------------------------------------------------------------------

%----------------------------------------------------------------------------
%	ADD YOUR DEFINITIONS AND COMMANDS (be careful of existing commands)
%----------------------------------------------------------------------------

%----------------------------------------------------------------------------
%	BEGIN OF YOUR DOCUMENT
%----------------------------------------------------------------------------

\begin{document}

\fancypagestyle{plain}{%
\fancyhf{} % Clear all header and footer fields
\fancyhead[RO,RE]{\thepage} %RO=right odd, RE=right even
\renewcommand{\headrulewidth}{0pt}
\renewcommand{\footrulewidth}{0pt}}

%----------------------------------------------------------------------------
%	TITLE PAGE
%----------------------------------------------------------------------------

\pagestyle{empty} % No page numbers
\frontmatter % Use roman page numbering style (i, ii, iii, iv...) for the preamble pages

\puttitle{
	title=Title, % Title of the thesis
	name=Name Surname, % Author Name and Surname
	course=Xxxxxxx Engineering - Ingegneria Xxxxxxx, % Study Programme (in Italian)
	ID  = 000000,  % Student ID number (numero di matricola)
	advisor= Prof. Name Surname, % Supervisor name
	coadvisor={Name Surname, Name Surname}, % Co-Supervisor name, remove this line if there is none
	academicyear={20XX-XX},  % Academic Year
} % These info will be put into your Title page 

%----------------------------------------------------------------------------
%	PREAMBLE PAGES: ABSTRACT (inglese e italiano), EXECUTIVE SUMMARY
%----------------------------------------------------------------------------
\startpreamble
\setcounter{page}{1} % Set page counter to 1

% ABSTRACT IN ENGLISH
\chapter*{Abstract} 
Here goes the Abstract in English of your thesis followed by a list of keywords.
The Abstract is a concise summary of the content of the thesis (single page of text)
and a guide to the most important contributions included in your thesis.
The Abstract is the very last thing you write.
It should be a self-contained text and should be clear to someone who hasn't (yet) read the whole manuscript.
The Abstract should contain the answers to the main scientific questions that have been addressed in your thesis.
It needs to summarize the adopted motivations and the adopted methodological approach as well as the findings of your work and their relevance and impact.
The Abstract is the part appearing in the record of your thesis inside POLITesi,
the Digital Archive of PhD and Master Theses (Laurea Magistrale) of Politecnico di Milano.
The Abstract will be followed by a list of four to six keywords.
Keywords are a tool to help indexers and search engines to find relevant documents.
To be relevant and effective, keywords must be chosen carefully.
They should represent the content of your work and be specific to your field or sub-field.
Keywords may be a single word or two to four words. 
\\
\\
\textbf{Keywords:} here, the keywords, of your thesis % Keywords

% ABSTRACT IN ITALIAN
\chapter*{Abstract in lingua italiana}
Qui va l'Abstract in lingua italiana della tesi seguito dalla lista di parole chiave.
\\
\\
\textbf{Parole chiave:} qui, vanno, le parole chiave, della tesi % Keywords (italian)

%----------------------------------------------------------------------------
%	LIST OF CONTENTS/FIGURES/TABLES/SYMBOLS
%----------------------------------------------------------------------------

% TABLE OF CONTENTS
\thispagestyle{empty}
\tableofcontents % Table of contents 
\thispagestyle{empty}
\cleardoublepage

%-------------------------------------------------------------------------
%	THESIS MAIN TEXT
%-------------------------------------------------------------------------
% In the main text of your thesis you can write the chapters in two different ways:
%
%(1) As presented in this template you can write:
%    \chapter{Title of the chapter}
%    *body of the chapter*
%
%(2) You can write your chapter in a separated .tex file and then include it in the main file with the following command:
%    \chapter{Title of the chapter}
%    \input{chapter_file.tex}
%
% Especially for long thesis, we recommend you the second option.

\addtocontents{toc}{\vspace{2em}} % Add a gap in the Contents, for aesthetics
\mainmatter % Begin numeric (1,2,3...) page numbering

% --------------------------------------------------------------------------
% NUMBERED CHAPTERS % Regular chapters following
% --------------------------------------------------------------------------
\chapter{Energy functional}
\label{chap:hf}
\section{Hartree-Fock theory}
An empirical description of nuclear structure can be carried out using phenomenological models, as reported in section (REF).
\\A more rigorous approach needs to take into account the fact that the mean field which the nucleons interact with, is generated by the nucleons themselves, due to some microscopic interaction.
\\The many-body hamiltonian of the system, given by
\begin{equation}
    \label{eq:mb_hamiltonian}
    \hat H = \hat T + \hat V = \sum_i -\frac{\hbar^2}{2m}\nabla^2_i + \sum_{i<j} v^{(2)}_{ij} + \sum_{i<j<k} v^{(3)}_{ijk }
\end{equation}
acts on the nucleus, a system of $A$ nucleons described by the Slater determinant
\begin{equation}
    \label{eq:slater_formula}
    \Psi = \frac{1}{\sqrt {A!}} \sum_{\{p\}} (-1)^{p}  \varphi_{p(1)}(\bm r_1)\ldots \varphi_{p(A)}(\bm r_A)
\end{equation}
That is, summing over all possible permutations of the $A$ fermions on the single particle states, with a $-$ sign according to the parity of the permutation.
\subsection{Variational principle}
It is possible to show \cite{ring2004nuclear} that the ground state of the many-body system, found by minimizing the functional
\begin{equation}
    \label{eq:functional_hf}
    E[\Psi] = \frac{\bra{\Psi} \hat H \ket{\Psi}}{\bra{\Psi} \ket{\Psi}}
\end{equation}
Is equivalent to the 




\section{Functional}


\section{Pairing in Hartree-Fock theory}
\label{sec:pairing_hf}
In this section, we will discuss the two common approaches to include nuclear pairing in the HF theory. The aim of these few pages is to provide a brief overview of how the BCS equations are derived and understand the basics of the more general Hartree-Fock-Bogoliubov theory. The former method is the most widely implemented thanks to its low complexity \cite{hfbcsqrpa,oldEv8,skyax}, while the latter, more sophisticated and advanced, is the standard in modern codes \cite{hfodd, hfbftt,MAREVIC2022108367}. We will touch on it so that the reader may appreciate in the numerical chapter the natural extension of this work to the more general Bogoliubov ansatz. 
\subsection{BCS theory}
The BCS approximation, from Bardeen-Cooper-Schrieffer, is the same theory used to describe Cooper pairs in superconductivity, applied to the nuclear case.
The ansatz of BCS is that nucleons are paired in states whose total angular momentum is zero, such a wavefunction can be expressed as $\ket{JM}=\ket{00}$ and reads
\begin{equation}
    \label{eq:zero_mom}
    \ket{00} = \sum_{m_j} \bra{jm_j j-m_j}\ket{00}\ket{jm_j}\ket{j-m_j}
\end{equation}
Introducing the time-reversal operator $\hat {\mathcal T}:t\mapsto -t$, it acts on $\ket{00}$ as 
\begin{equation}
    \hat {\mathcal T} \ket{jm_j} = \widetilde{\ket{{jm_j}}} = (-1)^{j+m_j}\ket{j-m_j},
\end{equation}
using this relation, equation \eqref{eq:zero_mom} becomes 
\begin{equation}
    \ket{00} =- \frac{1}{\sqrt{2j+1}}\sum_{m_j}\ket{jm_j}\widetilde{\ket{{jm_j}}}.
\end{equation}
Hence BCS amounts to replacing the Slater determinant with a more general wavefunction to describe the ground state, which reads
\begin{equation}
    \label{eq:bcs_det}
    \bcs = \prod_{k>0}(u_k+ v_k a_{\tilde k}^\dagger a_k^\dagger) \ket{-}
\end{equation}
where $u_k$ and $v_k$ are real parameters whose meaning will shortly be clear, $k$ is short-hand for $\ket{jm}$, and $\tilde k = -k$ denotes the time-reversal state of $k$; the product runs over positive $k$ only. The BCS wavefunction is the creation in the vacuum of quasi-particles made of time-reversal paired particles, instead of individual ones. The normalization condition on the BCS wavefunction reads
\begin{equation}
    \label{eq:bcs_norm}
    1=\braket{\text{BCS}} = \prod_{k>0} \bra{-}(u_{k}+v_{k}a_{k}a_{\tilde k})(u_k+v_k a_{\tilde k}^\dagger a_k^\dagger)\ket{-} =\prod_{k>0}(u_k^2+v_k^2)=1
\end{equation}
which implies, for every pair $k$, the condition
\begin{equation}
    \label{eq:norm_uv}
    u_k^2+v_k^2=1.
\end{equation}
Taking the expectation value of the particle number operator $\hat N = \sum_k a_k ^\dagger a_k$ yields \cite{bertulani2007nuclear}
\begin{equation}
    \bbcs \hat N \bcs = 2\sum_ {k>0} v_k^2,
\end{equation}
while the expectation value of the particle number dispersion reads
\begin{equation}
\expval{\Delta\hat N ^2} = \expval{\hat N^2} - \expval{\hat N}^2 = 4\sum_{k>0} v_k^2 u_k^2.
\end{equation}
The consequence of this result is profound. The BCS ansatz does not assume a fixed number of particles, rather it becomes an observable of the system, with an expectation value that depends on how the parameters $v_k^2$ are set, which represent the probability of finding a particle in the $k$-th state.
We can now write the many body Hamiltonian of the system as in equation \eqref{eq:mb_hamiltonian_sq}
\begin{equation}
    \hat H = \sum_{k_1 k_2}t_{k_1k_2} a_{k_1}^\dagger a_{k_2} + \frac 1 4 \sum_{k_1 k_2 k_3 k_4}\overline{v}_{k_1k_2k_3k_4} a_{k_1}^\dagger a_{k_2}^\dagger a_{k_3} a_{k_4}
\end{equation}
and replace it with the Routhian 
\begin{equation}
    \label{eq:bcs_routhian}
    \bbcs \hat H - \lambda \hat N \bcs
\end{equation}
so that the expected number of particles may be fixed, under the appropriate choice of $\lambda$, by the relation
\begin{equation}
\pdv{}{N}\bbcs \hat H  \bcs = \lambda.
\end{equation}
the Lagrange multiplier $\lambda$ takes on the meaning of the Fermi energy.
We can now apply the variational principle \eqref{eq:var_eq_res} to \eqref{eq:bcs_routhian} using the $v_k$ as variational quantities, which yields 
\begin{equation}
   4\tilde \varepsilon_k ^2 u_k^2 v_k^2 = \Delta_k ^2 - 4\Delta_k^2 u_k^2v_k^2,
\end{equation}
where the pairing gap $\Delta_k$ is defined as
\begin{equation}
    \label{eq:delta_k}
    \Delta_k = - \sum_{k'}\overline v_{k\tilde k k'\tilde{k'}}v_{k'}u_{k'} 
    \end{equation}
and the quantity $\tilde\varepsilon_k$ is defined as 
\begin{align}
    \label{eq:epsilon}
    \tilde \varepsilon_k &= \frac 1 2 \bigg[t_{kk} + t_{\tilde k \tilde k }-2\lambda +\sum_{k'}(\overline v_{k\tilde k k'\tilde{k'}}v_{k'}u_{k'} + \overline v_{\tilde k k' \tilde k k'})v_{k'}^2\bigg]
    \\&=\frac 1 2 [h_{kk}+h_{\tilde k \tilde k}]-\lambda.
\end{align}
Introducing the quasi-particle energy
\begin{equation}
    \label{eq:qpe}
    E_k = \sqrt{\tilde \varepsilon_k^2 +\Delta_k^2}
\end{equation}
we can combine definitions \eqref{eq:delta_k} and \eqref{eq:qpe} with equation \eqref{eq:epsilon}, under the normalization condition \eqref{eq:norm_uv}, to get an equation for $v_k^2$
\begin{equation}
    \label{eq:v2}
    v_k^2 = \frac 1 2 \pm \frac{|\tilde \varepsilon_k|}{2E_k}.
\end{equation}
Since in the Hartree-Fock limit, where the occupations $v_k^2$ are equal to one below the fermi energy and zero above, and the gaps $\Delta_k$ vanish, rendering $E_k=\tilde \varepsilon_k$, we only select the solution
\begin{equation}
v_k^2 = \frac 1 2 - \frac{\tilde \varepsilon_k}{2E_k}.
\end{equation}
Using the normalization condition to write $u_k^2=1-v_k^2$, and plugging it into the gaps definition \eqref{eq:delta_k}, we arrive to the gap equation
\begin{equation}
    \label{eq:gap}
    \Delta_k = - \sum_{k'}\frac{\Delta_{k'}\overline v_{k\tilde k k'\tilde{k'}}}{2E_{k'}}
\end{equation}
The system of equations (\ref{eq:gap}, \ref{eq:norm_uv}, \ref{eq:v2}, \ref{eq:qpe}, \ref{eq:epsilon}), together with the condition on $\hat N$ -- ie $\expval{\hat N}=N$ -- is closed and can be solved numerically, usually through an effective pairing interaction.
\subsection{Hartree-Fock-Bogoliubov theory}
\label{sec:hfb}
The most general ansatz to account for pairing interactions in Hartree-Fock theory is the Hartree-Fock-Bogoliubov (HFB) theory, it allows a treatment of the mean-field and pairing interactions in a unified way, the quasi-particles created on the vacuum are the most general ones, instead of being time-reversal paired particles.
Let us start by writing a Bogoliubov transformation from the particle basis $c_i$ to a quasi-particle one 
\begin{equation}
    \label{eq:bogoliubov_trans}
    \beta_k^\dagger = \sum_l U_{lk} c_l^\dagger + V_{lk} c_l.
\end{equation}
If we take the Hermitian conjugate of the relation \eqref{eq:bogoliubov_trans}, we get the transformation for $\beta_k$, we are then able to write in matrix form
\begin{equation}
    \label{eq:bogoliubov_trans_mat}
    \begin{pmatrix}
    \beta\\
    \beta^\dagger
    \end{pmatrix}
    =\begin{pmatrix}
        U^\dagger & V^\dagger \\
        V^T & U^T
    \end{pmatrix}
    \begin{pmatrix}
        c\\
        c^\dagger
    \end{pmatrix}
    =\mathcal W^\dagger \begin{pmatrix}
        c\\
        c^\dagger
    \end{pmatrix},
\end{equation}
where the matrix of matrices $\mathcal W$ reads
\begin{equation}
    \label{eq:bogoliubov_mat}
    \mathcal W = \begin{pmatrix}
        U & V^*\\
        V & U^*
    \end{pmatrix}.
\end{equation}
Taking the product $\mathcal W^\dagger \mathcal W$ and imposing separate fermionic commutation relations of the operators $\beta, \beta^\dagger, c, c^\dagger$, we get that $\mathcal W$ is unitary, hence
\begin{equation}
    \mathcal W^\dagger \mathcal W = \mathcal W \mathcal W^\dagger =  I.
\end{equation}
We can now invert equation \eqref{eq:bogoliubov_trans_mat} by multiplying both sides on the left by $\mathcal W$, which yields 
\begin{equation*}
    \mathcal W \begin{pmatrix}
        \beta\\
        \beta^\dagger
    \end{pmatrix}
    =\begin{pmatrix}
        c\\
        c^\dagger
    \end{pmatrix}.
\end{equation*}
Using the Messiah-Bloch decomposition \cite{blochmessiah}, we can write the unitary matrix $\mathcal W$ as
\begin{equation}
    \label{eq:decomposition}
    \mathcal W = \begin{pmatrix}
        D & 0 \\
        0 & D^*
    \end{pmatrix}
    \begin{pmatrix}
        \overline U & \overline V \\
        \overline V & \overline U
    \end{pmatrix}
    \begin{pmatrix}
        C & 0 \\
        0 & C^*
    \end{pmatrix}
\end{equation}
where $D$ and $C$ are unitary matrices and $\overline U$ and $\overline V$ are real matrices, which have a particular blocked form, expressed through the coefficents $u_k, v_k$; the reader may refer to appendix \ref{app:uv} for the explicit representation. We can also define the matrices $U, V$ as 
\begin{equation}
    U=D\overline U C,\quad V=D^*\overline V C.
\end{equation}
Using the decomposition \eqref{eq:decomposition} we can define the \textit{canonical basis} as
\begin{equation}
    \label{eq:canonical_basis}
    a_k^\dagger = \sum_l D_{lk}^\dagger c_l^\dagger,
\end{equation}
a \textit{special Bogoliubov transformation} between \textit{paired} levels as
\begin{align}
    \alpha_k^\dagger = u_k a_k^\dagger - v_k a_{\tilde k},\\
    \alpha_{\tilde k } ^\dagger = u_k a_{\tilde k} ^\dagger + v_k a_k,
\end{align}
and \textit{blocked} levels
\begin{align}
\alpha_i &= a_i,\quad \alpha_n^\dagger = a_n^\dagger
\\\alpha_i &= a_i^\dagger, \quad \alpha_n = a_n,
\end{align}
where $u_k=u_{\tilde k},\ v_k=-v_{\tilde k}$, and a unitary transformation of the quasi-particle operators $\alpha_k^\dagger$ among themselves
\begin{equation}
    \beta_k^ \dagger = \sum_{k'}C_{k'k}a_{k'}^\dagger.
\end{equation}
We are now able to define the Bogoliubov ground state $\ket{\text{HFB}}$, as the one for which
\begin{equation}
    \beta_k \ket{\text{HFB}} = 0 \ \forall k = 1,\ldots,M
\end{equation}
where $M$ is determined by the physical situation \cite{ring2004nuclear}.
The wavefunction that satisfies this condition reads
\begin{equation}
\ket{\text{HFB}}=\prod_k^M\beta_k \ket{-}.
\end{equation}
We can define the pairing tensor as
\begin{equation}
    \kappa_{ll'} = \bra{\text{HFB}}c_{l'}c_l\ket{\text{HFB}},
\end{equation}
which in matrix form reads, alongside the density matrix
\begin{equation}
    \kappa = UV^\dagger,\quad \rho = V^*V^T.
\end{equation}
We can now apply the variational principle \eqref{eq:var_eq_res}
\begin{equation}
    \label{eq:varhfb}
    \delta \frac{\bra{\text{HFB}} \hat H-\lambda \hat N\ket{\text{HFB}}}{\bra{\text{HFB}}\ket{\text{HFB}}} = 0.
\end{equation}
which yields the eigenvalue problem
\begin{equation}
    \label{eq:eighfb}
    \begin{pmatrix}
    h -\lambda& \Delta \\ -\Delta^* & -(h-\lambda)^*
    \end{pmatrix}
    \begin{pmatrix}
        U_k \\ V_k
    \end{pmatrix}
    =\mathcal H_\text{HFB}\begin{pmatrix}U_k \\V_k\end{pmatrix}= E_k \begin{pmatrix}
        U_k \\ V_k
    \end{pmatrix},
\end{equation}
Here, $h$ is the single-particle Hamiltanian, which reads
\begin{equation}
    h_{kk'} = t_{kk'} +\Gamma_{kk'},
\end{equation}
where $\Gamma_{kk'}$ is the mean field potential, given by
\begin{equation}
    \label{eq:mean_field_hfb}
    \Gamma_{kk'} = \sum_{ll'}\overline{v}_{kl'k'l}\rho_{ll'}
    \end{equation}
and the pairing field $\Delta$ reads
\begin{equation}
    \label{eq:pairing_field_hfb}
    \Delta_{kk'} = \sum_{ll'}\overline{v}_{kk'll'}\kappa_{ll'}.
\end{equation}
In the canonical basis, we are able to solve for the occupation numbers
\begin{equation}
    \label{eq:occ_hfb}
    u_k^2 = \frac 1 2 \bigg(1+\frac{h_{kk}+h_{\tilde k \tilde k}}{\sqrt{(h_{kk}+h_{\tilde k \tilde k})^2+4\Delta_{k\tilde k}^2}}\bigg)
\end{equation}
where $v_k^2 = 1 - u_k^2$ is guaranteed by the unitarity of the matrices.
Starting from an initial guess, we solve the eigenvalue problem \eqref{eq:eighfb}, we extract the occupation numbers \eqref{eq:occ_hfb}, use them to build the new mean field \eqref{eq:mean_field_hfb} and pairing field \eqref{eq:pairing_field_hfb}, and repeat the process until convergence.
\paragraph{HFB quasi-particle spectrum} Let us assume that $\Psi = (U,\  V )^T$ is a solution of equation \eqref{eq:eighfb} with eigenvalue $E$
\begin{equation}
    \label{eq:eigsolhfb}
    \mathcal H_\text{HFB}\Psi = E\Psi.
\end{equation}
Let the particle-hole matrix $\mathcal C$ be defined as
\begin{align}
    \mathcal C = \begin{pmatrix}
        0 & I \\ I&0
    \end{pmatrix},
\end{align}
it's trivial to show that 
\begin{equation}
    \label{eq:hfb_anticomm}
    \mathcal C \mathcal H_\text{HFB} \mathcal C = -\mathcal H_\text{HFB}^*,
\end{equation}
and 
\begin{equation}
    \label{eq:commutationhfb}
    \mathcal C = \mathcal C ^{-1}\implies \mathcal C\mathcal H_\text{HFB} = -\mathcal H_\text{HFB}^*\mathcal C.
\end{equation}
If we take the complex conjugate of equation \eqref{eq:eigsolhfb}, we get
\begin{equation}
    \label{eq:eigsolhfb_conj}
    \mathcal H_\text{HFB}^*\Psi ^* = E\Psi^*,
\end{equation}
if we multiply both sides on the left by $\mathcal C$ and use \eqref{eq:commutationhfb}, we get
\begin{align}
    -\mathcal H_\text{HFB}\mathcal C\Psi ^* &= E\mathcal C\Psi^*,\\
    \mathcal H_\text{HFB}\mathcal C\Psi ^* &= -E\mathcal C\Psi^*,
\end{align}
meaning that $\mathcal C \Psi^*$ is a solution of the eigenvalue problem \eqref{eq:eighfb} as well, with eigenvalue $-E$, hence for every quasi-particle energy we have a corresponding opposite-sign one; moreover, it can be proven that the HFB hamiltonian is unbounded, both from below and above \cite{Pei2012_HFBcontinuum}. This feature poses a challenge for numerical solutions of the HFB problem, as we shall see in chapter \ref{chap:numerical}.




\section{Skyrme force and functional}
Now that the theoretical and numerical framework is clear, we can investigate a plausible nucleonic interaction, which in the present work, takes the form of the Skyrme interaction.
\\It was first proposed by Tony Skyrme in 1958 \cite{SKYRME1958615} as a zero range force between nucleons, and has been used successfully as the building block of nuclear structure.
\\Nowadays, the standard form is slightly enriched to be more general \cite{CHABANAT1997710}. It comprises a two-body interaction, which reads
\begin{align}
v^{(2)}(\mathbf{r}_1, \mathbf{r}_2) &= t_0 \left(1 + x_0 P_\sigma \right) \delta(\mathbf{r}) \\
&\quad + \frac{1}{2} t_1 \left(1 + x_1 P_\sigma \right) \left[ \mathbf{P}'^2 \delta(\mathbf{r}) + \delta(\mathbf{r}) \mathbf{P}^2 \right] \\
&\quad + t_2 \left(1 + x_2 P_\sigma \right) \mathbf{P}' \cdot \delta(\mathbf{r}) \mathbf{P} \\
&\quad + \frac{1}{6} t_3 \left(1 + x_3 P_\sigma \right) \left[ \rho(\mathbf{R}) \right]^\sigma \delta(\mathbf{r}) \\
&\quad + i W_0 \boldsymbol{\sigma}\cdot \left[ \mathbf{P}' \times \delta(\mathbf{r}) \mathbf{P} \right]
\end{align}
And a three body interaction, that is
\begin{equation}
v^{(3)}(\mathbf r_1, \mathbf r_2)=\frac 1 6 t_3 \left(1 + x_3 P_\sigma \right) \left[ \rho(\mathbf{R}) \right]^\sigma \delta(\mathbf{r}) 
\end{equation}
Where 
\begin{align*}
\\\mathbf{r} &= \mathbf{r}_1 - \mathbf{r}_2
\\\mathbf{R} &= \frac{\mathbf{r}_1+\mathbf{r}_2}{2}
\\\mathbf{P} &= \frac{-i(\nabla_1 - \nabla_2)}{2}
\\\boldsymbol{\sigma} &= \boldsymbol{\sigma}_1 + \boldsymbol{\sigma}_2
\\\mathbf{P}_\sigma &= \frac{(1+\boldsymbol{\sigma}_1\cdot\boldsymbol{\sigma}_2)}{2}
\end{align*}
Primed operators refer to the complex conjugate acting on the bra space.
\\This formulation respects all symmetries required of a non relativistic nuclear interaction (Galilean boost, particle exchange, translation, rotation, parity, time reversal and translation).
\\Taking the expectation value of the many body hamiltonian, in the Hilbert space of Slater determinants, yields
\begin{equation}
    \expval{H} = \bra{\Psi} H \ket{\Psi} = \int (\mathcal E_\text{Skyrme} + \mathcal E_\text{Kin}) d\mathbf r
\end{equation}
In the case of even-even nuclei, time-odd components of the functional reduce to zero, leaving \cite{stevenson2019low}
\begin{align}
    \mathcal E_\text{Kin} &= \frac{\hbar^2}{2m}\tau \label{eq:kinfunc}\\
    \mathcal E_\text{Skyrme} &= \sum_{t=0,1}\bigg\{C_t^\rho [\rho_0]\rho_t^2+C_t^{\Delta \rho}\rho_t\nabla^2\rho_t+C_t^{\nabla\cdot J}\rho_t\nabla\cdot \mathbf J_t + C_t^\tau\rho_t\tau_t\bigg\}\label{eq:skfunc}
\end{align}
Here, $t=0,1$ refers to the isoscalar and isovector components of the densities, e.g.
\begin{align*}
    \rho_0 = \rho_p - \rho_n
    \\\rho_1 = \rho_p + \rho_n
\end{align*}
Where
\begin{align}
    C_0^\rho &= +\frac 3 8 t_0 + \frac 3 {48} t_3\rho_0^\sigma 
    \\C_1^\rho &= -\frac 1 8 t_0(1+2x_0)- \frac 1 {48} t_3(1+x_3)\rho_0^\sigma 
    \\C_0^\tau &= +\frac 3 {16} t_1 + \frac 1 {16} t_2 (5+4x_2)
    \\C_1^\tau &= -\frac 1 {16} t_1(1+2x_1)+\frac 1 {16}t_2(1+2x_2)
    \\C_0^{\Delta \rho} &= -\frac 9 {64}t_1+\frac 1 {64}t_2(5+4x_2)
    \\C_1^{\Delta \rho} &= +\frac 3 {64}t_1(1+2x_1)+\frac 1 {64}t_2(1+2x_2)
    \\C_0^{\nabla\cdot J} &= -\frac 3 4 W_0
    \\C_1^{\nabla\cdot J} &= -\frac 1 4 W_0
\end{align}
As outlined in previous chapters (REF), we can now derive the Kohn-Sham equations, by constraining orthonormality and enforcing the variation of the functional to be zero. What we end up with is
\begin{equation}
    \bigg[-\nabla\bigg(\frac{\hbar^2}{2m^{*}_q(\mathbf r)}\nabla \bigg) + U_q(\mathbf r) + \delta_{\text{q,proton}}U_C(\mathbf r)-i\mathbf B_q(\mathbf r)\cdot(\nabla \times \boldsymbol\sigma) \bigg]\varphi_\alpha=\varepsilon_\alpha\varphi_\alpha
\end{equation}
The index $q=n,p$ refers respectively to the neutron and proton quantites.
\\Where the different terms are given by
\begin{align}
    \frac{\hbar^2}{2m^{*}_q(\mathbf r)} &= \frac{\hbar^2}{2m}+\fdv{\mathcal H}{\tau_q}
    \\U_q(\mathbf r) &= \fdv{\mathcal H}{\rho_q}
    \\\mathbf B_q(\mathbf r) &= \fdv{\mathcal H}{\boldsymbol{\mathbf J_q}}
\end{align}
The coulomb field $U_C$, which is present only in the single particle equation for protons, doesn't come from the skyrme interaction, but from the Coulomb part of the complete functional. It will be properly derived in section (REF).
\\Following the rules for functional derivatives, outlined in the appendix (REF) for our particular case, we have
\begin{align}
    \frac{\hbar^2}{2m_q^*(\mathbf r)} =& +\frac{\hbar^2}{2m} \\&+ \frac 1 8 [t_1(2+x_1)+t_2(2+x_2)]\rho(\mathbf r) \\&- \frac 1 8 [t_1(1+2x_1)+t_2(1+2x_2)]\rho_q(\mathbf r ) \\\\
    U_q(\mathbf r) =& +\frac 1 8 [t_1(2+x_1)+t_2(2+x_2)]\rho \\&+ \frac 1 8 [t_2(1+2x_2)-t_1(1+2x_1)]\rho_q \\
    &+ \frac 1 8 [t_1(2+x_1)+t_2(2+x_2)]\tau \\&+ \frac 1 8 [t_2(1+2x_2)-t_1(1+2x_1)]\tau_q \\
    &+ \frac 1 {16} [t_2(2+x_2)-3t_1(2+x_1)] \nabla^2 \rho \\&+ \frac 1 {16} [3t_1(2x_1+1)+t_2(2x_2+1)] \nabla^2 \rho_q \\\\
    \mathbf W_q (\mathbf r ) = &+\frac 1 2 W_0 [\nabla\rho + \nabla \rho_q] \\&-\frac 1 8 (t_1 x_1 + t_2 x_2) \mathbf J + \frac 1 8 (t_1 - t_2) \mathbf J_q 
\end{align}
For ease of notation and implementation, unindexed densities refer to isovector quantites.
\subsection{Energy density functional}
The energy functional to be minimized is of the form \cite{Bender2003}
\begin{equation}
\label{eq:full_functional}
E_{\text{HF}} =  E_\text{Kin}+E_\text{Skyrme}+E_\text{Coul} = \int( \mathcal E_\text{Kin} + \mathcal E_\text{Skyrme} + \mathcal E_\text{Coul})d\bm r.
\end{equation}
\subsubsection{Densities}
Functional \eqref{eq:full_functional} can be expressed through a series of particle densities. Let us define them and express them on the spin coordiantes up ($\uparrow$) and down ($\downarrow$) for the convinience in a mesh representation.
\\The starting point is the density matrix, defined as
\begin{equation}
    \rho_q (\mathbf r \sigma, \mathbf r \sigma') = \sum_{\alpha} \phi_{\alpha, \sigma} (\mathbf r )\phi_{\alpha, \sigma'}^*(\mathbf r')
\end{equation}
where the index $\alpha$ goes through all single particle states of the particles of type $q$ (protons, neutrons) and the index $\sigma$ refers to the spin coordinate. The particle density is defined as 
\begin{align}
    \rho_q(\mathbf r) \coloneq\rho_q(\mathbf r, \mathbf r')\bigg|_{\mathbf{r} = \mathbf{r'}} \coloneq \sum_{\sigma}\rho(\mathbf r\sigma, \mathbf r'\sigma) \bigg|_{\mathbf{r} = \mathbf{r'}} &=\sum_{\alpha} \phi_{\uparrow}(\mathbf r)\phi_{\uparrow}^*(\mathbf r')+\phi_{\downarrow}(\mathbf r)\phi_{\downarrow}^*(\mathbf r') \bigg|_{\mathbf{r} = \mathbf{r'}} \nonumber
    \\&=\sum_{\alpha} |\phi_{\uparrow}(\mathbf r)|^2+|\phi_{\downarrow}(\mathbf r)|^2.
    \label{eq:part_density}
\end{align}
The kinetic density reads
\begin{align}
    \tau_q(\mathbf r) &\coloneq \sum_{\alpha} \nabla'\cdot\nabla\rho_q(\mathbf r, \mathbf r')\bigg|_{\mathbf r'=\mathbf r} \nonumber
    \\&= \sum_{\sigma, \alpha} \nabla \phi_\sigma (\mathbf r)\cdot \nabla \phi_\sigma^*(\mathbf r')\bigg|_{\mathbf r = \mathbf r'} = \sum_{\sigma, \alpha} |\nabla \phi_\sigma(\mathbf r)|^2 \nonumber
    \\&= \sum_{\alpha}|\nabla \phi_\uparrow(\mathbf r)|^2 + |\nabla \phi_\downarrow(\mathbf r)|^2 \label{eq:kin_density}.
\end{align}
The spin density  reads
\begin{align}
    s_q(\mathbf r, \mathbf r') &\coloneq\sum_{\sigma \sigma', i} \rho_q(\mathbf r \sigma, \mathbf r' \sigma')\bra{\sigma'} \hat {\boldsymbol{\sigma}} \ket{\sigma} = \sum_{\alpha} \begin{bmatrix} \phi_{\uparrow}^*(\mathbf r') \ \phi_{\downarrow}^*(\mathbf r') \end{bmatrix}\hat{\boldsymbol{\sigma}} \begin{bmatrix} \phi_{\uparrow}(\mathbf r) \\ \phi_{\downarrow}(\mathbf r) \end{bmatrix}
\end{align}
and lastly, the spin-orbit density tensor reads
\begin{align}
    J_{q, \mu\nu} &\coloneq \frac 1 {2i}(\partial_\mu - \partial_\mu') s_{q, \nu}(\mathbf r, \mathbf r')\bigg|_{\mathbf r'=\mathbf r}\nonumber \\
    &= \frac 1 {2i}\bigg(\begin{bmatrix}\phi_{\uparrow}^*(\boldsymbol r')\ \phi_{\downarrow}^*(\boldsymbol r')\end{bmatrix} \partial_\mu\hat{\sigma}_\nu\begin{bmatrix} \phi_{\uparrow}(\mathbf r) \\ \phi_{\downarrow}(\mathbf r) \end{bmatrix} - \begin{bmatrix}\phi_{\uparrow}(\boldsymbol{r})\ \phi_{\downarrow}(\boldsymbol{r})\end{bmatrix} \partial_\mu'\hat{\sigma}_\nu\begin{bmatrix} \phi_{\uparrow}^*(\mathbf r') \\ \phi_{\downarrow}^*(\mathbf r') \end{bmatrix}\bigg)_{\mathbf r'=\mathbf r}\nonumber
     \\&= \sum_\alpha\Im\bigg\{\begin{bmatrix}\phi_{\uparrow}^*(\boldsymbol r)\ \phi_{\downarrow}^*(\boldsymbol r) \end{bmatrix}\partial_\mu \hat{\sigma}_\nu\begin{bmatrix} \phi_{\uparrow}(\mathbf r) \\ \phi_{\downarrow}(\mathbf r) \end{bmatrix}\bigg\}
\end{align}
which also defines the spin-orbit current vector $\bm J$ that reads
\begin{equation}
     J_{q,\kappa} (\bm r ) = \sum_{\mu\nu}\epsilon_{\kappa\mu\nu} J_{q, \mu\nu}(\bm r).
\end{equation}
\subsubsection{Kinetic functional}
The kinetic term can be expressed as
\begin{equation}
    \label{eq:kinfunc}
    \mathcal E_\text{Kin} = \frac{\hbar^2}{2m}\tau
\end{equation}
which is found integrating by parts \eqref{eq:kin_functional}.
\subsubsection{Skyrme functional}
Since this work only deals with even-even nuclei, only time-even densities, which are the ones previously defined, are non-vanishing, due to the ground state being time-reversal invariant \cite{Bender2003}. This reduces the Skyrme functional to the following form \cite{stevenson2019low}
\begin{align}
    \mathcal E_\text{Skyrme} &= \sum_{t=0,1}\bigg\{C_t^\rho [\rho_0]\rho_t^2+C_t^{\Delta \rho}\rho_t\nabla^2\rho_t+C_t^{\nabla\cdot J}\rho_t\nabla\cdot \mathbf J_t + C_t^\tau\rho_t\tau_t\bigg\}\label{eq:skfunc}
\end{align}
where
\begin{align}
    %\label{eq:coefficients_func}
    C_0^\rho &= +\frac 3 8 t_0 + \frac 3 {48} t_3\rho_0^\sigma \label{eq:C0rho}
    \\C_1^\rho &= -\frac 1 8 t_0(1+2x_0)- \frac 1 {48} t_3(1+x_3)\rho_0^\sigma \label{eq:C1rho}
    \\C_0^\tau &= +\frac 3 {16} t_1 + \frac 1 {16} t_2 (5+4x_2) \label{eq:C0tau}
    \\C_1^\tau &= -\frac 1 {16} t_1(1+2x_1)+\frac 1 {16}t_2(1+2x_2) \label{eq:C1tau}
    \\C_0^{\Delta \rho} &= -\frac 9 {64}t_1+\frac 1 {64}t_2(5+4x_2) \label{eq:C0Deltarho}
    \\C_1^{\Delta \rho} &= +\frac 3 {64}t_1(1+2x_1)+\frac 1 {64}t_2(1+2x_2) \label{eq:C1Deltarho}
    \\C_0^{\nabla\cdot J} &= -\frac 3 4 W_0 \label{eq:C0nabladotJ}
    \\C_1^{\nabla\cdot J} &= -\frac 1 4 W_0 \label{eq:C1nabladotJ}.
\end{align}
Here, $t=0,1$ refers to the isoscalar and isovector components of the densities, eg
\begin{align*}
    \rho_0 = \rho_p + \rho_n
    \\\rho_1 = \rho_p - \rho_n.
\end{align*}
We can now derive the Kohn-Sham equations, by minimizing the functional under the constraint
\begin{equation}
    \label{eq:spe_ks_constraint}
    \bra{\varphi_i}\ket{\varphi_j}=\delta_{ij}.
\end{equation}
The resulting Kohn-Sham equations are of the form
\begin{equation}
    \label{eq:spe_ks}
    \bigg[-\nabla\bigg(\frac{\hbar^2}{2m^{*}_q(\mathbf r)}\nabla \bigg) + U_q(\mathbf r) + \delta_{\text{q,proton}}U_C(\mathbf r)-i\mathbf B_q(\mathbf r)\cdot(\nabla \times \boldsymbol\sigma) \bigg]\varphi_\alpha=\varepsilon_\alpha\varphi_\alpha
\end{equation}
where an effective mass field arises, which is defined as
\begin{equation}
    \frac{\hbar^2}{2m^{*}_q(\mathbf r)} = \fdv{\mathcal E}{\tau_q}
\end{equation}
a mean field potential, which reads
\begin{equation}
    U_q(\mathbf r) = \fdv{\mathcal E}{\rho_q}\label{eq:fdv_rho_skyrme}
\end{equation}
and a spin-orbit field, given by
\begin{equation}
    \mathbf B_q(\mathbf r) = \fdv{\mathcal E}{\boldsymbol{\mathbf J_q}}.
\end{equation}
The coulomb field $U_C$, which is present only in the single particle equation for protons, doesn't come from the Skyrme interaction, rather from the Coulomb part of the whole functional. It will be properly derived in section \ref{sec:coulomb_treatment}.
\\Following the rules for functional derivatives, outlined in the appendix \ref{app:func_der} we get
\begin{align}
    \frac{\hbar^2}{2m_q^*(\mathbf r)} =& +\frac{\hbar^2}{2m} \nonumber
    \\&+ \frac 1 8 [t_1(2+x_1)+t_2(2+x_2)]\rho(\mathbf r) \nonumber
    \\&- \frac 1 8 [t_1(1+2x_1)+t_2(1+2x_2)]\rho_q(\mathbf r ) \\\nonumber
\end{align}
\begin{align}
    U_q(\mathbf r) =& +\frac 1 8 [t_1(2+x_1)+t_2(2+x_2)]\rho \nonumber
    \\&+ \frac 1 8 [t_2(1+2x_2)-t_1(1+2x_1)]\rho_q \nonumber
    \\&+ \frac 1 8 [t_1(2+x_1)+t_2(2+x_2)]\tau \nonumber
    \\&+ \frac 1 8 [t_2(1+2x_2)-t_1(1+2x_1)]\tau_q \nonumber
    \\
    &+ \frac 1 {16} [t_2(2+x_2)-3t_1(2+x_1)] \nabla^2 \rho \nonumber
    \\&+ \frac 1 {16} [3t_1(2x_1+1)+t_2(2x_2+1)] \nabla^2 \rho_q \\\nonumber
\end{align}
\begin{align}
    \mathbf B_q (\mathbf r ) = &+\frac 1 2 W_0 [\nabla\rho + \nabla \rho_q] \nonumber\\
    &-\frac 1 8 (t_1 x_1 + t_2 x_2) \mathbf J + \frac 1 8 (t_1 - t_2) \mathbf J_q.
\end{align}
Unless otherwise specified, unlabelled densities denote isoscalar quantities (sum of neutron and proton).
\subsection{Funcitonals}
WRITE ABOUT DIFFERENT FUNCTIONALS FITS.
\section{Coulomb interaction}
\label{sec:coulomb_treatment}
Unlike the Skyrme interaction, the Coulomb force is finite-range, giving rise to an unwanted integral operator in the single-particle Hamiltonian.
A well known and widely used device is the Slater approximation \cite{SlaterApp}, which gives a local exchange interaction.
\\In this approximation, the Coulomb energy reads
\begin{align*}
    E_\text{Coul} = \int \mathcal E_\text{Coul}(\mathbf r) d\bm r
\end{align*}
where the energy density is given by
\begin{align}
    \mathcal E_\text{Coul}(\bm r) = \frac{e^2}{2}\bigg[\int  \frac{\rho_p(\mathbf r )\rho_p(\mathbf r ' )}{|\mathbf r-\mathbf r'|}d\mathbf r'  - \frac 3 2 \bigg(\frac 3 \pi \bigg) ^{\frac 1 3}\rho_p^{4/3}(\mathbf r)\bigg].
\end{align}
which results in the Coulomb potential field
\begin{equation}
    U_{C}(\mathbf r) = \fdv{\mathcal E_\text{Coul}}{\rho_p} = \frac{e^2}{2}\bigg[\int \frac{\rho_p(\mathbf r ')}{|\mathbf r-\mathbf r'|} d^3 \mathbf r' - 2\bigg(\frac 3 \pi \bigg) ^{\frac 1 3} \rho_p^{1/3}(\mathbf r ) \bigg]
\end{equation}
where the first term is the direct Coulomb interaction, which simply is the Coulomb energy generated by the proton density, while the second term is the exchange Coulomb interaction, which is local and depends on the proton density through a power factor of $1/3$.
From a computational standpoint, the exchange part is trivial, while the direct one is more involved.
One could compute the integral, but the complexity on a 3D mesh grows as $\mathcal O(N^6)$, where N is the total number of points on the mesh, rendering it unfeasible for fine calculations. 
\\An alternative approach is to solve the Poisson equation (from now on, $V_c$ refers to the direct part only)
\begin{equation}
    \label{eq:poisson}
    \nabla^2 V_c = 4\pi e^2 \rho_p.
\end{equation}
Given the proton density, we can impose Dirichlet boundary conditions, which can be extracted from a quadrupole expansion of the charge density \cite{Jackson1998}
\begin{equation}
V_c (\mathbf r) = 4\pi e^2 \sum_{\lambda=0}^2\sum_{\mu=-\lambda}^\lambda \frac{\expval{Q_{\lambda\mu}} Y_{\lambda\mu}}{r^{1+\lambda}}\text{ on }\partial \Omega
\end{equation}
where $\expval{Q_{\lambda\mu}}$ is defined as 
\begin{equation}
    \expval{Q_{\lambda\mu}} = \int r^\lambda Y_{\lambda\mu}^* (\mathbf r)\rho_p(\mathbf r ) d^3 \mathbf r
\end{equation}
Since we expect a charge density confined to the nuclear shape, higher order terms in the expansion can be neglected, provided that the box is sufficiently large.
\\In a reference frame where the nucleus center of mass is at the origin, the expansion reduces to
\begin{equation}
    V_{c}(\mathbf r ) = \frac{Ze^2}{r} + e^2\sum_{\mu=-2}^{2}\frac{\expval{Q_{2\mu}}Y_{2\mu}}{r^3} \text{ on } \partial \Omega.
\end{equation}
The reader can refer to appendix \ref{sec:spherical_harmonics} for the definition and numerical evaluation of the spherical harmonics $Y_{\lambda\mu}$.
\section{Energy calculation}
One, if not the most important physical quantity we want to compute is the total energy of the system.
\subsubsection{Integrated energy}
The obvious way would be to evaluate the functional for a given density. We will call this \textit{integrated energy}.
\begin{align*}
    E_\text{int} = E[\rho, \tau, J_{\mu\nu}]= \int \mathcal E d\mathbf r
\end{align*}
\subsubsection{Hartree-Fock energy}
An alternative approach can be used, as in a stationary point $\delta E = 0$, the single particle eigenvalue equation \eqref{eq:spe_ks} stands true, summarized as 
\begin{align}
    \label{eq:simple_spe}
    (\hat t + U)\varphi_k = \varepsilon_k \varphi_k
\end{align}
We can multiply \eqref{eq:simple_spe} on the left by $\varphi_k^*$ and integrate to get
\begin{align}
    \label{eq:int_simple_spe}
    \int -\varphi_k^* \frac{\hbar^2}{2m}\nabla^2\varphi_k d\bm r + \int \varphi_k^* U \varphi_k d\bm r = \int \varphi_l^* \varepsilon_k \varphi_k d\bm r
\end{align}
The integral on the right hand side of \eqref{eq:int_simple_spe} evaluates to $\varepsilon_k$ due to the orthonormality constraint. If we sum over all states $k$ we get
\begin{align}
    \sum_k \bigg\{\int -\varphi_k^* \frac{\hbar^2}{2m}\nabla^2\varphi_k d\bm r + \int \varphi_k^* U \varphi_k d\bm r \bigg\}= \sum_k \varepsilon_k 
    \\\sum_k t_k + \int \rho U = \sum_k \varepsilon_k \label{eq:sum_spe}
\end{align}
Since $U$ is calculated as \eqref{eq:fdv_rho_skyrme}, assuming that the functional has a power dependence from $\rho$ of the form $\mathcal E_\text{Skyrme} = A\rho^{\sigma+1}$ as in our case, we get the \textit{rearrangement energy}
\begin{align}
    \label{eq:sum_spe_new}
    \rho U = \rho \fdv{\mathcal E_\text{Skyrme}}{\rho} = \rho(\sigma +1)A \rho^\sigma = (\sigma+1)A \rho^{\sigma+1} = \mathcal E_\text{Skyrme} + \sigma \mathcal E_\text{Skyrme} = \mathcal E_\text{Skyrme} - \mathcal E_\text{rea}
\end{align}
If we explicit $\rho U$ in equation \eqref{eq:sum_spe} using \eqref{eq:sum_spe_new}, we get to
\begin{align*}
    \sum_k t_k + \int (\mathcal E_\text{Skyrme}-\mathcal E_\text{rea}) d\bm r = \sum_k \varepsilon_k 
\end{align*}
Isolating the Skyrme energy density
\begin{align}
    \label{eq:int_E_rea}
    \int \mathcal E_\text{Skyrme} d\bm r = \sum_k (\varepsilon_k -t_k) + \int \mathcal E_\text{rea} d\bm r
\end{align}
and given the total energy of the system from \eqref{eq:slater_exp}
\begin{align}
    \label{eq:brief_tot_energy}
E=\sum_k t_k + \frac 1 2 \int \mathcal E_\text{Skyrme} d\bm r 
\end{align}
substituting \eqref{eq:int_E_rea} in \eqref{eq:brief_tot_energy} yields
\begin{align}
    \label{eq:hf_energy}
E_\text{HF} = \frac 1 2 \sum_k (\varepsilon_k + t_k) +\int \mathcal E_\text{rea} d\bm r = \frac 1 2 \bigg(T+\sum_k\varepsilon_k\bigg) +E_\text{rea}
\end{align}
which will be called \textit{Hartree-Fock energy} throughout this text.
\paragraph{Sidenote:}
The actual functional has a plethora of $\rho$ terms, which can be summarized as
\begin{equation*}
    \mathcal E_\text{Skyrme} = \sum_j A_j \rho^{\sigma_j+1} \implies E_\text{rea} = -\sum_j \sigma_j A_j \rho^{\sigma_j+1} 
\end{equation*}
This means that only terms with a $\sigma_j\neq 0, -1$ actually contribute to the rearrangement energy.
\\Since equation \eqref{eq:hf_energy} is valid only for $\delta E = 0$, it's useful to check its equivalence with the integrated energy at convergence, so one can be sure to actually be in a stationary point.
\addtocontents{toc}{\vspace{2em}} % Add a gap in the Contents, for aesthetics
\bibliography{Thesis_bibliography} % The references information are stored in the file named "Thesis_bibliography.bib"

%-------------------------------------------------------------------------
%	APPENDICES
%-------------------------------------------------------------------------

\cleardoublepage
\addtocontents{toc}{\vspace{2em}} % Add a gap in the Contents, for aesthetics
\appendix
\chapter{Appendix A}
If you need to include an appendix to support the research in your thesis, you can place it at the end of the manuscript.
An appendix contains supplementary material (figures, tables, data, codes, mathematical proofs, surveys, \dots)
which supplement the main results contained in the previous chapters.

\chapter{Appendix B}
It may be necessary to include another appendix to better organize the presentation of supplementary material.


% LIST OF FIGURES
\listoffigures

% LIST OF TABLES
\listoftables

% LIST OF SYMBOLS
% Write out the List of Symbols in this page
\chapter*{List of Symbols} % You have to include a chapter for your list of symbols (
\begin{table}[H]
    \centering
    \begin{tabular}{lll}
        \textbf{Variable} & \textbf{Description} & \textbf{SI unit} \\\hline\\[-9px]
        $\bm{u}$ & solid displacement & m \\[2px]
        $\bm{u}_f$ & fluid displacement & m \\[2px]
    \end{tabular}
\end{table}

% ACKNOWLEDGEMENTS
\chapter*{Acknowledgements}
Here you might want to acknowledge someone.

\cleardoublepage

\end{document}

\section{Finite differences}
\label{sec:finite_diff}
The framework used to numerically solve the relevant PDEs of the problem, is the one of finite differences. The core idea is to discretize the domain on a 3D mesh, use Taylor expansions to approximate differential operators and then solve the resulting system of linear equations.
\subsection{3D mesh}
The first step of the process is representing the different fields in a numerical, discretized fashon.
Generally speaking, we deal at most with 2-rank tensors, which vary in space and spin. 
\\Discretizing the 3D cartesian space with a 3-index mesh, choosing a box which size on x, y, z is respectively $[-a_x, a_x]$, $[-a_y, a_y]$, $[-a_z, a_z]$, and a number of points $N_x, N_y, N_z$, the resulting lattice will be given by
\begin{equation*}
V=\{(-a_x+ih_x, -a_y+ih_y, -a_z+ih_z)\} = \{(x_i, y_j,z_k)\} 
\end{equation*}
Where the indices and step sizes are
\begin{align*}
  i&=0,\ldots,N_x-1\quad h_x = \frac{2a_x}{N_x-1}\\
  j&=0,\ldots,N_y-1\quad h_y = \frac{2a_y}{N_y-1}\\
  k&=0,\ldots,N_z-1\quad h_z = \frac{2a_z}{N_z-1}
\end{align*}
For ease of notation, we will assume $a=a_x=a_y=a_z=a$ and $N=N_x=N_y=N_z=N$, without losing generality.
\\Including the spin degree of freedom, we can finally represent fields in a numerical way through
\begin{equation}
    \varphi(\bm r, \sigma) \mapsto \varphi(x_i, y_j, z_k, s) = \varphi_{ijks}
\end{equation}
\subsubsection{Differential operators discretization}
By using Taylor series, it's possible to write approximations to derivatives \cite{Zhou1993}, in any point of the lattice, of any (reasonable) order of accuracy, involving only near neighbouring points. In the present work, 5-points derivates are used, meaning Taylor expansions are written for $\varphi(x\pm h)$ and $\varphi(x\pm 2h)$ to compute the differential operators. Formulae for first and second derivates are given in appendix \ref{app:5p_derivatives}.
\\From the theory background of chapter \ref{chap:hf}, we discern two main kinds of PDEs, the Schr\"odinger like KS equation, and the Poisson equation. Both are similar to a diffusion equation, but the former involves the spin degree of freedom, while the latter requires careful treatment of the boundary conditions.
\subsection{Schr\"odinger equation}
Starting from the Schr\"odinger equation \ref{eq:spe_ks}, reported here for clarity
\begin{equation*}
    \bigg[-\nabla\bigg(\frac{\hbar^2}{2m^{*}_q(\mathbf r)}\nabla \bigg) + U_q(\mathbf r) + \delta_{\text{q,proton}}U_C(\mathbf r)-i\mathbf B_q(\mathbf r)\cdot(\nabla \times \boldsymbol\sigma) \bigg]\varphi=\varepsilon\varphi
\end{equation*}
it can be compactly written as
\begin{equation}
    \label{eq:pde} f(\nabla^2 \varphi, \nabla \varphi, \varphi, \bm r, s) = \varepsilon\varphi.
\end{equation}
If $f$ is linear in $\varphi$, it would be possible to rewrite it as a linear combination of $\varphi$ on the mesh, after which we can use linear algebra methods to solve the problem.
\paragraph{Linearity}
Breaking down each part of the equation, the kinetic term
\begin{equation}
\label{eq:kin_lin_dim}
\nabla\bigg(\frac{\hbar^2}{2m^{*}_q(\mathbf r)}\nabla \bigg)\varphi = \frac{\hbar^2}{2m^{*}_q(\mathbf r)} \nabla^2\varphi + \nabla\bigg(\frac{\hbar^2}{2m^{*}_q(\mathbf r)}\bigg)\cdot \nabla\varphi
\end{equation}
Is evidently linear in $\varphi$. 
\\The spin-orbit coupling, which most generally reads
\begin{align*}
    \hat {{h}}_{\text{SO}}&=\bm f (\bm r )\cdot (\nabla \times \pauli ) 
    \\&=f_x(\bm r)(\sigma_z\partial_y - \sigma_y\partial_z) + f_y(\bm r)(\sigma_x\partial_z - \sigma_z\partial_x) + f_z(\bm r)(\sigma_y\partial_x - \sigma_x\partial_y)
\end{align*}
where $\hat {h}_\text{SO}$ acts on the spinor $\varphi_{ijk}$. Since $\bm \sigma$ is the vector of Pauli matrices acting as linear operators on $\varphi_{ijk}$, this portion of the equation is linear in $\varphi$.
\\Finally, the mean field terms $U_q, U_c$ 
\begin{equation*}
    (U_q + \delta_{\text{q,proton}}U_c)\varphi
\end{equation*}
are just multiplicative, hence linear.
\\Given that the whole equation is linear in $\varphi$, we can evaluate it on the chosen mesh, using finite differences to approximate the differential operators, yielding a linear eigenvalue problem of the form
\begin{align}
    \sum_n^{N_x\cdot N_y\cdot N_z \cdot 2} A_{mn}\varphi_{n} &= E\varphi_m
\end{align}
\subsubsection{Boundary conditions}
We expect the nucleus to be a localized object, prompting null Dirichlet boundary conditions for the Schr\"odinger equation.
Near the boundaries, the derivatives will involve points outside the box, setting these points to zero, is equivalent to solving 
\begin{equation}
    \begin{bmatrix}
        0 & 0 & 0 & 0 & 0
        \\0 & 0 & 0 & 0 & 0
        \\0 & 0 & A & 0 & 0
        \\0 & 0 & 0 & 0 & 0
        \\0 & 0 & 0 & 0 & 0
    \end{bmatrix}
    \begin{bmatrix}
        \varphi_{-2}\\\varphi_{-1}\\\varphi\\\varphi_{N}\\ \varphi_{N+1} 
    \end{bmatrix}
    = E 
    \begin{bmatrix}
        \varphi_{-2}\\\varphi_{-1}\\\varphi\\\varphi_{N}\\ \varphi_{N+1} 
    \end{bmatrix}
\end{equation}
From this system of equations, we get for points outside the boundary:
\begin{equation}
    \begin{cases}
        \varphi_{-2} = 0\\
        \varphi_{-1} = 0\\
        \ldots
    \end{cases}
\end{equation}
Meaning that $\varphi$ outside the box will automatically be set to zero if the $A$ matrix is built assuming those points to be zero when computing its coefficients.
\subsection{Poisson equation}
\label{subsec:poisson}
The other fundamental PDE we need to solve is the Poisson equation encountered in section (REF). Dropping the $c$ and $p$ subscripts, it reads
\begin{equation*}
\nabla^2 V = 4\pi e^2 \rho
\end{equation*}
It's simpler than the Schr\"odinger equation, as it only involves a laplacian and it's not an eigenvalue problem. The right is side is given, and the solution is found by inverting the coefficients' matrix.
\subsubsection{Boundary conditions}
Unlike the Schr\"odinger equation, we do not expect the solution to rapidly decay near the boundaries; as reported in section \ref{sec:coulomb_treatment}, we have fixed, non-null boundary conditions, which we have to properly impose on the system.
\\We can choose a direction, say $x$, and look at the discretized equation at the boundaries $x=\pm a$. Since the indeces $j,k$ won't vary, we can omit them, and ignore the other derivates in the following equations.
\begin{align}
    \label{eq:poisson_disc}
    \nabla^2 V &= \partial_{xx} V + \partial_{yy}V + \partial_{zz}V\nonumber \\
    &= \frac{-V_{i-2} + 16V_{i-1} -30V_i + 16V_{i+1} - V_{i+2}}{12h^2}+\cdots=4\pi e^2 \rho_i
\end{align}
Near a boundary, say $i = 0$, the formula calls for points outside the box, known as \textit{ghost points}. Since they are not part of the linear system, but they are known, we can bring them on the right side of equation \ref{eq:poisson_disc}.
\begin{align}
    \frac{-30 V_0 + 16 V_1 - V_2 }{12h^2} &= 4\pi e^2 \rho_0 +\frac{ V_{-2} -16V_{-1}}{12h^2} = \tilde\rho_0
\end{align}
The same procedure must be applied to all equations involving ghost points, e.g. for $i=1$
\begin{align}
    \frac{+16V_0 -30V_{1} + 16 V_2 - V_3}{12h^2} &= 4\pi e^2 \rho_1 + \frac{V_{-1} }{12h^2} = \tilde\rho_1.
\end{align}
The proper system to solve will then be
\begin{align}
A V = \tilde \rho
\end{align}
Where $A$ is constructed as previously specified. Solving with $\tilde \rho $ on the right hand side will force the solution to abide boundary conditions.
\subsubsection{On higher order approximations and performance}
Higher and higher order approximations for derivatives involve more points that are further away. This increases accuracy by reducing the finite differences error, but it also decreases matrix sparseness.
\\Algorithms like Conjugate Gradient, as we'll see in the next section, and linear algebra computations, benefit from matrix sparseness, since multiplication by zero is readily known. The implication is that performance \textit{vs} accuracy is a tradeoff that isn't univocal to every problem.
\\In the present work, the golden choice has been 5-point derivatives; but it's not definitive, as the lattice points rapidly cap depending on the system's memory.
\\As an example, take a seemingly harmless grid, made of 50 points in each direction. The resulting matrix will be $50\times 50 \times 50 \times 2=\num{2.5e5}$ both in columns and rows. 
It may be the case in the future, that higher order derivatives will be needed, to compensate for the limitation brought by the $\mathcal O (h^n)$ polynomial accuracy of the method in the step size.
\subsection{Constraints}
The HF ground state is a great starting point to get theoretical results regarding many nuclear properties; however, for a plethora of applications, calculating the state of the nuclear system under certain constraints becomes necessary.
\\Effectively, a need to explore the energy surface arises. This means minimizing the energy functional, under a series of constraints of the type $\expval{ {\mathcal Q}}=q_0$, where $q_0$ is the desired expectation value of the operator ${\mathcal Q}$.
\\This is an equality-constrained optimization problem (ECP), formulated as
\begin{align}
\min_{\ket \Psi}&E 
\\\text{constrainted to } &\bra{\Psi} {\mathcal Q} \ket {\Psi} = \expval{ {\mathcal Q}} = q_0
\end{align}
Which yields the Lagrangian
\begin{equation}
    \label{eq:constr_lagrangian}
    E' = E + \lambda (\expval{{\mathcal Q}}-q_0)
\end{equation}
where $\lambda$ is a Lagrange multiplier determined by the condition $\expval {{\mathcal Q}} = q_0$. 
After finding the minimum of $E'$, it's trivial to show that for a given $\lambda$, we get \cite{FLOCARD1973433}
\begin{equation}
\dv{E}{\expval{{\mathcal Q}}} = -\lambda 
\end{equation}
If $\lambda$ is properly adjusted, a certain value for $q_0$ can be obtained. From a numerical standpoint, $\lambda$ needs to be tuned at each iteration to reach the desired value. 
This method was the one used in early constrained Hartree-Fock calculations \cite{Cusson1985}.
\\Although the method is simple, it's often the case that it fails. Moreover, for the same value of $\lambda$, many, possibly infinite values of $\expval{{\mathcal Q}}$ can be obtained, for which we are only allowed to get the one with the most stable solution.
\\A different method is provided by the Quadratic Penalty Method (QPM). Briefly speaking, instead of a Lagrange multiplier, we add a quadratic contribution to the functional, such that
\begin{equation}
    \label{eq:constr_quadratic}
    E' = E + \frac c 2 (\expval{{\mathcal Q}}-q_0)^2.
\end{equation}
This is a straightforward method; intuitively one penalizes (hence the name), any solution for which $\expval{{\mathcal Q}}\neq q_0$ by increasing its energy. However, the success of such procedure is heavily influenced by the choice of $c$, often leading to instabilities for large values.
\\What happens is that for small values of $c$, the penalty may be insufficient to reach the target $q_0$, while for large values, the penalty may be so big that the self consistent calculation oscillates and fails.
\subsubsection{Augmented Lagrangian Method}
A modern, robust approach, used by HF/HFB codes \cite{RYSSENS2015175,CHEN2022108344} is given in the form of the Augmented Lagrangian Method (ALM) \cite{Staszczak2010}.
Its main idea is to combine the precision of the ECP with the accuracy of the QPM.
\\Without delving into cumbersome mathematical details, we'll see how the algorithm is practically implemented in the code.
\\Given the functional
\begin{equation}
    \label{eq:alm_lagrangian}
    E' = E + \lambda (q-q_0) + \frac c 2 (q-q_0)^2
\end{equation}
where $q = \expval{{\mathcal Q}}$, the resulting mean field potential will be given by
\begin{align}
    U' &= U + \lambda  {\mathcal Q} + c(q-q_0) {\mathcal Q} 
    \\ &= U + c(q-q_0(\lambda)) {\mathcal Q}
\end{align}
Where $q_0(\lambda)$ is updated at each iteration with the formula
\begin{align}
    \label{eq:alm_q0}
    q_0(\lambda) = q_0 - \frac{\lambda}{c}
    \\\lambda ^{(i+1)} = \lambda^{(i)}+\mu c(q-q_0)
\end{align}
Here, a slight deviation from the original ALM is present. Since the original work \cite{Staszczak2010} doesn't provide guidance regarding what is considered an \textit{iteration}, we employ the strategy \cite{CHEN2022108344} of using a damping factor $\mu\in [0, 1]$, so $\lambda$ can be updated at each HF iteration for fast convergence, without large oscillations or instabilities of any kind.
\\This method is what powers the deformation curves that are shown in section (REF), allowing to explore the energy surface with arbitrary precision in reaching the value of $q_0$ at convergence, provided that enough HF iterations are performed.
\\Note that, since $\lambda^{(0)}=0$, for $\mu=0$ ALM reduces to the standard QPM.


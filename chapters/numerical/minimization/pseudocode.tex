\subsection{Code implementation}
The whole Hartree-Fock framework presented up to this point, has been implemented using the C++ language \cite{stroustrup1986overview} and the Eigen linear algebra library \cite{eigenweb}, which implements linear algebra operatrions through low level routines such as $\texttt{LAPACK}$ and $\texttt{BLAS}$. In figure \ref{fig:pseudocode}, the schematics of the program structure is reported.
\begin{figure}[h!]
\begin{forest}
  for tree={
    draw,
    %font=\ttfamily,
    grow'=0,
    child anchor=west,
    parent anchor=south,
    anchor=west,
    calign=first,
    edge path={
      \noexpand\path [draw, \forestoption{edge}]
      (!u.south west) +(5.5pt,0) |- node[inner sep=1.15pt] {} (.child anchor)\forestoption{edge label};
    },
    before typesetting nodes={
      if n=1
        {insert before={[,phantom]}}
        {}
    },
    fit=band,
    before computing xy={l=15pt},
  }
  [Code
    [Parse user input]
    [Generate mesh]
    [Solve (deformed) Woods-Saxon
      [Generate harmonic oscillator guess]
      [Diagonalize Woods-Saxon hamiltonian]
    ]
    [Normalize wavefunctions]
    [HF iterations, name=1
      [Solve BCS self-consistent equations]
      [Generate iteration data
        [Generate new densities mix with previous ones]
        [Generate mean fields]
        [Generate constraint fields mix with previous ones]
      ]
      [Compute single-particle hamiltonian]
      [Diagonalize $h$]
      [Normalize wavefunctions]
      [Check convergence
        [If above tolerance, name=3]
      ]
    ]
    [Generate output
    [{Compute multipole moments, energies, quantum numbers}] 
    [Write program output]
    ]
  ]
  % square arrow from (1) -> (2)
  \draw[-latex, to path={(\tikztostart.east) -- ++(15em,0) |- (\tikztotarget.east)}] (3) to (1);
\end{forest}
\caption{Pseudocode of the Hartree-Fock program.}
\label{fig:pseudocode}
\end{figure}
\section{Nuclear structure models}
The description of nuclear structure has been proven to be a difficult task over the years. Due to the extremely rich phenomenology of nuclei and the challenges brought by the strong force, as we shall see, many models and further approximations to give a satisfactory description of all nuclides have been proposed.
\subsection{Liquid drop model}
One, if not the first successful model, is the liquid drop model. It is based on the assumption that the nucleus behaves as a liquid droplet, where forces among consituents tend to saturate. This hypothesis, formulated by G. Gamow culminated in the formalization of the semi-empirical mass formula (SEMF) by N. Bohr and C. F. von Weizsäcker in 1935 \cite{Weizsacker1935}, which reads
\begin{equation}
    \label{eq:semf}
    E_B=a_V A - a_S A^{2/3} - a_C \frac{Z(Z-1)}{A^{1/3}} - a_A \frac{(N-Z)^2}{A} + \delta_P
\end{equation}
where $E_B$ is the binding energy of the nucleus. Each term has a different physical and phenomenological meaning:
\begin{itemize}
    \item $a_V A$ is the volume energy of the nucleus, given by the approximately constant binding energy per nucleon, which makes the total energy roughly proportional to $A$;
    \item $a_S A^{2/3}$ is the surface energy, a correction to the volume energy due to outer nuclei -- on the surface -- interacting with fewer nucleons than those in the inner bulk;
    \item $a_C Z(Z-1)/A^{1/3}$ is the approximation to the Coulomb energy repulsion of the nucleus, assuming the protons are distributed uniformly;
    \item $a_A (N-Z)^2/A$ is the asymmetry energy, which is due to the Pauli exclusion principle, since protons and neutrons occupy their respective states, a high imbalance of one species or the other implies loosely bound nucleons, thus a higher energy contribution of those states; and
    \item $\delta_P$ refers to the pairing energy of the nucleus, whose physical significance will be later discussed in section \ref{sec:intro_pairing}.
\end{itemize}
\begin{figure}[h]
    \centering
    \includegraphics[width=1.0\textwidth]{Images/Liquid_drop_model.pdf}
    \caption{Visual representation of the liquid drop model from \cite{ldmimg}}
    \label{fig:liquid_drop_model}
\end{figure}
The SEMF can be fitted on current data to get a good estimate of binding energies \cite{Benzaid2020}, but it still lacks the ability of describing many aspects of nuclear structure, mainly, the nuclear shell structure, which can account for magic numbers, nuclear deformations, and so on.
\subsection{Shell structure}
The shell model, accounts for the effect of a mean field, central potential, to which nucleons are subjected. Unlike the `atomic' case, we don't have an exact source of this field, since it's generated by the nucleons themselves. Nonetheless, the formulation of a potential which reproduces experimental data has been proven to be successful.
\\The so called Woods-Saxon potential is the empirical field used in this kind of model. It can take different parametrizations depending on the data that one wants to reproduce. It reads
\begin{equation}
    \label{eq:sphWS}
    U(\bm r) = -\frac{U_0(A, N)}{1+e^\frac{r - R}{a}}
\end{equation}
where $U_0$ is the potential depth
\begin{equation}
    U_0(A, N) = U_0\bigg(1\pm \kappa \frac{2N -A }A\bigg)
\end{equation}
where the $+$ and $-$ signs refer to protons and neutrons respectively. $R$ refers to the radius of the nuclear surface, generally parametrized as 
\begin{equation}
    R=r_0 A^{1/3}
\end{equation}
and $a$ is the surface diffuseness, as in the density expression in \eqref{eq:emp_dens}, which motivates the reason for such a construction in equation \eqref{eq:sphWS}, it has the shape of the nuclear density.
\paragraph{Spin-orbit coupling} 
The success of the shell model is due to the possibility of accounting for spin-orbit interactions, included through a term which reads
\begin{equation}
    U_{\text{LS}}(\bm r )=U_0^{\text{LS}}\bigg(\frac{r_0}{\hbar}\bigg)^2 \frac 1 r \dv{}{r}\bigg (\frac{1}{1+e^{\frac{r-R}{a}}}\bigg).
\end{equation}
\paragraph{Coulomb interaction}
In the spherical case, the coulomb interaction can be taken as the energy potential produced by a sphere of charge $Z$ and radius $R$, which reads
\begin{equation}
    U_{\text{C}}(r) = Ze^2
    \begin{cases}
        \frac{3-(r/R)^2}{2R} & r \le R \\
        \frac 1 r & r > R
    \end{cases}
\end{equation}
The complete Hamiltonian then reads
\begin{equation}
    \hat H = \hat T + U + U_{\text{LS}}+U_C
\end{equation}
where $U_C$ is present only when solving for the proton shells. The solution to the eigenvalue problem $\hat H \psi = E\psi$ is of the form
\begin{equation}
   \psi_{nljm_j} = \frac{u_{nl}(r)}{r}[Y_{l}(\hat {\bm r})\otimes \chi_{1/2}]_{jm_j} 
\end{equation}
where $Y_{nl}(\hat {\bm r})$ is the spherical harmonic function of degree $l$ and order $m$, the $\hat r$ is used to denote dependence on the azimuthal and polar angles of $\bm r$ and $\otimes$ takes the meaning of the angular momentum coupling and $u_{nl}(r)$ satisfies the reduced Schr\"odinger equation
\begin{equation}
    \label{eq:red}
    \bigg(-\frac {\hbar^2}{2m}\dv[2]{r}+\frac{\hbar l(l+1)}{2mr^2}+U(r)\bigg)\psi_{nl} = E\psi_{nl}
\end{equation}
\subsubsection{Harmonic oscillator}
A small digression on the harmonic oscillator is in order. The solution of the spherical potential 
\begin{equation}
U_{\text{HO}} (\bm r ) = \frac 1 2 m\omega^2 r^ 2
\end{equation}
produces the spherical harmonic oscillator basis, which for the lowest bound state of a Woods-Saxon potential, where $\omega$ is taken as $41/A^{1/3}$ MeV, shares very similar states. As a matter of fact, the harmonic oscillator is very often used to perform basis calculations in nuclear physics. We will see in section \ref{sec:minimization} that a harmonic oscillator basis is used as starting guess for the numerical solution of a Woods-Saxon potential.
\subsubsection{Shell structure}
A graphical representation of the shell model solution is shown in figure \ref{fig:shell_model}. 
\begin{figure}[h]
    \centering
    \includegraphics[width=0.5\textwidth]{Images/ShellModel.png}
    \caption{Graphical representation of the shell model solution}
    \label{fig:shell_model}
\end{figure}



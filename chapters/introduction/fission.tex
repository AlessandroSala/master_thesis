\section{Nuclear fission}
\label{sec:fission}
Nuclear fission is the process by which a nucleus splits into two -- sometimes three -- nuclei, whether spontaneously, or induced by a reaction.
The physics that governs nuclear fission is the one of many-body, large amplitude collective modes, which elongate the the nuclear shape, until the so called \textit{fission barrier} is surmounted and the path to the energy minimum is one where the nuclei fragments itself.
\subsubsection{Spontaneous fission model}
It should be ovious that a formal treatment of deformations and collective modes is necessary to give a theoretical description of fission reactions. We can derive a simple spontaneous fission model by studying the effect of a simple axial quadrupole deformation on the semiempirical mass formula \ref{eq:semf}.
\\Let us assume that the nuclear radius may be expanded, as previously done in section \ref{sec:deformations}, as
\begin{equation}
    R = R_0[1+\alpha_{20}Y_{20}].
\end{equation}
Assuming the nuclear volume is conserved across the fission path, the volume energy will not change. As for the surface energy, its variation can be expressed at the lowest order in $\alpha_{20}$ as
\begin{equation}
    \Delta E_\text{surf} = E_\text{surf}
    -E_{0,\text{surf}} = E_{0, \text{surf}}\frac 2 5 \alpha_{20}^2.
\end{equation}
Regarding the Coulomb energy, the variation is given by
\begin{equation}
    \Delta E_\text{coul} = E_\text{coul} - E_{0, \text{coul}} = -E_{0, \text{coul}}\frac 1 5 \alpha_{20}.
\end{equation}
Since the neutron and proton count does not change, the surface and Coulomb energies are the only contributions to the total energy difference. We can write
\begin{equation}
    \label{eq:fission_semf}
    \Delta E = \frac 2 5 \alpha_{20}^2 a_s A^{2/3}- \frac 1 5 \alpha_{20}^2 a_c Z^2 A^{-1/3},
\end{equation}
if we set equation \eqref{eq:fission_semf} to zero, we get, other than the undeformed solution for $\alpha_{20}=0$, 
\begin{equation}
    \frac{ Z^2}{A} = \frac{2 a_s}{a_c},
\end{equation}
where the ratio $2a_s/a_c$ amounts to $\approx 50$ in typical parametrizations of the SEMF. Equation \eqref{eq:fission_semf}, shows that for values of the so called \textit{fissility parameter} $Z^2/A$ larger than $50$, the energy change becomes negative, favouring a configuration in which the nucleus fragments due to the spontaneous fission.
\subsection{Octupole deformations}

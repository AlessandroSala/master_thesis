\section{Nuclear pairing}
\label{sec:pairing_intro}
In the semi-empirical mass formula \eqref{eq:semf}, the $\delta_p$ term can be parametrised as 
\begin{equation}
    \delta_p = \begin{cases}
        +\delta_0 & \text{ if N and Z are even}, \\
        0 & \text{ if A is odd}, \\
        -\delta_0 & \text{ if N and Z are odd},
    \end{cases}
\end{equation}
hence having an even number of neutrons and/or protons increases the binding energy of the nucleus. A common choice for $\delta_0$ is
\begin{equation*}
\delta_0 = a_P A^{-1/2}\text{ MeV}.
\end{equation*}
A typical value for $a_P$ is reported in table \ref{tab:semb_coeff}.
This is a phenomena closely related to superconductivity, as nucleons of the same type form pairs that are strongly correlated and increase the nuclear binding. An experimental evidence of this fact is known as odd-even staggering, where the separation energy
\begin{equation}
    S_n = E_B(A+1, Z) - E_B(A, Z),
\end{equation}
is higher for even $A$, an increase that corresponds to the energy necessary to break a pair. A graphical representation of the odd-even staggering for Sn isotopes is shown in figure \ref{fig:staggering}. We will see in section \ref{sec:pairing_hf} the two main methods to account for pairing at a microsopic level.

\begin{figure}[h]
    \centering
    \includegraphics[width=0.9\textwidth]{Images/staggering}
    \caption[Odd-even staggering for Sn isotopes.]{Odd-even staggering for Sn isotopes. Data taken from \cite{AMDC_website}.}
    \label{fig:staggering}
\end{figure}


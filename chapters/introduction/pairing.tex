\section{Nuclear pairing}
\label{sec:pairing_intro}
In the semi-empirical mass formula \eqref{eq:semf}, the $\delta_p$ term is parametrised as 
\begin{equation}
    \delta_p = \begin{cases}
        +\delta_0 & \text{ if N and Z are even}, \\
        0 & \text{ if A is odd}, \\
        -\delta_0 & \text{ if N and Z are odd},
    \end{cases}
\end{equation}
hence having an even number of neutrons and/or protons increases the binding energy of the nucleus. A common choice for $\delta_0$ is
\begin{equation*}
\delta_0 = 12 A^{1/2}\text{ MeV}.
\end{equation*}
This is a phenomena closely related to superconductivity, as nucleons of the same type form pairs that lie in higher energy states. An experimental evidence of this fact is knwon as odd-even staggering, where the separation energy
\begin{equation}
    S_n = E_B(A, Z) - E_B(A-1, Z),
\end{equation}
is higher for even $A$, an increase that corresponds to the energy necessary to break a pair. We will see in section \ref{sec:pairing_hf} the two main methods to account for pairing at a microsopic level.



\section{Microscopic theory and approaches to nuclear fission}
\label{sec:microscopic_fission}
Having described the framework of nuclear DFT and the treatment of pairing correlations within the HFB formalism, we wish to spend a few pages in regards to the microscopic theory of fission. As we shall see, the application of these frameworks to the description of nuclear fission is one of the main drives for the development of new, spatially unconstrained solvers.
\subsection{Microscopic theory}
The use of phenomenological macroscopic-microscopic models has long provided valuable insight into fission processes, allowing for the prediction of barrier heights and fragment yields through parametrised shape degrees of freedom and empirical shell corrections \cite{Brack1972,Bjornholm1980,Krappe2012}.  
In these models, the total energy is expressed as
\begin{equation}
E_{\mathrm{tot}}(\boldsymbol{q}) = E_{\mathrm{LD}}(\boldsymbol{q}) + \delta E_{\mathrm{shell}}(\boldsymbol{q}) + \delta E_{\mathrm{pair}}(\boldsymbol{q}),
\end{equation}
where $E_{\mathrm{LD}}$ is the macroscopic liquid-drop term depending on deformation coordinates $\boldsymbol{q}$, while $\delta E_{\mathrm{shell}}$ and $\delta E_{\mathrm{pair}}$ account for shell and pairing corrections, respectively.  
While such models reproduce many global observables, they lack a true microscopic foundation.  
In particular, the coordinates $\boldsymbol{q}$ are not derived from the underlying many-body dynamics, and the empirical shell corrections cannot describe the self-consistent rearrangement of the mean-field along the fission path.

A more fundamental understanding is achieved within self-consistent mean-field approaches such as the HF or HFB formalisms.
The resulting constrained HF/HFB calculations produce the potential energy surface (PES) $E(\boldsymbol{q})$, mapping the energy of the system as a function of collective deformations such as the quadrupole ($Q_{20}$), octupole ($Q_{30}$), and triaxial ($Q_{22}$) moments.  
The minima and saddle points of this multidimensional PES determine the fission barriers and shape isomeric states \cite{Dubray2012,Schunck2016}.

However, static mean-field approaches are limited by their single-reference character: the HFB vacuum represents only one configuration at a time, typically corresponding to a local minimum of the PES.  
In the vicinity of the fission barrier, where several configurations with different intrinsic quantum numbers coexist, this approximation breaks down.  
The wave function should instead be expressed as a superposition of several self-consistent configurations $\{|\Psi(\boldsymbol{q})\rangle\}$, leading to a correlated state of the form
\begin{equation}
|\Psi\rangle = \int f(\boldsymbol{q})\, |\Psi(\boldsymbol{q})\rangle\, d\boldsymbol{q},
\end{equation}
which is the essence of the \emph{Generator Coordinate Method} (GCM) \cite{Goutte2005,Regnier2019}.  
The GCM maps the microscopic many-body problem onto a \emph{collective Schrödinger equation} (CSE)
\begin{equation}
\left[ -\frac{\hbar^2}{2}\sum_{ij}\frac{\partial}{\partial q_i} B_{ij}(\boldsymbol{q}) \frac{\partial}{\partial q_j} + V(\boldsymbol{q}) \right] g_k(\boldsymbol{q})
= E_k g_k(\boldsymbol{q}),
\end{equation}
where $B_{ij}(\boldsymbol{q})$ is the collective inertia tensor and $V(\boldsymbol{q})$ the potential energy extracted from constrained HFB.  
This framework naturally incorporates tunnelling through the barrier and provides access to observables such as fission lifetimes and fragment distributions.  

Beyond-mean-field extensions also restore symmetries that are spontaneously broken at the mean-field level.  
For instance, particle-number, parity, and angular-momentum projection techniques \cite{Bender2004,Samyn2005} are required to recover good quantum numbers and remove spurious symmetry mixing.  
In multi-reference DFT \cite{Bender2003}, these symmetry restorations can be combined with configuration mixing, yielding highly accurate fission barrier calculations.  

\subsection{Time-dependent DFT and the TDHF formalism}

Static HF/HFB solutions provide a sequence of constrained energy minima along deformation coordinates, but they cannot describe the real-time evolution of the system. A fully dynamical extension is obtained within time-dependent Density Functional Theory (TDDFT), whose nuclear implementation corresponds to the time-dependent HF (TDHF) or time-dependent HFB(TDHFB) equations \cite{Negele1982}.

TDHF follows from the time-dependent variational principle applied to a Slater determinant
\[
\delta \int dt\, \langle \Psi(t) | i\hbar \partial_t - \hat{H} | \Psi(t) \rangle = 0,
\]
The resulting Euler-Lagrange equations lead to a set of non-linear Schr\"odinger equations for the occupied single-particle orbitals
\begin{equation}
i\hbar \partial_t \varphi_k(\mathbf{r},t)
= h[\rho(t)]\, \varphi_k(\mathbf{r},t),
\label{eq:TDHF}
\end{equation}
with $h[\rho(t)]$ the self-consistent mean-field Hamiltonian obtained from the Energy Density Functional. The time evolution of the one-body density matrix,
\[
\rho_{ij}(t) = \langle \Psi(t) | a_j^\dagger a_i | \Psi(t) \rangle,
\]
is therefore governed by the Liouville-von Neumann equation
\begin{equation}
i\hbar \frac{d\rho}{dt} = [h[\rho],\, \rho],
\label{eq:Liouville}
\end{equation}
which ensures hermiticity, idempotency, and particle-number conservation. For Skyrme EDFs, the Hamiltonian includes local densities and currents, so the TDHF evolution also incorporates time-odd fields essential for dynamical processes \cite{Bulgac2008}.

In the presence of pairing, the formalism generalises to TDHFB through the Bogoliubov density matrices
\[
\mathcal{R}(t)=
\begin{pmatrix}
\rho(t) & \kappa(t) \\
-\kappa^\ast(t) & 1-\rho^\ast(t)
\end{pmatrix},
\qquad
\mathcal{H}(t)=
\begin{pmatrix}
h(t) & \Delta(t) \\
-\Delta^\ast(t) & -h^\ast(t)
\end{pmatrix},
\]
which satisfy the generalized equation of motion
\begin{equation}
i\hbar \frac{d\mathcal{R}}{dt} = [\mathcal{H},\, \mathcal{R}].
\label{eq:TDHFB}
\end{equation}

Time-dependent methods offer several key advantages for the description of fission dynamics. First, they treat collective motion and intrinsic excitations on the same footing: changes in configuration occur through level crossings, rearrangements of occupation probabilities, and dynamical time-odd fields. This enables the exploration of diabatic pathways unavailable to static constrained minimisation. Second, the scission process is described naturally as the neck density decreases and two self-bound fragments emerge, carrying well-defined densities, currents, and angular momenta. Fragment intrinsic excitation energy may be extracted from post-scission relaxation \cite{Bulgac2016}.

Despite these advantages, TDHF has fundamental limitations. Being restricted to a single Slater determinant, it cannot describe quantum tunnelling (necessary for spontaneous fission) nor generate fragment mass and charge distributions. Its dynamics is deterministic, yielding only the mean trajectory in collective space. Extensions such as the time-dependent random-phase approximation (TDRPA), the stochastic mean-field approach \cite{Ayik2008}, and the time-dependent GCM with Gaussian overlap approximation \cite{Goutte2005} introduce fluctuations and correlations around the TDHF path, enabling more realistic descriptions of fission fragment observables.

\subsection{Unconstrained Calculations and Symmetry Breaking}

An equally important aspect of microscopic fission theory is the treatment of spatial symmetries.  
Historically, many calculations imposed constraints such as axial symmetry or reflection symmetry with respect to a plane to reduce the computational cost of solving the HFB equations.  
While such restrictions simplify the description of the nucleus, they artificially constrain the fission path and may even prevent the identification of energetically preferred configurations \cite{Warda2002,Bertsch2018}, as shown in Figure~\ref{fig:fission_barrier}.  
Fission involves strongly deformed, triaxial, and reflection-asymmetric shapes; the correct description of barrier heights and scission configurations therefore requires breaking as many spatial symmetries as possible.

In a self-consistent mean-field framework, spontaneous symmetry breaking is a feature rather than a flaw: it allows the system to adopt a deformed intrinsic shape corresponding to a broken rotational or parity symmetry, while the symmetry of the total many-body Hamiltonian is preserved.  
For example, an axially deformed HFB state violates rotational invariance, but the restoration of this symmetry through angular-momentum projection recovers the correct laboratory-frame properties.  
Triaxiality, for example, has been shown to lower the inner barrier of actinides by several MeV \cite{Warda2002,Schunck2016}.
Similarly, parity breaking through octupole deformation is essential to describe asymmetric fission fragment distributions. 
Likewise, reflection-asymmetric (octupole) degrees of freedom are necessary to reproduce mass-asymmetric fission in heavy nuclei.

Recent computational developments have made possible fully symmetry-unrestricted HFB and TDDFT calculations, in which all spatial and time-reversal symmetries can be broken if energetically favourable \cite{Simenel2018,Schunck2016}.  
Codes such as \texttt{HFODD} and \texttt{Sky3D} implement three-dimensional solvers capable of describing triaxial, octupole, and time-odd components of the density matrix.  
These advances have revealed new fission pathways, scission configurations, and fragment-spin correlations inaccessible to axially symmetric models.

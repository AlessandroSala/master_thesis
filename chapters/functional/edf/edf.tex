\subsection{Energy density functional}
The energy functional to be minimised is of the form \cite{Bender2003}
\begin{equation}
\label{eq:full_functional}
E_{\text{HF}} =  E_\text{Kin}+E_\text{Skyrme}+E_\text{Coul} = \int( \mathcal E_\text{Kin} + \mathcal E_\text{Skyrme} + \mathcal E_\text{Coul})d\bm r.
\end{equation}
\subsubsection{Densities}
Functional \eqref{eq:full_functional} can be expressed through a series of generlized particle densities. Let us define them and express them on the spin coordinates up ($\uparrow$) and down ($\downarrow$) for the convenience in a mesh representation.
\\The starting point is the density matrix, defined as
\begin{equation}
    \rho_q (\mathbf r \sigma, \mathbf r \sigma') = \sum_{\alpha} \phi_{\alpha, \sigma} (\mathbf r )\phi_{\alpha, \sigma'}^*(\mathbf r')
\end{equation}
where the index $\alpha$ goes through all single particle states of the particles of type $q$ (protons, neutrons) and the index $\sigma$ refers to the spin coordinate. The particle density is defined as 
\begin{align}
    \rho_q(\mathbf r) \coloneq\rho_q(\mathbf r, \mathbf r')\bigg|_{\mathbf{r} = \mathbf{r'}} \coloneq \sum_{\sigma}\rho(\mathbf r\sigma, \mathbf r'\sigma) \bigg|_{\mathbf{r} = \mathbf{r'}} &=\sum_{\alpha} \phi_{\uparrow}(\mathbf r)\phi_{\uparrow}^*(\mathbf r')+\phi_{\downarrow}(\mathbf r)\phi_{\downarrow}^*(\mathbf r') \bigg|_{\mathbf{r} = \mathbf{r'}} \nonumber
    \\&=\sum_{\alpha} |\phi_{\uparrow}(\mathbf r)|^2+|\phi_{\downarrow}(\mathbf r)|^2.
    \label{eq:part_density}
\end{align}
The kinetic density reads
\begin{align}
    \tau_q(\mathbf r) &\coloneq \sum_{\alpha} \nabla'\cdot\nabla\rho_q(\mathbf r, \mathbf r')\bigg|_{\mathbf r'=\mathbf r} \nonumber
    \\&= \sum_{\sigma, \alpha} \nabla \phi_\sigma (\mathbf r)\cdot \nabla \phi_\sigma^*(\mathbf r')\bigg|_{\mathbf r = \mathbf r'} = \sum_{\sigma, \alpha} |\nabla \phi_\sigma(\mathbf r)|^2 \nonumber
    \\&= \sum_{\alpha}|\nabla \phi_\uparrow(\mathbf r)|^2 + |\nabla \phi_\downarrow(\mathbf r)|^2 \label{eq:kin_density}.
\end{align}
The spin density  reads
\begin{align}
    s_q(\mathbf r, \mathbf r') &\coloneq\sum_{\sigma \sigma', i} \rho_q(\mathbf r \sigma, \mathbf r' \sigma')\bra{\sigma'} \hat {\boldsymbol{\sigma}} \ket{\sigma} = \sum_{\alpha} \begin{bmatrix} \phi_{\uparrow}^*(\mathbf r') \ \phi_{\downarrow}^*(\mathbf r') \end{bmatrix}\hat{\boldsymbol{\sigma}} \begin{bmatrix} \phi_{\uparrow}(\mathbf r) \\ \phi_{\downarrow}(\mathbf r) \end{bmatrix}
\end{align}
and lastly, the spin-orbit density tensor reads
\begin{align}
    J_{q, \mu\nu} &\coloneq \frac 1 {2i}(\partial_\mu - \partial_\mu') s_{q, \nu}(\mathbf r, \mathbf r')\bigg|_{\mathbf r'=\mathbf r}\nonumber \\
    &= \frac 1 {2i}\bigg(\begin{bmatrix}\phi_{\uparrow}^*(\boldsymbol r')\ \phi_{\downarrow}^*(\boldsymbol r')\end{bmatrix} \partial_\mu\hat{\sigma}_\nu\begin{bmatrix} \phi_{\uparrow}(\mathbf r) \\ \phi_{\downarrow}(\mathbf r) \end{bmatrix} - \begin{bmatrix}\phi_{\uparrow}(\boldsymbol{r})\ \phi_{\downarrow}(\boldsymbol{r})\end{bmatrix} \partial_\mu'\hat{\sigma}_\nu\begin{bmatrix} \phi_{\uparrow}^*(\mathbf r') \\ \phi_{\downarrow}^*(\mathbf r') \end{bmatrix}\bigg)_{\mathbf r'=\mathbf r}\nonumber
     \\&= \sum_\alpha\Im\bigg\{\begin{bmatrix}\phi_{\uparrow}^*(\boldsymbol r)\ \phi_{\downarrow}^*(\boldsymbol r) \end{bmatrix}\partial_\mu \hat{\sigma}_\nu\begin{bmatrix} \phi_{\uparrow}(\mathbf r) \\ \phi_{\downarrow}(\mathbf r) \end{bmatrix}\bigg\}
\end{align}
which also defines the spin-orbit current vector $\bm J$ that reads
\begin{equation}
     J_{q,\kappa} (\bm r ) = \sum_{\mu\nu}\epsilon_{\kappa\mu\nu} J_{q, \mu\nu}(\bm r).
\end{equation}
\subsubsection{Kinetic functional}
The kinetic term can be expressed as
\begin{equation}
    \label{eq:kinfunc}
    \mathcal E_\text{Kin} = \frac{\hbar^2}{2m}\tau
\end{equation}
which is found integrating by parts \eqref{eq:kin_functional}.
\subsubsection{Skyrme functional}
Since this work only deals with even-even nuclei, only time-even densities, which are the ones previously defined, are non-vanishing, due to the ground state being time-reversal invariant \cite{Bender2003}. This reduces the Skyrme functional to the following form \cite{stevenson2019low}
\begin{align}
    \label{eq:skfunc}
    \mathcal E_\text{Skyrme} &= \sum_{t=0,1}\bigg\{C_t^\rho [\rho_0]\rho_t^2+C_t^{\Delta \rho}\rho_t\nabla^2\rho_t+C_t^{\nabla\cdot J}\rho_t\nabla\cdot \mathbf J_t + C_t^\tau\rho_t\tau_t\bigg\}
\end{align}
where
\begin{align}
    %\label{eq:coefficients_func}
    C_0^\rho &= +\frac 3 8 t_0 + \frac 3 {48} t_3\rho_0^\sigma \label{eq:C0rho}
    \\C_1^\rho &= -\frac 1 8 t_0(1+2x_0)- \frac 1 {48} t_3(1+x_3)\rho_0^\sigma \label{eq:C1rho}
    \\C_0^\tau &= +\frac 3 {16} t_1 + \frac 1 {16} t_2 (5+4x_2) \label{eq:C0tau}
    \\C_1^\tau &= -\frac 1 {16} t_1(1+2x_1)+\frac 1 {16}t_2(1+2x_2) \label{eq:C1tau}
    \\C_0^{\Delta \rho} &= -\frac 9 {64}t_1+\frac 1 {64}t_2(5+4x_2) \label{eq:C0Deltarho}
    \\C_1^{\Delta \rho} &= +\frac 3 {64}t_1(1+2x_1)+\frac 1 {64}t_2(1+2x_2) \label{eq:C1Deltarho}
    \\C_0^{\nabla\cdot J} &= -\frac 3 4 W_0 \label{eq:C0nabladotJ}
    \\C_1^{\nabla\cdot J} &= -\frac 1 4 W_0 \label{eq:C1nabladotJ}.
\end{align}
Here, $t=0,1$ refers to the isoscalar and isovector components of the densities, that is
\begin{align*}
    \rho_0 = \rho_p + \rho_n
    \\\rho_1 = \rho_p - \rho_n
\end{align*}
and the same holds for all generalised densities.
We can now derive the Kohn-Sham equations, by minimising the functional under the constraint
\begin{equation}
    \label{eq:spe_ks_constraint}
    \bra{\varphi_i}\ket{\varphi_j}=\delta_{ij}.
\end{equation}
The resulting Kohn-Sham equations are of the form
\begin{equation}
    \label{eq:spe_ks}
    \bigg[-\nabla\bigg(\frac{\hbar^2}{2m^{*}_q(\mathbf r)}\nabla \bigg) + U_q(\mathbf r) + \delta_{\text{q,proton}}U_C(\mathbf r)-i\mathbf B_q(\mathbf r)\cdot(\nabla \times \boldsymbol\sigma) \bigg]\varphi_\alpha=\varepsilon_\alpha\varphi_\alpha
\end{equation}
where an effective mass field arises, which is defined as
\begin{equation}
    \frac{\hbar^2}{2m^{*}_q(\mathbf r)} = \fdv{\mathcal E}{\tau_q}
\end{equation}
a mean field potential, which reads
\begin{equation}
    U_q(\mathbf r) = \fdv{\mathcal E}{\rho_q}\label{eq:fdv_rho_skyrme}
\end{equation}
and a spin-orbit field, given by
\begin{equation}
    \label{eq:sofield}
    \mathbf B_q(\mathbf r) = \fdv{\mathcal E}{\boldsymbol{\mathbf J_q}}.
\end{equation}
The coulomb field $U_C$, which is present only in the single particle equation for protons, doesn't come from the Skyrme interaction, rather from the Coulomb part of the whole functional. It will be properly derived in section \ref{sec:coulomb_treatment}.
\\Following the rules for functional derivatives, outlined in the appendix \ref{app:func_der} we get
\begin{align}
    \frac{\hbar^2}{2m_q^*(\mathbf r)} =& +\frac{\hbar^2}{2m} \nonumber
    \\&+ \frac 1 8 [t_1(2+x_1)+t_2(2+x_2)]\rho(\mathbf r) \nonumber
    \\&- \frac 1 8 [t_1(1+2x_1)+t_2(1+2x_2)]\rho_q(\mathbf r ) \\\nonumber
\end{align}
\begin{align}
    U_q(\mathbf r) =& +\frac 1 8 [t_1(2+x_1)+t_2(2+x_2)]\rho \nonumber
    \\&+ \frac 1 8 [t_2(1+2x_2)-t_1(1+2x_1)]\rho_q \nonumber
    \\&+ \frac 1 8 [t_1(2+x_1)+t_2(2+x_2)]\tau \nonumber
    \\&+ \frac 1 8 [t_2(1+2x_2)-t_1(1+2x_1)]\tau_q \nonumber
    \\
    &+ \frac 1 {16} [t_2(2+x_2)-3t_1(2+x_1)] \nabla^2 \rho \nonumber
    \\&+ \frac 1 {16} [3t_1(2x_1+1)+t_2(2x_2+1)] \nabla^2 \rho_q \\\nonumber
\end{align}
\begin{align}
    \mathbf B_q (\mathbf r ) = &+\frac 1 2 W_0 [\nabla\rho + \nabla \rho_q] \nonumber\\
    &-\frac 1 8 (t_1 x_1 + t_2 x_2) \mathbf J + \frac 1 8 (t_1 - t_2) \mathbf J_q.
\end{align}
Unless otherwise specified, unlabelled densities denote isoscalar quantities (sum of neutron and proton).
\subsection{Functionals}
The set of parameters $(t_0, t_1, t_2, t_3, x_0, x_1, x_2, \sigma, W_0)$ in the Skyrme functional \eqref{eq:skfunc} is not universal and varies between different parametrisations. These parameters are fitted to experimental data so that the resulting EDF can reproduce known nuclear properties with good accuracy. 

The last widely used family of Skyrme forces derived directly from an effective interaction is the SLy series, which was primarily fitted to properties of magic nuclei and to constraints from neutron-rich matter in neutron stars \cite{chabanat2}. Subsequently, the development of new parametrisations has been driven by the need to reproduce more accurately the masses and radii of nuclei across the nuclear chart, including open-shell and deformed systems. Notable examples include the classic SIII force \cite{Beiner1974}, the deformation-oriented SkM* parametrisation \cite{Bartel1982} to accurately reproduce the $^{240}$Pu fission barrier, and modern large-scale optimisations such as UNEDF0 \cite{KortUNEDF0}.
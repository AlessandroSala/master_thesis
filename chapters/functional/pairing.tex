\section{Pairing in Hartree-Fock theory}
\label{sec:pairing_hf}
In this section, we will discuss the two common approaches to include nuclear pairing in the HF theory. The aim of these few pages is to provide a brief overview of how the BCS equations are derived and understand the basics of the more general Hartree-Fock-Bogoliubov theory. The former method is the most widely implemented thanks to its low complexity \cite{hfbcsqrpa,oldEv8,skyax}, while the latter, more sophisticated and advanced, is the standard in modern codes \cite{hfodd, hfbftt,MAREVIC2022108367}, we will touch on it so that the reader may appriciate in the numerical chapter the natural extension of this work to the more general Bogoliubov ansatz. 
\subsection{BCS theory}
The BCS approximation, from Bardeen-Cooper-Schrieffer, is the same theory used to describe Cooper pairs in superconductivity, applied to the nuclear case.
The ansatz of BCS is that nucleons are paired in states whose total angular momentum is zero, such a wavefunction can be expressed as $\ket{JM}=\ket{00}$ and reads
\begin{equation}
    \label{eq:zero_mom}
    \ket{00} = \sum_m \bra{jm_j j-m_j}\ket{00}\ket{j-m}
\end{equation}
Introducting the time-reversal operator $\hat {\mathcal T}:t\mapsto -t$, it acts on $\ket{00}$ as 
\begin{equation}
    \hat {\mathcal T} \ket{jm} = \widetilde{\ket{{jm}}} = (-1)^{j+m}\ket{j-m},
\end{equation}
using this relation, equation \eqref{eq:zero_mom} becomes 
\begin{equation}
    \ket{00} =- \frac{1}{\sqrt{2j+1}}\ket{jm}\widetilde{\ket{{jm}}}.
\end{equation}
Hence BCS amounts to using in the HF method the more general Slater determinant
\begin{equation}
    \label{eq:bcs_det}
    \bcs = \prod_{k>0}(u_k+ v_k a_{\tilde k}^\dagger a_k^\dagger) \ket{-}
\end{equation}
where $u_k$ and $v_k$ are real parameters whose meaning will shortly be clear, $k$ is short-hand for $\ket{jm}$, and $\tilde k = -k$ denotes the time-reversal state of $k$; the product runs over positive $k$ only. The BCS wavefunction is the creation in the vacuum of quasi-paticles made of time-reversal paired particles, instead of individual ones. The normalization condition on the BCS wavefunction reads
\begin{equation}
    \label{eq:bcs_norm}
    1=\braket{\text{BCS}} = \prod_{k>0} \bra{-}(u_{k}+v_{k}a_{k}a_{\tilde k})(u_k+v_k a_{\tilde k}^\dagger a_k^\dagger)\ket{-} =\prod_{k>0}(u_k^2+v_k^2)=1
\end{equation}
which implies, for every pair $k$, the condition
\begin{equation}
    \label{eq:norm_uv}
    u_k^2+v_k^2=1.
\end{equation}
Taking the expectation value of the particle number operator $\hat N = \sum_k a_k ^\dagger a_k$ yields \cite{bertulani2007nuclear}
\begin{equation}
    \bbcs \hat N \bcs = 2\sum_ {k>0} v_k^2,
\end{equation}
while the expectation value of the particle number dispersion reads
\begin{equation}
\expval{\Delta\hat N ^2} = \expval{\hat N^2} - \expval{\hat N}^2 = 4\sum_{k>0} v_k u_k.
\end{equation}
The consequence of this result is profound. The BCS ansatz does not assume a fixed number of particles, rather it becomes an observables of the system, with an expectation value that depends on how the parameters $v_k^2$ are set, which represent the probability of finding a particle in the $k$-th state.
We can now write the many body Hamiltonian of the system as in equation \eqref{eq:mb_hamiltonian_sq}
\begin{equation}
    \hat H = \sum_{k_1 k_2}t_{k_1k_2} a_{k_1}^\dagger a_{k_2} + \frac 1 4 \sum_{k_1 k_2 k_3 k_4}\overline{v}_{k_1k_2k_3k_4} a_{k_1}^\dagger a_{k_2}^\dagger a_{k_3} a_{k_4}
\end{equation}
and replace it with the Routhian 
\begin{equation}
    \label{eq:bcs_routhian}
    \bbcs \hat H - \lambda \hat N \bcs
\end{equation}
so that the expected number of particles may be fixed, under the appropriate choice of $\lambda$, by the relation
\begin{equation}
\pdv{}{N}{\bbcs \hat H - \lambda \hat N \bcs} = \lambda.
\end{equation}
the Lagrange multiplier $\lambda$ takes on the meaning of the Fermi energy.
We can now apply the variational principle \eqref{eq:var_eq_res} to \eqref{eq:bcs_routhian} using the $v_k$ as variational quantities, which yields 
\begin{equation}
   4\tilde \varepsilon_k ^2 u_k^2 v_k^2 = \Delta_k ^2 - \Delta_k^2 u_k^2v_k^2,
\end{equation}
where the pairing gap $\Delta_k$ is defined as
\begin{equation}
    \label{eq:delta_k}
    \Delta_k = - \sum_{k'}\overline v_{k\tilde k k'\tilde{k'}}v_{k'}u_{k'} 
    \end{equation}
and the quantity $\tilde\varepsilon_k$ is defined as 
\begin{align}
    \label{eq:epsilon}
    \tilde \varepsilon_k &= \frac 1 2 \bigg[t_{kk} + t_{\tilde k \tilde k }-2\lambda +\sum_{k'}(\overline v_{k\tilde k k'\tilde{k'}}v_{k'}u_{k'} + \overline v_{\tilde k k' \tilde k k'})v_{k'}^2\bigg]
    \\&=\frac 1 2 [h_{kk}+h_{\tilde k \tilde k}]-\lambda.
\end{align}
Introducing the quasi-particle energy
\begin{equation}
    \label{eq:qpe}
    E_k = \sqrt{\tilde \varepsilon_k^2 +\Delta_k^2}
\end{equation}
we can combine definitions \eqref{eq:delta_k} and \eqref{eq:qpe} with equation \eqref{eq:epsilon}, under the normalization condition \eqref{eq:norm_uv}, to get an equation for $v_k^2$
\begin{equation}
    \label{eq:v2}
    v_k^2 = \frac 1 2 \pm \frac{|\tilde \varepsilon_k|}{2E_k}.
\end{equation}
Since in the Hartree-Fock limit, where the occupations $v_k^2$ are equal to one below the fermi energy and zero above, and the gaps $\Delta_k$ vanish, rendering $E_k=\tilde \varepsilon_k$, we only select the solution
\begin{equation}
v_k^2 = \frac 1 2 - \frac{\tilde \varepsilon_k}{2E_k}.
\end{equation}
Using the normalization condition to write $u_k^2=1-v_k^2$, and plugging it into the gaps definition \eqref{eq:delta_k}, we arrive to the gap equation
\begin{equation}
    \label{eq:gap}
    \Delta_k = - \sum_{k'}\frac{\Delta_{k'}\overline v_{k\tilde k k'\tilde{k'}}}{2E_{k'}}
\end{equation}
The system of equations (\ref{eq:gap}, \ref{eq:norm_uv}, \ref{eq:v2}, \ref{eq:qpe}, \ref{eq:epsilon}), together with the condition $\expval{\hat N}=N$ is closed and can be solved numerically, usually through an effective pairing interaction.
\subsection{Hartree-Fock-Bogoliubov theory}
The most general ansatz to account for pairing interactions in Hartree-Fock theory is the Hartree-Fock-Bogoliubov (HFB) theory, it allows a treatment of the mean-field and pairing interactions in a unified way, the quasi-particles created on the vacuum are the most general ones, instead of being time-reversal paired particles.
Let us start by writing a Bogoliubov transformation from the particle basis $c_i$ to a quasi-particle one 
\begin{equation}
    \label{eq:bogoliubov_trans}
    \beta_k^\dagger = \sum_l U_{lk} c_l^\dagger + V_{lk} c_l.
\end{equation}
If we take the Hermitian conjugate of the relation \eqref{eq:bogoliubov_trans}, we get the transformation for $\beta_k$, we are then able to write in matrix form
\begin{equation}
    \label{eq:bogoliubov_trans_mat}
    \begin{pmatrix}
    \beta\\
    \beta^\dagger
    \end{pmatrix}
    =\begin{pmatrix}
        U^\dagger & V^\dagger \\
        V^T & U^T
    \end{pmatrix}
    \begin{pmatrix}
        c\\
        c^\dagger
    \end{pmatrix}
    =\mathcal W^\dagger \begin{pmatrix}
        c\\
        c^\dagger
    \end{pmatrix},
\end{equation}
where the matrix of matrices $\mathcal W$ reads
\begin{equation}
    \label{eq:bogoliubov_mat}
    \mathcal W = \begin{pmatrix}
        U & V^*\\
        V & U^*
    \end{pmatrix}.
\end{equation}
Taking the product $\mathcal W^\dagger \mathcal W$ and imposing separate fermionic commutation relations of the operators $\beta, \beta^\dagger, c, c^\dagger$, we get that $\mathcal W$ is unitary, hence
\begin{equation}
    \mathcal W^\dagger \mathcal W = \mathcal W \mathcal W^\dagger =  I.
\end{equation}
We can now invert equation \eqref{eq:bogoliubov_trans_mat} by multiplying both sides on the left by $\mathcal W$, which yields 
\begin{equation*}
    \mathcal W \begin{pmatrix}
        \beta\\
        \beta^\dagger
    \end{pmatrix}
    =\begin{pmatrix}
        c\\
        c^\dagger
    \end{pmatrix}.
\end{equation*}
Using the Messiah-Bloch decomposition \cite{blochmessiah}, we can write the unitary matrix $\mathcal W$ as
\begin{equation}
    \label{eq:decomposition}
    \mathcal W = \begin{pmatrix}
        D & 0 \\
        0 & D^*
    \end{pmatrix}
    \begin{pmatrix}
        \overline U & \overline V \\
        \overline V & \overline U
    \end{pmatrix}
    \begin{pmatrix}
        C & 0 \\
        0 & C^*
    \end{pmatrix}
\end{equation}
where $D$ and $C$ are unitary matrices and $\overline U$ and $\overline V$ are real matrices, which have a particular blocked form, expressed through the coefficents $u_k, v_k$; the reader may refer to appendix \ref{app:UV} for the explicit representation. We can also define the matrices $U, V$ as 
\begin{equation}
    U=D\overline U C,\quad V=D^*\overline V C.
\end{equation}
Using the decomposition \eqref{eq:decomposition} we can define the \textit{canonical basis} as
\begin{equation}
    \label{eq:canonical_basis}
    a_k^\dagger = \sum_l D_{lk}^\dagger c_l^\dagger,
\end{equation}
a \textit{special Bogoliubov transformation} between \textit{paired} levels as
\begin{align}
    \alpha_k^\dagger = u_k a_k^\dagger - v_k a_{\tilde k},\\
    \alpha_{\tilde k } ^\dagger = u_k a_{\tilde k} ^\dagger + v_k a_k,
\end{align}
and \textit{blocked} levels
\begin{align}
\alpha_i &= a_i,\quad \alpha_n^\dagger = a_n^\dagger
\\\alpha_i &= a_i^\dagger, \quad \alpha_n = a_n,
\end{align}
where $u_k=u_{\tilde k},\ v_k=-v_{\tilde k}$, and a unitary transformation of the quasi-particle operators $\alpha_k^\dagger$ among themselves
\begin{equation}
    \beta_k^ \dagger = \sum_{k'}C_{k'k}a_{k'}^\dagger.
\end{equation}
We are now able to define the Bogoliubov ground state $\ket{\text{HFB}}$, as the one for which
\begin{equation}
    \beta_k \ket{\text{HFB}} = 0 \ \forall k = 1,\ldots,M
\end{equation}
where $M$ is determined by the physical situation \cite{ring2004nuclear}.
The wavefunction that satisfies this condition reads
\begin{equation}
\ket{\text{HFB}}=\prod_k^M\beta_k \ket{-}.
\end{equation}
We can define the pairing tensor as
\begin{equation}
    \kappa_{ll'} = \bra{\text{HFB}}c_{l'}c_l\ket{\text{HFB}},
\end{equation}
which in matrix form reads, alongside the density matrix
\begin{equation}
    \kappa = V^*U^T,\quad \rho = V^*V^T.
\end{equation}
We can now apply the variational principle \eqref{eq:var_eq_res}
\begin{equation}
    \label{eq:varhfb}
    \delta \frac{\bra{\text{HFB}} \hat H-\lambda \hat N\ket{\text{HFB}}}{\bra{\text{HFB}}\ket{\text{HFB}}} = 0.
\end{equation}
which yields the eigenvalue problem
\begin{equation}
    \label{eq:eighfb}
    \begin{pmatrix}
    h -\lambda& \Delta \\ -\Delta^* & -(h-\lambda)^*
    \end{pmatrix}
    \begin{pmatrix}
        U_k \\ V_k
    \end{pmatrix}
    =\mathcal H_\text{HFB}\begin{pmatrix}U_k \\V_k\end{pmatrix}= E_k \begin{pmatrix}
        U_k \\ V_k
    \end{pmatrix},
\end{equation}
Here, $h$ is the single-particle Hamiltanian, which reads
\begin{equation}
    h_{kk'} = t_{kk'} +\Gamma_{kk'},
\end{equation}
where $\Gamma_{kk'}$ is the mean field potential, given by
\begin{equation}
    \label{eq:mean_field_hfb}
    \Gamma_{kk'} = \sum_{ll'}\overline{v}_{kl'k'l}\rho_{ll'}
    \end{equation}
and the pairing field $\Delta$ reads
\begin{equation}
    \label{eq:pairing_field_hfb}
    \Delta_{kk'} = \sum_{ll'}\overline{v}_{kk'll'}\kappa_{ll'}.
\end{equation}
In the canonical basis, we are able to solve for the occupation numbers
\begin{equation}
    \label{eq:occ_hfb}
    u_k^2 = \frac 1 2 \bigg(1+\frac{h_{kk}+h_{\tilde k \tilde k}}{\sqrt{(h_{kk}+h_{\tilde k \tilde k})^2+4\Delta_{k\tilde k}}}\bigg)
\end{equation}
where $v_k^2 = 1 - u_k^2$ is guaranteed by the unitarity of the matrices.
Starting from an initial guess, we solve the eigenvalue problem \eqref{eq:eighfb}, we extract the occupation numbers \eqref{eq:occ_hfb}, use them to build the new mean field \eqref{eq:mean_field_hfb} and pairing field \eqref{eq:pairing_field_hfb}, and repeat the process until convergence.
\paragraph{HFB quasi-particle spectrum} Let us assume that $\Psi = (U,\  V )^T$ is a solution of equation \eqref{eq:eighfb} with eigenvalue $E$
\begin{equation}
    \label{eq:eigsolhfb}
    \mathcal H_\text{HFB}\Psi = E\Psi.
\end{equation}
Let the particle-hole matrix $\mathcal C$ be defined as
\begin{align}
    \mathcal C = \begin{pmatrix}
        0 & I \\ I&0
    \end{pmatrix},
\end{align}
it's trivial to show that 
\begin{equation}
    \label{eq:hfb_anticomm}
    \mathcal C \mathcal H_\text{HFB} \mathcal C = -\mathcal H_\text{HFB}^*,
\end{equation}
and 
\begin{equation}
    \label{eq:commutationhfb}
    \mathcal C = \mathcal C ^{-1}\implies \mathcal C\mathcal H_\text{HFB} = -\mathcal H_\text{HFB}\mathcal C.
\end{equation}
If we take the complex conjugate of equation \eqref{eq:eigsolhfb}, we get
\begin{equation}
    \label{eq:eigsolhfb_conj}
    \mathcal H_\text{HFB}^*\Psi ^* = E\Psi^*,
\end{equation}
if we multiply both sides on the left by $\mathcal C$ and use \eqref{eq:commutationhfb}, we get
\begin{align}
    -\mathcal H_\text{HFB}\mathcal C\Psi ^* &= E\mathcal C\Psi^*,\\
    \mathcal H_\text{HFB}\mathcal C\Psi ^* &= -E\mathcal C\Psi^*,
\end{align}
meaning that $\mathcal C \Psi^*$ is a solution of the eigenvalue problem \eqref{eq:eighfb} as well, with eigenvalue $-E$, hence for every quasi-particle energy we have a corresponding opposite-sign one; moreover, it can be proven that the HFB hamiltonian is unbounded, both from below and above \cite{Pei2012_HFBcontinuum}. This feature poses a challenge for numerical solutions of the HFB problem, as we shall see in chapter \ref{chap:numerical}.




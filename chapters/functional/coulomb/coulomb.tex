\section{Coulomb interaction}
\label{sec:coulomb_treatment}
Unlike the Skyrme interaction, the Coulomb force is finite-range, giving rise to an unwanted integral operator in the single-particle Hamiltonian.
A well known and widely used device is the Slater approximation \cite{SlaterApp}, which gives a local exchange interaction.
\\In this approximation, the Coulomb energy reads
\begin{align*}
    E_\text{Coul} = \int \mathcal E_\text{Coul}(\mathbf r) d\bm r
\end{align*}
where the energy density is given by
\begin{align}
    \mathcal E_\text{Coul}(\bm r) = \frac{e^2}{2}\bigg[\int  \frac{\rho_p(\mathbf r )\rho_p(\mathbf r ' )}{|\mathbf r-\mathbf r'|}d\mathbf r'  - \frac 3 2 \bigg(\frac 3 \pi \bigg) ^{\frac 1 3}\rho_p^{4/3}(\mathbf r)\bigg].
\end{align}
which results in the Coulomb potential field
\begin{equation}
    U_{C}(\mathbf r) = \fdv{\mathcal E_\text{Coul}}{\rho_p} = \frac{e^2}{2}\bigg[\int \frac{\rho_p(\mathbf r ')}{|\mathbf r-\mathbf r'|} d^3 \mathbf r' - 2\bigg(\frac 3 \pi \bigg) ^{\frac 1 3} \rho_p^{1/3}(\mathbf r ) \bigg]
\end{equation}
where the first term is the direct Coulomb interaction, which simply is the Coulomb energy generated by the proton density, while the second term is the exchange Coulomb interaction, which is local and depends on the proton density through a power factor of $1/3$.
From a computational standpoint, the exchange part is trivial, while the direct one is more involved.
One could compute the integral, but the complexity on a 3D mesh grows as $\mathcal O(N^6)$, where N is the total number of points on the mesh, rendering it unfeasible for fine calculations. 
\\An alternative approach is to solve the Poisson equation (from now on, $V_c$ refers to the direct part only)
\begin{equation}
    \label{eq:poisson}
    \nabla^2 V_c = 4\pi e^2 \rho_p.
\end{equation}
Given the proton density, we can impose Dirichlet boundary conditions, which can be extracted from a quadrupole expansion of the charge density \cite{Jackson1998}
\begin{equation}
V_c (\mathbf r) = 4\pi e^2 \sum_{\lambda=0}^2\sum_{\mu=-\lambda}^\lambda \frac{\expval{Q_{\lambda\mu}} Y_{\lambda\mu}}{r^{1+\lambda}}\text{ on }\partial \Omega
\end{equation}
where $\expval{Q_{\lambda\mu}}$ is defined as 
\begin{equation}
    \expval{Q_{\lambda\mu}} = \int r^\lambda Y_{\lambda\mu}^* (\mathbf r)\rho_p(\mathbf r ) d^3 \mathbf r
\end{equation}
Since we expect a charge density confined to the nuclear shape, higher order terms in the expansion can be neglected, provided that the box is sufficiently large.
\\In a reference frame where the nucleus center of mass is at the origin, the expansion reduces to
\begin{equation}
    V_{c}(\mathbf r ) = \frac{Ze^2}{r} + e^2\sum_{\mu=-2}^{2}\frac{\expval{Q_{2\mu}}Y_{2\mu}}{r^3} \text{ on } \partial \Omega.
\end{equation}
The reader can refer to appendix \ref{sec:spherical_harmonics} for the definition and numerical evaluation of the spherical harmonics $Y_{\lambda\mu}$.
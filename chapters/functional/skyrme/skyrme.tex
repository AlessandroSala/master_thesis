\section{Skyrme}
\label{sec:skyrme}
Now that the theoretical framework is clear, we can investigate a plausible nucleonic interaction, which in the present work, takes the form of the Skyrme force.
\subsection{Skyrme force}
It was first proposed by Tony Skyrme in 1958 \cite{SKYRME1958615} as a zero range force between nucleons, and has been used successfully as the building block of theoretical nuclear structure.
It comprises a two body attractive term that reads
\begin{align*}
v^{(2)}(\mathbf{r}_1, \mathbf{r}_2) &= t_0 \left(1 + x_0 P_\sigma \right) \delta(\mathbf{r}) \\
&\quad + \frac{1}{2} t_1 \left(1 + x_1 P_\sigma \right) \left[ \mathbf{P}'^2 \delta(\mathbf{r}) + \delta(\mathbf{r}) \mathbf{P}^2 \right] \\
&\quad + t_2 P_\sigma  \mathbf{P}' \cdot \delta(\mathbf{r}) \mathbf{P} \\
&\quad + i W_0 \boldsymbol{\sigma}\cdot \left[ \mathbf{P}' \times \delta(\mathbf{r}) \mathbf{P} \right]
\end{align*}
And a three body interaction, that is
\begin{equation*}
    v^{(3)}(\bm r_1, \bm r_2, \bm r_3) = t_3\delta(\bm r_1-\bm r_2)\delta(\bm r_2 -\bm r_3)
%v^{(3)}(\mathbf r_1, \mathbf r_2)=\frac 1 6 t_3 \left(1 + x_3 P_\sigma \right) \left[ \rho(\mathbf{R}) \right]^\sigma \delta(\mathbf{r}) 
\end{equation*}
which mimics the repulsive three-body force; without it, a collapse of the nuclear density would occur.
\\It's trivial to show that the three-body term is equivalent to a two-body, density-dependent interaction: \cite{VauhBrinkOriginal}
\begin{equation}
v^{(3)}(\bm r_1, \bm r_2) = \frac 1 6 t_3 (1+P_\sigma)  \delta(\bm r )\rho(\bm R)
\end{equation}
The different operators here are defined as
\begin{equation}
\mathbf{r} = \mathbf{r}_1 - \mathbf{r}_2\quad \mathbf{R} = \frac{\mathbf{r}_1+\mathbf{r}_2}{2}
\end{equation}
which are respectively the relative position of two particles and their center of mass coordinate, assuming equal masses.
\begin{equation}
\mathbf{P} = \frac{-i(\nabla_1 - \nabla_2)}{2}
\end{equation}
which is the so called relative wave-number operator.
\begin{equation}
\boldsymbol{\sigma} = \boldsymbol{\sigma}_1 + \boldsymbol{\sigma}_2
\end{equation}
being the total spin of the two interacting particles, and lastly
\begin{equation}
    \mathbf{P}_\sigma = \frac{(1+\boldsymbol{\sigma}_1\cdot\boldsymbol{\sigma}_2)}{2}
\end{equation}
which represents the spin-exchange operator.
Primed operators refer to the adjoint acting on the left.
\\The zero-range characteristic takes the form of a Dirac delta $\delta(\mathbf r)$, which allows the writing of the Fock term detailed in (REF) as a purely local one.
\\Taking the expectation value of the many body hamiltonian, in the Hilbert space of Slater determinants, yields an energy density which can be expressed as a function of $\rho_q, \tau_q, \bm J_q$ \cite{VauhBrinkOriginal}.
\begin{equation}
    \label{eq:skyrme_int_en_dens}
    \expval{H} = \bra{\Psi} H \ket{\Psi} = \int \mathcal H(\bm r) d\mathbf r
\end{equation}
\subsubsection{Modern parametrization}
The Skyrme force has evolved from the original one to accomodate new nuclei, done through the addition of a few parameters, yielding the following form of the interaction \cite{CHABANAT1997710}
    \begin{align*}
v^{(2)}(\mathbf{r}_1, \mathbf{r}_2) &= t_0 \left(1 + x_0 P_\sigma \right) \delta(\mathbf{r}) \\
&\quad + \frac{1}{2} t_1 \left(1 + x_1 P_\sigma \right) \left[ \mathbf{P}'^2 \delta(\mathbf{r}) + \delta(\mathbf{r}) \mathbf{P}^2 \right] \\
&\quad + t_2 \left(1 + x_2 P_\sigma \right) \mathbf{P}' \cdot \delta(\mathbf{r}) \mathbf{P} \\
&\quad + \frac{1}{6} t_3 \left(1 + x_3 P_\sigma \right) \left[ \rho(\mathbf{R}) \right]^\sigma \delta(\mathbf{r}) \\
&\quad + i W_0 \boldsymbol{\sigma}\cdot \left[ \mathbf{P}' \times \delta(\mathbf{r}) \mathbf{P} \right]\\
&\quad + \frac 1 6 t_3 \left(1 + x_3 P_\sigma \right) \left[ \rho(\mathbf{R}) \right]^\sigma \delta(\mathbf{r}) 
\end{align*}
Here, the boundary between Hartree-Fock and DFT starts to thin out, as the exponent $\sigma$ of the density makes that piece of the force a true three-body interaction only for the value $\sigma=1$ \cite{Erler_2010}.
\\On top of that, additional, empirical tweaking of the resulting energy density needed to reach satisfactory physical accuracy, such as the case for the spin-orbit couplings \cite{REINHARD1995467}, prompts for the following, well established proceeding: use the Skyrme interaction to obtain the energy density in equation \eqref{eq:skyrme_int_en_dens} and use it as a starting point to build an energy density functional and employ DFT.
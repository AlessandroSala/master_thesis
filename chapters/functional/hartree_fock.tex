\section{Hartree-Fock theory}
An empirical description of nuclear structure can be carried out using phenomenological models, as reported in section (REF).
\\A more rigorous approach needs to take into account the fact that the mean field which the nucleons interact with, is generated by the nucleons themselves, due to some microscopic interaction.
Starting from the many-body Hamiltonian of the system, we will be able to extract a single particle hamiltonian, where the nucleon is subject to a mean field potential generated by the effective microscopic force.
\\The many-body hamiltonian of the system, made of $A$ fermions, given by
\begin{equation}
    \label{eq:mb_hamiltonian}
    \hat H = \hat T + \hat V = \sum_i -\frac{\hbar^2}{2m}\nabla^2_i + \sum_{i<j} v^{(2)}_{ij} + \sum_{i<j<k} v^{(3)}_{ijk }
\end{equation}
which acts on the slater determinant given by 
\begin{equation}
    \label{eq:slater_formula}
    \Psi = \frac{1}{\sqrt {A!}} \sum_{\{p\}} (-1)^{p}  \varphi_{p(1)}(\bm r_1)\ldots \varphi_{p(A)}(\bm r_A)
\end{equation}
where $\varphi_i$ are single-particle states.
The slater determinant sums over all possible permutations of the $A$ fermions on the single particle states, with a $-$ sign according to the parity of the permutation.
\subsection{Hartree-Fock equations}
It is possible to show \cite{ring2004nuclear} that the ground state of the many-body system, found by minimizing the functional
\begin{equation}
    \label{eq:functional_hf}
    E[\Psi] = \frac{\bra{\Psi} \hat H \ket{\Psi}}{\bra{\Psi} \ket{\Psi}}
\end{equation}
Is found in the basis of eigenstates of the single-particle hamiltonian, found by setting to $0$ the functional variation of $E[\Psi]$ in $\varphi_i^*$ with the orthonormality constraint.
\begin{equation}
    \label{eq:fdv_hf}
\fdv{}{\varphi_i^*} \bigg(E[\Psi] - \lambda \int \varphi_j^*\varphi_i d\bm r \bigg) = 0
\end{equation}
Doing this yields the \textit{Hartree-Fock equations}
\begin{align}
    \label{eq:hf_equations}
    -\frac{\hbar^2}{2m} \nabla^2 \varphi_i +\sum_{j}^A\int \varphi_j ^* (\bm r') v_{ij}(\bm r, \bm r')\varphi_j (\bm r') \varphi_i (\bm r) d\bm r' - \sum_j^A\int \varphi_j^* (\bm r') v_{ij}(\bm r, \bm r')\varphi_j (\bm r)\varphi_i (\bm r') d\bm r' = \varepsilon_i \varphi_i
\end{align}
Here, a couple of observations are in order.
\\The first interaction term, called Hartree term, arises from considering independent particles, and is also routinely found in classical physics. The second one, called Fock term, or exchange term is non-local and is given by the quantum mechanical nature of the system.
\\From the standpoint of the solution of the eigenvalue problem, the Fock term is very problematic, and is usually avoided using finite-range interaction like the Gogny force \cite{Robledo_2019}, or contact forces like the Skyrme one \cite{SKYRME1958615}, used in the present work. They render the exchange term as a local one.
\\Even if the interaction terms are local, the equation is still highly non-linear, since the mean field potential will be a function of the various eigenstates. The consequece is that the equation will be solved \textit{self-consistently}, that is, by solving for the set of orbitals $\{\varphi_i\}$, using them to build the new mean field, and solving again, repeating the process until convergence.
\subsection{Symmetries}
Since the objective of this work is to solve the Hartree-Fock equations without spatial symmetry assumptions, it is useful to first understand how symmetries propagate along the self-consistent calculation.
\\We start by defining the creation and annihilation operators of the single particle hamiltonian eigenstates, $a_i^\dagger, a_i$, which abide the usual anticommutation relations of fermions
\begin{equation}
    \label{eq:fermion_anticommutations}
    \{a_i, a_j^\dagger\} = \delta_{ij}
\end{equation}
If we expand on a different, orthonormal complete basis $\{\chi_l\}$, we can write the corresponding creation and annihilation operators $c_l^\dagger, c_l$ as
\begin{align}
    \label{eq:basis_change}
    \varphi_k = \sum_l D_{lk} \chi_l\\
    a_k^\dagger = \sum_l D_{lk} c_l^\dagger\\
    a_k = \sum_l D_{lk}^\dagger c_l
\end{align}
Since orthonormality is guaranteed for both sets, taking 
\begin{equation}
    \delta_{jk} = \bra{\varphi_j}\ket{\varphi_k} = \sum_{ll'} D_{l'j}^\dagger D_{lk}\bra{\chi_l}\ket{\chi_l'}\implies DD^\dagger = 1
\end{equation}
\subsubsection{Symmetry propagation}
COMPLETARE
\subsection{Density Functional Theory}
It shall be evident shortly, in section \ref{sec:skyrme}, that a more general approach to microscopic models has to be taken, in order to give a satisfactory description of the nuclear world.
\\The framework that we'll briefly outline here is called Density Functional Theory (DFT). DFT was introduced by P. Hohenberg and W. Kohn in 1964 \cite{HK}, by proving two theorems.
\\The \textbf{first} HK theorem states that the energy of a fermion system, subject to an external potential $V_\text{ext}$ can be expressed solely as a functional of the particle density $\rho$ of the system.
\begin{equation}
    \label{eq:hk_theorem1}
    E[\rho] = F[\rho] + \int V_\text{ext} \rho(\bm r) d\mathbf r
\end{equation}
While the \textbf{second} one states that the ground state of the system is found by minimizing its variation with respect to $\rho$. 
\\HK theorems are fundamental but not constructive \cite{NDFT}, since they do not provide a form for the functional $F$, which is intrinsic to the physics of the fermions at hand.
COMPLETARE




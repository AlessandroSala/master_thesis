\section{Hartree-Fock theory}
\label{sec:hf}
While a phenomenological description of bulk nuclear structure properties can be carried out using the liquid drop model or empirical mean-field potentials like Woods-Saxon or Nilsson, as we have seen in section \ref{sec:models}, this is not sufficient to accurately reproduce all experimental ground state observables, systematically throughout the chart of nuclei, in an accurate manner. 
\\A more rigorous approach needs to take into account the fact that the mean field which the nucleons interact with, is generated by the nucleons themselves.
Starting from the many-body hamiltonian of the system, we will be able to extract a single particle Hamiltonian, including an effective mean field potential generated by the effective microscopic force, through the use of the Hartree-Fock method.
\\We start by writing the many-body hamiltonian of the system, which is a collection of A interacting fermions, given by
\begin{equation}
    \label{eq:mb_hamiltonian}
    \hat H = \hat T + \hat V = \sum_i -\frac{\hbar^2}{2m}\nabla^2_i + \sum_{i<j} v^{(2)}_{ij} + \sum_{i<j<k} v^{(3)}_{ijk }+\ldots
\end{equation}
The corresponding Schr\"odinger equation reads
\begin{equation}
    \label{eq:many_body_schrodinger}
    \hat H \Psi = E\Psi.
\end{equation}
\subsection{Variational principle}
Since $\hat H$ is a many body operator, finding its eigenstates would be a rather challenging task. To our aid, comes the variational principle, from which we can show that equation \eqref{eq:many_body_schrodinger} is equivalent to 
\begin{equation}
    \label{eq:variational_Psi}
    \delta E[\Psi] = \delta \frac{\bra{\Psi} E \ket{\Psi}}{\bra{\Psi}\ket{\Psi}} = 0.
\end{equation}
The variation \eqref{eq:variational_Psi} can be obtained from an arbitrary variation of $\Psi$, which can be done indipendently on $\bra{\Psi}$ and $\ket{\Psi}$, since $\Psi$ is complex, yielding
\begin{equation}
    \label{eq:variational_Psi2}
    \bra{\delta \Psi} \hat H  -E\ket{\Psi} + \bra{\Psi} \hat H-E \ket{\delta \Psi} = 0
\end{equation}
since the variation is arbitrary, we can multiply by a phase factor $\ket{\delta \Psi}\mapsto i\ket{\delta \Psi}$ and get
\begin{equation}
    \label{eq:variational_Psi3}
    -i\bra{\delta \Psi} \hat H-E \ket{\Psi} + i\bra{\Psi} \hat H-E \ket{\delta \Psi} = 0.
\end{equation}
Combining equations \eqref{eq:variational_Psi2} and \eqref{eq:variational_Psi3}, we get
\begin{equation}
    \label{eq:var_eq_res}
    \bra{\delta \Psi} \hat H-E \ket{\Psi} = 0.
\end{equation}
Again, since the variation is arbitrary, equation \eqref{eq:var_eq_res} satisfies equation \eqref{eq:many_body_schrodinger}.
\subsubsection{Ground state}
Since we always restrict ourselves to a certain subspace of the full Hilbert space, we can only find an approximate solution to the eigenvalue problem. Expanding this solution on the complete set of exact eigenstates of $\hat H$, we have
\begin{equation}
    \label{eq:expansion_gs}
    \ket\Psi = \sum_n a_n \ket{\Psi_n}    
\end{equation}
the total energy amounts to 
\begin{equation}
    \label{eq:ground_state_min}
    E[\Psi] = \frac{\sum_{nn'}\bra {a_{n'}\Psi_{n'}}\hat H \ket{a_n \Psi_n}}{\sum_{nn'}\bra{a_{n'}\Psi_{n'}}\ket{a_n\Psi_{n}}} = \frac{\sum_{n}E_n|a_n|^2}{\sum_n|a_n|^2}\ge \frac{\sum_{n}E_0|a_n|^2}{\sum_n|a_n|^2}\ge E_0,
\end{equation}
where $E_0$ is the ground state energy of the system. The orthonormality of the Hamiltonian eigenfunctions $\bra{\Psi_{n'}}\ket{\Psi_n}=\delta_{nn'}$ has been used. Equation \eqref{eq:ground_state_min} tells us that the minimum of the functional $E[\Psi]$ in any variationl subspace we are considering is bound from below by the true ground state energy.
\subsection{Hartree-Fock equations}
The Hartree-Fock method is the application of the variational principle \eqref{eq:var_eq_res} to a system of independent particles, whose wavefunction takes the form of a Slater determinant, which reads
\begin{equation}
    \label{eq:slater_formula}
    \Psi = \frac{1}{\sqrt {A!}} \sum_{\{p\}} (-1)^{p}  \varphi_{p(1)}(\bm r_1)\ldots \varphi_{p(A)}(\bm r_A)
\end{equation}
where $\varphi_i$ are the single-particle orthonormal states, which serve the role of variational parameters in Hartree-Fock.
The Slater determinant sums over all possible permutations of the $A$ fermions on the single particle states, with a $-$ sign according to the parity of the permutation.
The Slater determinant statisfies the permutation symmetry of fermions, so that the Pauli exclusion principle is not violated.
\\To ensure the orthonormality of the single-particle states, we need to add a Lagrange multiplier to the variation \eqref{eq:var_eq_res} of $E$, which ends up reading
\begin{equation}
    \label{eq:fdv_hf}
\delta \bigg(E[\Psi] - \sum_i\lambda_i \int \varphi_i^*\varphi_i d\bm r \bigg) = 0.
\end{equation}
The total energy of the system is given by
\begin{equation}
    \label{eq:total_energy_HF}
    E[\Psi] = \bra{\Psi} \hat T + \hat V\ket \Psi = \bra{\Psi}\hat T \ket{\Psi}+ \bra{\Psi}\hat V \ket{\Psi},
\end{equation} 
which can be expressed through the single-particle states $\{\varphi_i\}$, yielding
\begin{align}
    \label{eq:exp_T_V}
    \bra\Psi\hat T\ket \Psi &= \sum_i^A -\frac{\hbar^2}{2m}\int \varphi_i^*(\bm r) \nabla^2 \varphi_i(\bm r) d\bm r=\sum_i \bra{i}t\ket{i}\\
    \bra\Psi\hat V\ket \Psi &= \frac 1 2 \sum_{ij}\int\varphi_i^*(\bm r) \varphi_j^*(\bm r') v_{ij}(\bm r, \bm r')\varphi_i(\bm r)\varphi_j(\bm r') d\bm r d\bm r'\\
    &- \frac 1 2 \sum_{ij}\int \varphi_i^*(\bm r) \varphi_j^*(\bm r') v_{ij}(\bm r, \bm r')\varphi_i(\bm r')\varphi_j(\bm r) d\bm r d\bm r' \\&= \frac 1 2 \bra{ij}\overline v \ket{ij}.
\end{align}
If we use $\varphi_i^*(\bm r)$ as the quantity to be varied in equation \eqref{eq:fdv_hf}, we get the \textit{Hartree-Fock equations}
\begin{align}
    \label{eq:hf_equations}
    &-\frac{\hbar^2}{2m} \nabla^2 \varphi_i\\
    &+ \frac 1 2 \sum_{j}^A \int \varphi_j^* (\mathbf r') 
        v_{ij}(\mathbf r, \mathbf r')\, 
        \varphi_j (\mathbf r')\, 
        \varphi_i (\mathbf r)\, d\mathbf r'\label{eq:hartree} \\
    &- \frac 1 2 \sum_{j}^A \int \varphi_j^* (\mathbf r')\, 
        v_{ij}(\mathbf r, \mathbf r')\,
        \varphi_j (\mathbf r)\,
        \varphi_i (\mathbf r')\, d\mathbf r'\label{eq:fock}
        = h \varphi_i = \varepsilon_i \varphi_i
\end{align}
here the Lagrange multipliers $\lambda_i$ have been replaced by $\varepsilon_i$, since they can be interpreted as the energy of the single-particle states.\\
Now, a couple of remarks are in order.
\paragraph{Exchange interaction}
The first interaction term \eqref{eq:hartree}, called Hartree term, arises from considering independent particles, and is also routinely found in classical physics. The second one in \eqref{eq:fock}, called Fock term, or exchange term, takes the form of an integral operator and is present when considering quantum mechanical indistinguishable particles.
\\For what concerns the solution of equation \eqref{eq:hf_equations}, the Fock term is problematic, and can be avoided using zero-range forces like the Skyrme one \cite{SKYRME1958615}, which is used in the present work, since it renders the exchange term as a local one.
\paragraph{Self-consistent solution}
Even if the interaction terms are local, the equation is still highly non-linear, since the mean field potential will be a function of the eigenfunctions themselves. The consequece is that the solution needs to be found \textit{self-consistently}, that is, by solving for the set of eigenfunctions $\{\varphi_i\}$, using them to build the new mean field, and solving again, repeating the process until convergence.
\subsection{Symmetries in Hartree-Fock}
Since the objective of this work is to solve the Hartree-Fock equations without spatial symmetry assumptions, it is useful to first understand how symmetries propagate along the self-consistent calculation.
\\We start by defining the creation and annihilation operators of the single particle hamiltonian eigenstates, $a_i^\dagger, a_i$, which abide the usual anticommutation relations of fermions
\begin{equation}
    \label{eq:fermion_anticommutations}
    \{a_i, a_j^\dagger\} = \delta_{ij}
\end{equation}
If we expand on a different, orthonormal complete basis $\{\chi_l\}$, we can write the corresponding creation and annihilation operators $c_l^\dagger, c_l$ as
\begin{align}
    \label{eq:basis_change}
    \varphi_k = \sum_l D_{lk} \chi_l\\
    a_k^\dagger = \sum_l D_{lk} c_l^\dagger\\
    a_k = \sum_l D_{lk}^\dagger c_l
\end{align}
Since orthonormality is guaranteed for both sets, taking $\bra{\varphi_j}\ket{\varphi_k}$ yields 
\begin{equation}
    \label{eq:Disunitary}
    \delta_{jk} = \bra{\varphi_j}\ket{\varphi_k} = \sum_{ll'} D_{l'j}^\dagger D_{lk}\bra{\chi_l}\ket{\chi_l'}\implies DD^\dagger = 1,
\end{equation}
meaning that $D$ is a unitary transformation.
We can define the density matrix as
\begin{equation}
    \label{eq:density_op}
    \rho_{ll'} = \bra{\Psi}c_{l'}^\dagger c_l\ket{\Psi},
\end{equation}
whose trace is equal to the particle number $A$, as per equation \eqref{eq:trace_rho}
\begin{equation}
    \label{eq:trace_rho}
   \Tr \rho =  \sum_{ll}\bra{\Psi}c_{l}^\dagger c_l\ket{\Psi} = \sum_{ll}\bra{\Psi}a_l^\dagger a_l\ket{\Psi} = \sum_{ll}^A \bra{\Psi}\ket{\Psi} = A.
\end{equation}
\\Writing the many body hamiltonian \eqref{eq:many_body_schrodinger} in the arbitrary basis of second quantization operators $c_l^\dagger, c_l$, we get
\begin{equation}
    \label{eq:mb_hamiltonian_sq}
    \hat H = \sum_{l_1 l_2}t_{l_1l_2} c_{l_1}^\dagger c_{l_2} + \frac 1 4 \sum_{l_1 l_2 l_3 l_4}\overline{v}_{l_1l_2l_3l_4} c_{l_1}^\dagger c_{l_2}^\dagger c_{l_3} c_{l_4}
\end{equation}
where $t_{l_1l_2}$ and $\overline{v}_{l_1 l_2 l_3 l_4}$ are defined as 
\begin{align}
    \label{eq:t_op}
    t_{l_1l_2} &= \bra{-}c_{l_1}c_{l_2} t c_{l_1}^\dagger c_{l_2}^\dagger \ket{-} = \bra{l_1 l_2}t\ket{l_1 l_2}
    \\ \overline{v}_{l_1 l_2 l_3 l_4} &= \bra{l_1 l_2 l_3 l_4} v \ket{l_1 l_2 l_3 l_4} - \bra{l_1 l_2 l_4 l_3} v \ket{l_1 l_2 l_4 l_3}.
\end{align}
The minimization \eqref{eq:fdv_hf} can be restated as the variation of $\bra{\Psi}\hat H \ket{\Psi}$, with respect to the density matrix $\rho_{ll'}$, which yields
\begin{equation}
    \label{eq:spe_scnd}
    h_{ll'} = \pdv{E[\rho]}{\rho_{ll'}} = t + \sum_{kk'}\overline{v}_{lk'l'k}\rho_{kk'} = t+\Gamma_{ll'}
\end{equation}
where $\Gamma_{ll'}$ is the mean field potential in the arbitrary basis.
Being $h$ diagonal in the Hartree-Fock basis, the self-consistent solution is the one for which
\begin{equation}
    [h, \rho] = 0
\end{equation}
holds.

\subsubsection{Symmetry propagation}
Let us suppose to start a Hartree-Fock calculation with an initial guess for which the corresponding density matrix $\rho^{(0)}$ is symmetric under the action of a many-body symmetry operator $S$ which commutes with the Hamiltonian
\begin{equation}
    \label{eq:comm_S_H}
    [S, \hat H] = 0.
\end{equation}
It can be shown \cite{ring2004nuclear} that 
\begin{equation}
    \label{eq:comm_S_Gamma}
    S\Gamma[\rho] S^\dagger  = \Gamma[S\rho S^\dagger].
\end{equation}
The single particle Hamiltonian $h$ will then display the same property
\begin{equation}
    \label{eq:comm_h_S}
    S h[\rho] S^\dagger = h[S\rho S^\dagger] = h[\rho]
\end{equation}
meaning that $h$ will be symmetric under the action of $S$, as well as the next iteration's density matrix $\rho^{(1)}$. The symmetry $S$ gets propagated self-consistently until the minimum is found.
\\This has profound numerical implications, since the minimum energy configuration of a deformed nuclei can be reached only by starting guesses with the same broken symmetries. It can be the case that numerical noise allows to explore the full energy surface, but if one has to take into consideration the numerical cost of a bad guess, then it's still advantageous to start from a correct one in terms of symmetries.
%\subsubsection{Broken symmetries}

\subsection{Density Functional Theory}
\label{sec:dft}
It shall be evident shortly, in section \ref{sec:skyrme}, that a more general approach to microscopic models has to be taken in order to give a satisfactory description of nuclei and nuclear matter. The method that we will briefly outline here is called Density Functional Theory (DFT).
\\DFT was introduced by P. Hohenberg and W. Kohn in 1964 \cite{HK}, by proving two theorems.
\\The \textbf{first} Hohenberg Kohn (HK) theorem states that the energy of a fermion system, subject to an external potential $V_\text{ext}$ can be expressed solely as a functional of the particle density $\rho$ of the system
\begin{equation}
    \label{eq:hk_theorem}
    E[\rho] = F[\rho] + \int V_\text{ext} \rho(\bm r) d\mathbf r,
\end{equation}
where $F[\rho]$ is a universal functional given by the type of fermions considered, while $V_\text{ext}$ term is the external potential to which the system is subject to; when treating atomic nuclei, the potential is generated by the nucleons themselves, so this term will be omitted in the following.
The \textbf{second} HK theorem states that the ground state of the system is found by minimizing the functional \eqref{eq:hk_theorem} with respect to $\rho$. 
\\HK theorems are fundamental but not constructive \cite{NDFT}, since they do not provide a form for the functional $F$, which is intrinsic to the physics of the fermions at hand.
A pragmatic approach to using DFT was outlined by Kohn and Sham in 1965 \cite{KS}. They proposed expressing the system as a set of non-interacting particles occupying auxiliary orbitals $\varphi_i$, which yield the particle density
\begin{equation}
    \rho(\bm r)=\sum_i |\varphi_i(\bm r)|^2 
\end{equation}
and an energy functional of the form
\begin{equation}
    E[\rho] = T[\rho] + E_H[\rho] + E_\text{xc}[\rho].
\end{equation}
where $T$ is the kinetic energy, which reads
\begin{equation}
    \label{eq:kin_functional}
    T[\rho] = -\frac{\hbar^2}{2m}\sum_i \varphi_i ^*(\bm r)\nabla^2 \varphi_i(\bm r)
\end{equation}
and $E_H$ is the classical Hartree term, which in an electronic system may read
\begin{equation}
    \label{eq:elec}
    E_H[\rho] = \iint \frac{\rho(\bm r)\rho(\bm r')}{|\bm r-\bm r'|} d\bm r d\bm r'
\end{equation}
while $E_\text{xc}$ is an unknown exchange term. In electronic systems, the Hartree term is known \eqref{eq:hartree} and the exchange term can be approximated thanks to the compensation of its error with the one of particles correlation neglection \cite{Martin2004}. In nuclear physics, things are more complicated, since both terms are unknown; historically, effective interactions in HF have been used to extract an effective Hamiltonian density from which an energy density functional (EDF) can be formulated, whenever a pure interaction is not sufficient to describe nuclear systems, as we shall see in section \ref{sec:skyrme}.
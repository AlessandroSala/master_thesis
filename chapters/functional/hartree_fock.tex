\section{Hartree-Fock theory}
An empirical description of nuclear structure can be carried out using phenomenological models, as reported in section (REF).
\\A more rigorous approach needs to take into account the fact that the mean field which the nucleons interact with, is generated by the nucleons themselves, due to some microscopic interaction.
\\The many-body hamiltonian of the system, given by
\begin{equation}
    \label{eq:mb_hamiltonian}
    \hat H = \hat T + \hat V = \sum_i -\frac{\hbar^2}{2m}\nabla^2_i + \sum_{i<j} v^{(2)}_{ij} + \sum_{i<j<k} v^{(3)}_{ijk }
\end{equation}
acts on the nucleus, a system of $A$ nucleons described by the Slater determinant
\begin{equation}
    \label{eq:slater_formula}
    \Psi = \frac{1}{\sqrt {A!}} \sum_{\{p\}} (-1)^{p}  \varphi_{p(1)}(\bm r_1)\ldots \varphi_{p(A)}(\bm r_A)
\end{equation}
That is, summing over all possible permutations of the $A$ fermions on the single particle states, with a $-$ sign according to the parity of the permutation.
\subsection{Variational principle}
It is possible to show \cite{ring2004nuclear} that the ground state of the many-body system, found by minimizing the functional
\begin{equation}
    \label{eq:functional_hf}
    E[\Psi] = \frac{\bra{\Psi} \hat H \ket{\Psi}}{\bra{\Psi} \ket{\Psi}}
\end{equation}
Is equivalent to the 




\section{Functional}


\section{Energy calculation}
One, if not the most important physical quantity we want to compute is the total energy of the system.
\\The obvious way would be to evaluate the functinal for a given density. We will call this \textit{integrated energy}.
\begin{align}
    E_\text{int} = E[\rho, \tau, J_{\mu\nu}]= \int \mathcal E d\mathbf r
\end{align}
An alternative approach is to realize that, in a stationary point $\delta E = 0$, the single particle eigenvalue equation stands true
\begin{align}
    \label{eq:simple_spe}
    (\hat t + U)\varphi_k = \varepsilon_k \varphi_k
\end{align}
We can multiply \ref{eq:simple_spe} on the left by $\varphi_k^*$ and integrate to get
\begin{align}
    \label{eq:int_simple_spe}
    \int -\varphi_k^* \frac{\hbar^2}{2m}\nabla^2\varphi_k d\bm r + \int \varphi_k^* U \varphi_k d\bm r = \int \varphi_l^* \varepsilon_k \varphi_k d\bm r
\end{align}
The integral on the right hand side of \ref{eq:int_simple_spe} evaluates to $\varepsilon_k$ due to the orthonormality constraint. If we sum over all states $k$ we get
\begin{align}
    \sum_k \bigg\{\int -\varphi_k^* \frac{\hbar^2}{2m}\nabla^2\varphi_k d\bm r + \int \varphi_k^* U \varphi_k d\bm r \bigg\}= \sum_k \varepsilon_k 
    \\\sum_k t_k + \int \rho U = \sum_k \varepsilon_k \label{eq:sum_spe}
\end{align}
Since $U$ is calculated as \ref{eq:fdv_rho_skyrme}, assuming that the functional has a power dependence from $\rho$ of the form $\mathcal E = A\rho^{\sigma+1}$ as in our case, we get the \textit{rearrangement energy}
\begin{align}
    \label{eq:sum_spe_new}
    \rho U = \rho \fdv{\mathcal E}{\rho} = \rho(\sigma +1)A \rho^\sigma = (\sigma+1)A \rho^{\sigma+1} = \mathcal E + \sigma \mathcal E = \mathcal E + \mathcal E_\text{rea}
\end{align}

Given the energy of the total system from \ref{eq:slater_exp}, we have
\begin{align}
\sum_k t_k + \frac 1 2 \int \mathcal E d\bm r = E
\end{align}
If we explicit $\mathcal E$ in equation \ref{eq:sum_spe} using \ref{eq:sum_spe_new}, we finally get to
\begin{align}
    \int \mathcal Ed\bm r = \sum_k (\varepsilon_k -t_k) - \sigma\int \mathcal Ed\bm r
\end{align}
Substituting it in equation \ref{eq:sum_spe_new} yields
\begin{align}
    \label{eq:hf_energy}
E_\text{HF} = \frac 1 2 \sum_k (\varepsilon_k + t_k) - \sigma \int \mathcal E d\bm r = \frac 1 2 \bigg(T+\sum_k\varepsilon_k\bigg) +E_\text{rea}
\end{align}
Which will be called \textit{Hartree-Fock energy} throughout this text.
\subsubsection{Sidenote: actual calculation}
As shown in \ref{eq:coefficients_func}, the actual functional has a plethora of $\rho$ terms, which can be summarized as
\begin{equation}
    \mathcal E = \sum_j A_j \rho^{\sigma_j+1} \implies E_\text{rea} = -\sum_j \sigma_j A_j \rho^{\sigma_j+1} 
\end{equation}
This means that only terms with a $\sigma_j\neq 1, -1$ actually contribute to the rearrangement energy.
\\Since equation \ref{eq:hf_energy} is valid only in the minimum of the functional, it's useful in Hartree-Fock calculations to check its equivalence with the integrated energy, so one can be sure to actually be in a minimum.
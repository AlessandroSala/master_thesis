\section{Energy calculation}
One, if not the most important physical quantity we want to compute is the total energy of the system.
\subsubsection{Integrated energy}
The obvious way would be to evaluate the functional for a given density. We will call this \textit{integrated energy}.
\begin{align*}
    E_\text{int} = E[\rho, \tau, J_{\mu\nu}]= \int \mathcal E d\mathbf r
\end{align*}
\subsubsection{Hartree-Fock energy}
An alternative approach can be used, as in a stationary point $\delta E = 0$, the single particle eigenvalue equation \eqref{eq:spe_ks} stands true, summarized as 
\begin{align}
    \label{eq:simple_spe}
    ( t + U)\varphi_k = \varepsilon_k \varphi_k
\end{align}
We can multiply \eqref{eq:simple_spe} on the left by $\varphi_k^*$ and integrate to get
\begin{align}
    \label{eq:int_simple_spe}
    \int -\varphi_k^* \frac{\hbar^2}{2m}\nabla^2\varphi_k d\bm r + \int \varphi_k^* U \varphi_k d\bm r = \int \varphi_k^* \varepsilon_k \varphi_k d\bm r
\end{align}
The integral on the right hand side of \eqref{eq:int_simple_spe} evaluates to $\varepsilon_k$ due to the orthonormality constraint. If we sum over all states $k$ we get
\begin{align}
    \sum_k \bigg\{\int -\varphi_k^* \frac{\hbar^2}{2m}\nabla^2\varphi_k d\bm r + \int \varphi_k^* U \varphi_k d\bm r \bigg\}= \sum_k \varepsilon_k 
    \\\sum_k t_k + \int \rho U = \sum_k \varepsilon_k \label{eq:sum_spe}
\end{align}
Since $U$ is calculated as the functional derivative with respect to the density $\rho$ \eqref{eq:fdv_rho_skyrme}, assuming that the functional has a power dependence from $\rho$ of the form $\mathcal E_\text{Skyrme} = A\rho^{\sigma+1}$ as in our case, we get the \textit{rearrangement energy}
\begin{align}
    \label{eq:sum_spe_new}
    \rho U = \rho \fdv{\mathcal E_\text{Skyrme}}{\rho} = \rho(1+\sigma )A \rho^\sigma = (1+\sigma)A \rho^{\sigma+1} = \mathcal E_\text{Skyrme} + \sigma \mathcal E_\text{Skyrme} = \mathcal E_\text{Skyrme} - \mathcal E_\text{rea}
\end{align}
If we explicit $\rho U$ in equation \eqref{eq:sum_spe} using \eqref{eq:sum_spe_new}, we get to
\begin{align*}
    \sum_k t_k + \int (\mathcal E_\text{Skyrme}-\mathcal E_\text{rea}) d\bm r = \sum_k \varepsilon_k 
\end{align*}
Isolating the Skyrme energy density
\begin{align}
    \label{eq:int_E_rea}
    \int \mathcal E_\text{Skyrme} d\bm r = \sum_k (\varepsilon_k -t_k) + \int \mathcal E_\text{rea} d\bm r
\end{align}
and given the total energy of the system from \eqref{eq:slater_exp}
\begin{align}
    \label{eq:brief_tot_energy}
E=\sum_k t_k + \frac 1 2 \int \mathcal E_\text{Skyrme} d\bm r 
\end{align}
substituting \eqref{eq:int_E_rea} in \eqref{eq:brief_tot_energy} yields
\begin{align}
    \label{eq:hf_energy}
E_\text{HF} = \frac 1 2 \sum_k (\varepsilon_k + t_k) +\int \mathcal E_\text{rea} d\bm r = \frac 1 2 \bigg(T+\sum_k\varepsilon_k\bigg) +E_\text{rea}
\end{align}
which will is called \textit{Hartree-Fock energy}.
\paragraph{Sidenote}
The actual functional, including the Coulomb exchange term, has different $\rho$ terms, which can be summarized as
\begin{equation*}
    \mathcal E_\text{Skyrme} = \sum_j A_j \rho^{\sigma_j+1} \implies E_\text{rea} = -\sum_j \sigma_j A_j \rho^{\sigma_j+1} 
\end{equation*}
This means that only terms with a $\sigma_j\neq 0, -1$ actually contribute to the rearrangement energy.
\\Since equation \eqref{eq:hf_energy} is valid only for $\delta E = 0$, it's useful to check its equivalence with the integrated energy at convergence, so one can be sure to actually be in a stationary point.
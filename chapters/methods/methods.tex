\chapter{State of the art, objectives and methods}
\label{chap:methods}
\section{State of the art and motivation}
The need to account for nuclear deformations has been already highlighted as a central theme in Chapter~\ref{chap:intro}, particularly in the context of heavy nuclei and of the complex fission dynamics discussed in Section~\ref{sec:fission}. Without allowing for deformation, several key properties of nuclear systems, such as the rotational bands in the excitation spectrum since spherical nuclei cannot rotate, cannot be captured. Limiting the description to only a subset of shapes, such as axially symmetric configurations, often proves to be inadequate. This is evident in the case of fission dynamics, where the evolution stages happen through very deformed and exotic configurations, explored over relatively long time scales.

The central issue in nuclear structure theory is that the Hamiltonian is not known exactly and must be approximated. At the same time, the problem is inherently a quantum many-body one, which remains computationally demanding, requiring approximations to make calculations tractable. Symmetry assumptions such as spherical or axial symmetry have therefore been widely used in numerical implementations to simplify the calculations and reduce the computational cost. However, these constraints become insufficient in situations where deformation is a defining aspect of the system.

With the increasing availability of computational power and the development of modern numerical techniques, the field is now in a position to move beyond these restrictive assumptions. This motivates the development of new codes capable of treating the many-body problem without imposed symmetries, allowing the full range of nuclear deformations to emerge naturally from the underlying theory.

\section{State of the Art}
The approaches to the nuclear many-body problem can be divided into three families: shell model, ab initio, and Density Functional Theory.
\paragraph{Shell model}
In the interacting shell model, one assumes that a basis is constituted by nucleons in single-particle orbitals. While some orbitals are \emph{frozen}, there is a space called \emph{valence space} in which one includes all possible configurations with nucleons arbitrarily put in the valence orbitals. The nuclear Hamiltonian is diagnonalized in this configuration space. This model is very successful for spectroscopy but also computationally very demanding. Often, the predictive power is srongly dependent on the specific effective Hamiltonian that is assumed.
\paragraph{Ab initio}
One may be inclined towards studying nuclei starting from the fundamental interaction which binds them, the strong force. This would be done starting from the QCD Lagrangian, however, two problems arise. First, nucleons are bound states of quarks and gluons and what binds the nucleus together is the residual interaction arising from these bound systems. Second, the QCD coupling is large at the energy scale $\Lambda_\chi\approx 0.5-1.0$~GeV of nucleons and heavy mesons and as such cannot be treated perturbatively. This is at variance with the higher energy scales like $100-1000$ GeV where QCD is perturbative.

The solution to this problem is the use of effective chiral Lagrangians, equivalent to the QCD one at low energy scales. The interactions among the nucleons arise from the Feynman diagrams between two, three, or more nucleons, which must be hierarchically ordered, according to some perturbative parameter, usually denoted by $(Q/\Lambda_\chi)^\nu$, where $Q$ is the typical nucleon momentum. The power of $\nu$ at which the expansion stops determines the order of the potential that is calculated and used.

These methods are in principle exact, but require nevertheless a series of approximations to make calculations tractable; moreover, their applicability is currently limited to spherical and light deformed nuclei.

\paragraph{Density Functional Theory}
The issues brought forth by the shell model and the ab initio approaches may be addressed by using an \textit{effective} interaction among nucleons that is not derived from exact principles but reliably reproduces experimental data. This was done starting from the 1970s, by minimising the energy functional $\bra{\Psi}\hat{H}_\text{eff}\ket{\Psi}$, where $\hat{H}_\text{eff}$ is a properly designed effective Hamiltonian and $\Psi$ is a Slater determinant.

It was realised early on that the only way to obtain realistic results from this approach was to use density dependent forces, which somewhat account for the important many-body effects. The historical development of this realization has been to first use the Hartree--Fock expectation value of an effective interaction as a starting point for the design of an Energy Density Functional (EDF), which can be used to do nuclear structure calculations using Density Functional Theory. Later, it has been customary to directly write an EDF.

In particular, the minimisation of these EDFs has been done using two numerical approaches: (a) \textit{basis expansion methods}, which represent single-particle states on truncated bases, mainly harmonic oscillator bases, 
and (b) \textit{coordinate-space (mesh) methods}, which discretise space directly by using a mesh. 
In the following sections, we review these two classes of methods and motivate the need for more flexible and computationally efficient unconstrained solvers.

\subsection{Basis expansion methods}

Basis expansion approaches are among the most widely used techniques for solving the HF and HFB equations. 
In these methods, single-particle wavefunctions are expanded on a finite HO basis, chosen for its flexibility and easiness of use, together with the qualitative similarity to the mean-field potential of bound nuclei, as explained in Section~\ref{sec:models}. 
Codes such as \texttt{HFBTHO}, also used in this work for benchmarking our implementation in Section~\ref{sec:hfbtho}, are based on this framework.

The HO basis, although efficient, introduces inherent limitations.
First, weakly bound and continuum-like states, crucial for nuclei near the drip lines, are poorly represented because their asymptotic behaviour differs 
fundamentally from that of HO functions. 
Whereas HO states decay as $e^{-ar^{2}}$, quasi-bound states decay as $e^{-\kappa r}$, 
leading to slow convergence and difficulties in describing haloes, neutron skins, and quasi-resonant states 
\cite{Stoitsov2003_PRCC68_054312,Dobaczewski1996_PRCC53_2809}.  
Second, large deformations in heavy nuclei may require many HO shells to reproduce the stretched spatial geometry, significantly increasing the computational cost.  
The computational complexity grows rapidly with the maximum number of oscillator shells used in the calculation, resulting in demanding memory and CPU requirements for strongly deformed configurations \cite{Marevic2022_CPC}.

In summary, despite their efficiency for near-spherical and moderately deformed nuclei, 
basis-expansion methods become inadequate for describing nuclei near drip lines, far from stability and largely deformed.

\subsection{Symmetry-Restricted mesh methods}

A second major class of HF/HFB solvers uses a spatial mesh.  
Historically, fully unconstrained three-dimensional meshes were computationally prohibitive, which motivated the introduction of \textit{symmetry constraints} to reduce the dimensionality of the problem.  
By enforcing specific spatial symmetries, the number of degrees of freedom is greatly reduced, making coordinate-space calculations tractable on available hardware.

The most common choices are spherical and axial symmetry.  
Spherical HF/HFB solvers \cite{VauhBrinkOriginal,hfbcsqrpa} reduce the equations to a radial problem, achieving excellent computational efficiency and precision for the structure of spherical or near-spherical nuclei.  
Axially symmetric solvers \cite{Pei2008_HFBAX} generalise this approach to two dimensions, allowing axial deformations while still benefitting from significant computational cost reductions.

However, the limitations of symmetry-restricted approaches are inherent to the constraints themselves, as they forbid the emergence of intrinsic shapes such as triaxial or more general octupole-deformed configurations.

\subsection{Unconstrained coordinate-space (mesh) methods}
To overcome the limitations of basis truncation and symmetry constraints, modern HF / HFB solvers have increasingly adopted coordinate-space discretizations, typically based on three-dimensional Cartesian meshes. 
Notable examples include \texttt{MOCCa} \cite{Ryssens2016_MOCCa}, 
\texttt{Sky3D} \cite{Maruhn2014_Sky3D}, and \texttt{HFBFFT} \cite{CHEN2022108344}. 
These codes solve the HF or HFB equations directly in coordinate space, allowing arbitrary deformations and spontaneous symmetry breaking to emerge naturally.

However, coordinate-space solvers come with their own challenges.  
High spatial resolution is required to accomodate sufficient numerical accuracy, leading to large three-dimensional grids, thus substantial computational cost. 
Even with modern resources, fully unconstrained calculations remain computationally intensive, and additional assumptions such as plane reflection symmetry are often introduced to reduce the domain size \cite{Ryssens2015_EV8,Ryssens2016_MOCCa}.
Thus, while mesh-based solvers offer maximal flexibility, their computational demands motivate the search for more efficient numerical approaches.

\subsection{Towards more efficient unconstrained methods}
The limitations discussed above highlight the need for methods that combine the flexibility of coordinate-space solvers with improved computational efficiency.
In this thesis, we investigate such an approach through the use of the \textit{Generalised Conjugate Gradient} (GCG) method, presented in detail in section~\ref{sec:gcg}.
In the HF and energy-density-functional frameworks, the core of the many-body problem reduces to solving a set of single-particle Schrödinger or Kohn--Sham eigenvalue equations, coupled self-consistently through the mean-field.
These equations must be solved repeatedly during the iterative HF/HFB cycle, and their efficient solution dominates the overall computational cost.

As shown in the present work, applying GCG to the HF single-particle problem provides a promising route towards efficient, symmetry-unrestricted many-body calculations while mitigating the main bottlenecks of fully coordinate-mesh methods.

\section{Objectives}
The aim of this work is to develop a new implementation of the Hartree--Fock method on an unconstrained 3D mesh, by the use of the Generalised Conjugate Gradient method. The goals addressed by this work are the following:
\begin{itemize}
    \item demonstrate the feasibility of the Generalised Conjugate Gradient for the solution of large-scale eigenvalue problems;
    \item solve the self-consistent Hartree--Fock equations on an unconstrained 3D mesh;
    \item verify the numerical accuracy of the new implementation, first against existing spherical codes;
    \item second against well-established deformed codes; and
    \item attempt to produce novel results, and establish the advancement brought to the field by this work.
\end{itemize}

\section{Methods}
The methods used in this thesis can be grouped into two main components: the formulation of the Energy Density Functional and the solution of the resulting self-consistent equations.

\paragraph{Skyrme Energy Functional}  
The many-body nuclear problem is approached within the Hartree--Fock framework, described in section~\ref{sec:hf}.  
As discussed in Chapter~\ref{chap:intro}, a pure HF treatment is not sufficient for a quantitative description of nuclear structure, 
and a more general Energy Density Functional (EDF) formulation must be adopted.  
In this work, we employ the Skyrme EDF, whose construction and resulting mean-field equations are developed in section~\ref{sec:skyrme}.  
This provides the self-consistent single-particle Hamiltonian that forms the basis for the numerical treatment.

\paragraph{Finite Differences and Generalised Conjugate Gradient}  
Once the equations to be solved have been derived, their numerical solution requires both a spatial discretization scheme and an efficient solver for the large-scale eigenvalue problem that arises at each iteration of the self-consistent procedure.
Chapter~\ref{chap:numerical} details the methods adopted in this work, starting with the finite-difference discretization of derivatives in section~\ref{sec:finite_diff}.  
The resulting discretised eigenvalue problem is then treated using the GCG method to extract the relevant low-lying eigenstates, as described in section~\ref{sec:gcg}.  
Section~\ref{sec:minimisation} discusses implementation-specific aspects of the code, including convergence criteria, mixing strategies, and the choice of parameters required to ensure stable and efficient minimisation of the energy functional.


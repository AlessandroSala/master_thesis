\chapter{Results for deformed nuclei}
\label{chap:res_def}
Having established that the code works as expected for spherical nuclei, we can extend the calculations to deformed systems. In this chapter, the GCG implementation is validated against well-established deformed codes.
In addition, we also present some new original results with a focus on $\alpha$-clustering and nuclei near the drip lines.

This chapter is organised as follows. First, in Section~\ref{sec:beta}, the quadrupole deformation parameter $\beta_2$ is introduced and its two different definitions are compared. In Section~\ref{sec:mg_benchmark}, results for the deformed $^{24}$Mg are presented and used as a benchmark for the performance of our code, first against the HO basis expansion code \texttt{HFBTHO} in Section~\ref{sec:hfbtho}, second against the Cartesian mesh code \texttt{EV8} in Section~\ref{sec:ev8}. 
Finally, in Section~\ref{sec:novel}, original results on other deformed nuclei are discussed, in particular, in Section~\ref{sec:ne20} results regarding an alpha-clustering study in $^{20}$Ne, and in Section~\ref{sec:si42} results for the near-drip line nuclei $^{42}$Si and $^{28}$S.

\section{Deformation parameters}
\label{sec:beta}
When dealing with deformed nuclei, mean square radii are not sufficient to characterise the nuclear density. The main quantity used is the quadrupole deformation parameter $\beta_2$, similar to the one encountered in Section~\ref{sec:deformations}. There is an ambiguity in its definition in literature, where some \cite{MAREVIC2022108367} compute it by using the actual mean square radius of the density
\begin{equation}
    \label{eq:beta_real_radius}
    \beta_2 = \frac{4\pi\expval{\mathcal Q_{2 0}}}{5A\expval{r^2}},
\end{equation}
where $\expval{r^2}$ is the total mean square radius of the nucleus 
\begin{equation}
    \expval{r^2} = \frac{\int (\rho_n + \rho_p )r^2 d\bm r}{\int (\rho_n + \rho_p )d\bm r} = \expval{x^2+y^2+z^2}.
\end{equation}
While others \cite{HilaireGirod2006,Ryssens2015_EV8} normalise the quadrupole moment with respect to the parametetrised nuclear radius in formula \eqref{eq:nuc_radius}, which yields 
\begin{equation}
  \label{eq:beta}
  \beta_{2} = \frac{4\pi\expval{\mathcal Q_{20}}}{3AR^2}
\end{equation}
where the relation 
\begin{equation}
  \expval{r^2} = \frac 3 5 {R}^2
\end{equation}
is used and the value for $r_0$ is taken to be $1.2$ fm.

Using either of the two formulae (\ref{eq:beta_real_radius}, \ref{eq:beta}), because of the normalization with respect to the total radius and mass, the $\beta_2$ parameter allows the comparison of different nuclei across the chart.

In Figure~\ref{fig:beta_compare}, a comparison between the two formulae is presented; as shown by the shape of the curves, it is possible to observe that the normalisation of $\beta_2$ with respect to the real radius has an asymptotic behaviour for large deformations, since for $\expval{z^2}\to \infty$, assuming axial symmetry and constant volume $V$
\begin{equation}
\beta_2 \propto \frac{2z^2-2V/z}{z^2 + 2V/z}\underset{{z\to\infty}}{\to} 2, 
\end{equation}
while normalising with respect to the parametrised radius $R$ has no such asymptotic behaviour, since $R$ is constant for any deformation.
\begin{figure}[!ht]
  \centering
  \includegraphics[width=0.8\textwidth]{Images/beta_compare.pdf}
  \caption[Comparison between the two formulae for $\beta_2$.]{Comparison between the two formulae for $\beta_2$ in a deformation curve for $^{24}$Mg. $\beta_2'$ refers to the deformation parameter normalised by the parametrised radius, while $\beta_2$ refers to the deformation parameter normalised by the actual mean square radius.}
  \label{fig:beta_compare}
\end{figure}

The choice of $\beta_2$ used in the present work is the one computed with the parametrised radius~\eqref{eq:beta}. To avoid confusion in the reader and present results in a consistent manner, when comparing results with the \texttt{HFBTHO} code, which uses formula \eqref{eq:beta_real_radius}, the bare quadrupole moment $\expval{\mathcal Q_{20}}$ is shown instead.

\subsection{Code parameters and axial constraint}
As for our code, calculations are performed in a box $[-10,+10]$ fm.
In the case of the ground state calculation, a step size of $0.33$ fm is used, with a starting guess of a deformed Woods--Saxon with $\beta_2=0.4$.

The calculation in the case of the deformation curve is carried out imposing the following constraints:
\begin{align}
  \expval{\Re\mathcal Q_{22}} = \expval{\Im \mathcal Q_{22}} = 0,
  \\\expval{x} = \expval{y} = \expval{z} = 0,
  \\\expval{\mathcal Q_{20}} = q_{20}.
\end{align}
These constraints altogether impose an axial deformation on the system, the constraint on $\expval{\mathcal Q_{20}}$ alone is not sufficient because on an unconstrained mesh like in our case, the nucleus may deform on a different axis from the chosen one ($z$) or the $\expval{\mathcal Q_{20}}$ moment may be subject to spurious contributions from a centre of mass not in the origin; moreover, the axial symmetry of $\texttt{HFBTHO}$ doesn't allow broken axial symmetry configurations.

Regarding the stiffness $c$ and damping parameter $\mu$ of the ALM described in Section~\ref{sec:alm}, $c=0.005$ and $\mu=0.02$ were used. As for convergence criteria, a tolerance of $0.001$ on the value of $\beta_2 - \beta_{2, \text{target}}$ was used.
\section{$^{24}$Mg}
In the following section, results for $^{24}$Mg are presented, it's a natural choice to study how well deformations are represented by our framework, since it's light, very deformed and shows no pairing interaction in its ground state.
\subsection{\texttt{HFBTHO} code and calculation details}
\subsubsection{\texttt{HFBTHO}}
To benchmark the code in the case of nuclear deformation, the \texttt{HFBTHO} code was used \cite{MAREVIC2022108367}, it's a HFB code which minimizes the energy functional on a (Transformed) harmonic oscillator basis. Since $^{24}$Mg is a light nucleus, it still works well in this case.
All calculations were performed using 12 oscillator shells and assuming a zero pairing interaction. 
Default parameters were adopted for the quadrupole constraints. 
Since the version of \texttt{HFBTHO} used in this work has been compiled with the $J^2$ terms disabled, we present the results from our code both with and without them. 
The results obtained without them serve as a benchmark for the code, 
while those including the $J^2$ contribution illustrate their impact on the calculated observables.
\subsubsection{Code parameters and axial constraint}
As for our code, calculations are performed on a box $[-10, 10]$ fm.
In the case of the ground state calculation, a step size of $0.33$ fm is used, with a starting guess of a deformed Woods-Saxon with $\beta_2=0.4$.
\\The calculation in the case of the deformation curve is carried out imposing the following constraints
\begin{align}
  \expval{\Re\mathcal Q_{22}} = 0
  \\\expval{\Im \mathcal Q_{22}} = 0
  \\\expval{\mathcal Q_{20}} = q_{20}.
\end{align}
These constraints altogether impose an axial deformation on the system. This is done because on a full mesh like in our case, the nucleus may deform on a different axis from the chosen one (z), resulting in spurious contributions to the real deformation curve; moreover, the axial symmetry of $\texttt{HFBTHO}$ doesn't allow broken axial symmetry configurations.
\\Regarding the stiffness $c$ and damping parameter $\mu$ of ALM \ref{sec:alm}, $c=0.001$ and $\mu=0.1$ were used. As for convergence criteria, a tolerance of $0.001$ on the value of $\beta_2 - \beta_{2, \text{target}}$ was used.
\subsubsection{Ground state}
Table \ref{tab:mg_table} reports data for the ground state of $^{24}$Mg, compare with $\texttt{HFBTHO}$ results, while figure \ref{fig:mg_gs_density_axial} shows the total particle density.
Charge radii for the two codes are displayed but not compared, due to different formulas used for their computation.
\begin{table}[ht]
  \centering
  \begin{tabular}{lrccccc}
    \addlinespace[0.3em]
    \toprule
    && GCG & GCG no $J^2$ & \texttt{HFBTHO} & $\Delta$ & $\Delta\%$ \\
    \midrule
    $E_{\text{TOT}}$& [MeV]    & -195.854 & -197.219 &-197.030 & 0.189 & \num{9.52e-2} \\
    $\expval{ r^2_n}^{1/2}$    &[fm] & 3.0124 & 2.9998    & 2.9996 & 0.0002  & \num{6.67e-3}\\
    $\expval{ r^2_p}^{1/2}$    &[fm] & 3.0475 & 3.0346    & 3.0326  & 0.0020 & \num{6.59e-2}\\
    $\expval{ r^2_{ch}}^{1/2}$ &[fm] & 3.1364 & 3.1240    & 3.4614  & - & - \\
    $\expval{z^2}^{1/2}$ &[fm] & 2.145 &2.128 &- &-&-\\
    $\expval{x^2}^{1/2}$ &[fm] & 1.511 &1.511 & -&-&-\\
    $\expval{y^2}^{1/2}$ &[fm] & 1.514 &1.514 & -&-&-\\
    $\beta_2$ &[-] & 0.399 &0.390 & 0.390 & - & -  \\
    \bottomrule
  \end{tabular}
  \caption{Results for $^{24}$Mg ground state, no pairing interaction, box $[-10, 10]$ fm, step size 0.33 fm, SKM* parametrization.}
  \label{tab:mg_table}
\end{table}
\begin{figure}[h]
  \centering
  \includegraphics[width=1.0\linewidth]{Images/mg_gs_density_axial.pdf}
  \caption{Magnesium ground state density, calculation done on a box $[-10, 10]$ fm, step size 0.33 fm, SKM* parametrization}
  \label{fig:mg_gs_density_axial}
\end{figure}
\subsubsection{Deformation curve}
In figure \ref{fig:mg_no_pair_deformation}, the deformation curve is shown for $^{24}$Mg, without pairing. To counteract the sharp rise in CPU time, due to the high number of points in the curve, a coarser grid is used, hence the different minimum energy and $\beta_2$ values than the ones reported in table \ref{tab:mg_table}.
\begin{figure}[h]
  \centering
  \includegraphics[width=0.8\linewidth]{Images/mg_nopair_curve.pdf}
  \caption{Magnesium deformation curve, no pairing interaction, calculation done on a box $[-10, 10]$ fm, step size 0.66 fm, SKM* parametrization, neglecting $J^2$ terms.}
  \label{fig:mg_no_pair_deformation}
\end{figure}
\section{Original results}
\label{sec:novel}
In this final section, some original results for nuclei of interest are discussed.
In Section~\ref{sec:ne20}, some results are presented for the $^{20}$Ne nucleus, in particular, the Nucleon Localisation Function (NLF) is computed to highlight the formation of clusters.
In Section~\ref{sec:si42}, results for two near drip line nuclei are presented, namely $^{42}$Si and $^{28}$S.
\subsection{$\alpha$-clustering in light nuclei} 
\label{sec:ne20}
The formation of clusters in light nuclei has been a research focus object in recent years. The interest stems from different reasons. The formation of clusters at low density is a strong indicator of specific correlations (n-p correlations or alpha-particle, ie `quartetting' correlations) and a strong test for theory. At the same time, clustering may have impact on reactions and astrophysical processes \cite{Lombardo2023}. It has to be noted that cluster formation during fission has been highlighted \cite{Vretenar}. One of the hypothesis is that some light nuclei tend to form clusters of lighter particles, mainly $\alpha$-particles, as to minimise their energy, by displaying `molecular-like' bonds and resonances among the clusters. The phenomenon is not yet understood, with the following results we show that in the framework presented in this work, the formation of clusters is present in $^{20}$Ne.

\subsubsection{Nucleon localisation function}
The study of clusters is made possible by the use of the nucleon localisation function (NLF) \cite{NLF}, it is a measure of the conditional probability of finding a nucleon in the short vicinity of another one in space. When dealing with spin-saturated nuclei, as is the case of $^{20}$Ne, the NLF reduces to
\begin{equation}
    \label{eq:nlf}
    C_q(\bm r) = \bigg[1+ \bigg(\frac{\tau_q\rho_q -\frac 1 4 |\nabla\rho_q|^2}{\rho_q \tau_{q}^{\text{TF}}}\bigg)^2\bigg]^{-1}
\end{equation}
where $\tau_q^{\text{TF}}$ is the Thomas-Fermi kinetic energy density, defined as
\begin{equation}
  \tau_q^{\text{TF}}= \frac 3 5 (6\pi^2)^{2/3}\rho_q^{5/3}.
\end{equation}
\subsubsection{$\alpha-$clustering in $^{20}$Ne}
In Figure~\ref{fig:clustering}, the total particle densities and proton NLF of $^{20}$Ne are shown for different Skyrme functionals. It is possible to observe that while some functionals like KDE33 show strong peaks in the particle densities, all the considered functionals display well defined clusters in their respective NLF contours. Note that clustering in the intrisic frame of the nucleus does not necessarily imply clustering in the laboratory frame, for which projection methods are required \cite{clusterCondensation}.
\begin{figure}[htbp]
    \centering
    %--- Row 1 ---
    \begin{subfigure}[b]{0.45\textwidth}
        \centering
        \includegraphics[width=\textwidth]{Images/clustering/SLy4_density}
        \caption{Particle density, SLy4 functional.}
        \label{fig:ski3_density}
    \end{subfigure}
    \begin{subfigure}[b]{0.45\textwidth}
        \centering
        \includegraphics[width=\textwidth]{Images/clustering/SLy4_localization}
        \caption{NLF, SLy4 functional.}
        \label{fig:ski3_nlf}
    \end{subfigure}

    %--- Row 2 ---
    \begin{subfigure}[b]{0.45\textwidth}
        \centering
        \includegraphics[width=\textwidth]{Images/clustering/skm_density}
        \caption{Particle density, SkM* functional.}
        \label{fig:skm_density}
    \end{subfigure}
    \begin{subfigure}[b]{0.45\textwidth}
        \centering
        \includegraphics[width=\textwidth]{Images/clustering/skm_localization}
        \caption{NLF, SkM* functional.}
        \label{fig:skm_nlf}
    \end{subfigure}


    %--- Row 3 ---
    \begin{subfigure}[b]{0.45\textwidth}
        \centering
        \includegraphics[width=\textwidth]{Images/clustering/kde33_density}
        \caption{Particle density, KDE33 functional.}
        \label{fig:kde33_density}
    \end{subfigure}
    \begin{subfigure}[b]{0.45\textwidth}
        \centering
        \includegraphics[width=\textwidth]{Images/clustering/kde33_localization}
        \caption{NLF, KDE33 functional.}
        \label{fig:kde33_nlf}
    \end{subfigure}
    \caption[Particle densities and NLFs in $^{20}$Ne for different Skyrme functionals.]{Particle densities and NLFs in $^{20}$Ne for different Skyrme functionals. The densities show the nuclues to be very deformed and prolate, with possible clusters. The NLFs show a pronounced cluster formation on the top and bottom of a central core. } 
    \label{fig:clustering}
\end{figure}

\subsection{Nuclei near the drip lines}
The need of a mesh representation to account for weakly bound systems has been largely emphasised in previous chapters (see \ref{chap:intro}, \ref{chap:methods}). In this section, results regarding the two nuclei near drip line $^{42}$Si and $^{28}$S are presented, the former being a neutron-rich nucleus, the latter being a proton-rich nucleus. Being weakly bound systems, taking direct measurements of quantities like radii, deformations through spectroscopy etc, is not yet possible. 

We shall compare the experimental neutron $S_n$ or proton $S_p$ separation energy with the theoretical value calculated using Koopmans' theorem \cite{Koopmans1934_Theorem}. The theorem states that in a frozen orbitals approximation, where the mean-field is assumed to remain constant after the removal of a particle, the energy required to remove that particle is equal to the eigenvalue of the last occupied single-particle orbital with an opposite sign.

\subsubsection{$^{42}$Si}
\label{sec:si42}
$^{42}$Si is a deformed, light, neutron-rich nucleus, having $Z=14$ and $N=28$. We may look at its ground state neglecting the pairing interaction thanks to the magic number of neutrons, and the closed sub-shell $1d_{5/2}$. In Table~\ref{tab:si42_table}, data computed with some functionals is reported, along with the experimental extrapolated binding energy and neutron separation energy. In Figure~\ref{fig:si42_density}, the particle density of $^{42}$Si is shown for the SLy4 functional. 

\begin{table}[ht]
  \centering
  \begin{tabular}{lrcccccc}
    \toprule
    && SLy4 & SkM* & KDE33 & SkP & SkI3 & Exp. \\
    \midrule
    $E$& [MeV]    & -313.129    & -320.760 &-326.102 & -317.163 & -338.047 & -311.22\\
    $S_n$ &[MeV] & 4.349 & 4.990 & 4.132 & 4.221 & 5.439 & 4.458 \\
    $\expval{ r^2_n}^{1/2}$    &[fm] & 3.716 & 3.705    & 3.666 & 3.707  & 3.664& \\
    $\expval{ r^2_p}^{1/2}$    &[fm] & 3.294 &  3.276   & 3.247  & 3.284 & 3.200& \\
    $\expval{ r^2_{ch}}^{1/2}$    &[fm] & 3.380 & 3.362    & 3.334  & 3.370 & 3.287& \\
    $\beta_2$ &[-] & -0.332 & 0.313  & -0.308 & -0.302 &  -0.298              & \\
    \bottomrule
  \end{tabular}
  \caption[Results for $^{42}$Si.]{Results for $^{42}$Si, box $[-11,+11]$ fm, step size 0.37 fm. Experimental data taken from \cite{AMDC_website}.}
  \label{tab:si42_table}
\end{table}
\begin{figure}
  \centering
  \includegraphics[width=0.8\textwidth]{Images/si42_density_sly4}
  \caption[$^{42}$Si density $\rho(x, 0, z)$.]{$^{42}$Si density $\rho(x, 0, z)$, calculation done on a box $[-11,+11]$ fm, step size 0.37 fm, the nucleus is shown to be very deformed and oblate.} 
  \label{fig:si42_density}
\end{figure}

\subsubsection{$^{28}$S}
$^{28}$S is a deformed, light, proton-rich nucleus, having $Z=16$ and $N=12$. We may look at its ground state neglecting the pairing interaction thanks to the closed sub-shell $2s_{1/2}$ and a number of neutrons analogous to the one in $^{24}$Mg. In Table~\ref{tab:s28_table}, data computed with some functionals is reported, along with the experimental extrapolated binding energy and proton separation energy. In Figure~\ref{fig:s28_density}, the particle density of $^{28}$S is shown for the SLy4 functional.

\begin{table}[ht]
  \centering
  \begin{tabular}{lrccccc}
    \toprule
    && SLy4 & SkM* & KDE33 & SkI3 & Exp. \\
    \midrule
    $E$& [MeV]    & -209.688     & -211.642 &-221.668 & -226.337 & -209.406\\
    $S_n$ &[MeV] & 3.370 & 3.330 & 3.135 & 3.332 & 2.556 \\
    $\expval{ r^2_n}^{1/2}$      &[fm] & 3.013 & 2.997    & 2.964  & 2.930& \\
    $\expval{ r^2_p}^{1/2}$      &[fm] & 3.235 & 3.225   & 3.185  & 3.168 & \\
    $\expval{ r^2_{ch}}^{1/2}$   &[fm] & 3.318 & 3.308    & 3.269  & 3.252& \\
    $\beta_2$ &[-]               & 0.314 & 0.289  & 0.293 &  0.315              & \\
    \bottomrule
  \end{tabular}
  \caption[Results for $^{28}$S.]{Results for $^{28}$S, box $[-10,+10]$ fm, step size 0.34 fm. Experimental data taken from \cite{AMDC_website}.}
  \label{tab:s28_table}
\end{table}
\begin{figure}
  \centering
  \includegraphics[width=0.8\textwidth]{Images/s28_density_sly4}
  \caption[$^{28}$S density $\rho(x, 0, z)$.]{$^{28}$S density $\rho(x, 0, z)$, calculation done on a box $[-10,+10]$ fm, step size 0.34 fm, the nucleus is shown to be deformed and oblate.}
  \label{fig:s28_density}
\end{figure}




\section{$^{24}$Mg Benchmarks}
\label{sec:mg_benchmark}
In the following section, results for $^{24}$Mg are presented and used as a benchmark for the performance of our code in a deformed case. $^{24}$Mg is a light, deformed nucleus, whose pairing correlations are negligible in its ground state and for deformations around it.

In section \ref{sec:hfbtho}, the axially symmetric HO basis code \texttt{HFBTHO} is used, since it is a well established code for axial calculations. We will compare results regarding the ground state and the constrained deformation curves of $^{24}$Mg. In section \ref{sec:ev8}, a more accurate comparison of the deformation curve is presented using the cartesian mesh code \texttt{EV8}.
\subsection{\texttt{HFBTHO} code and calculation details}
\label{sec:hfbtho}
\subsubsection{\texttt{HFBTHO}}
To benchmark the code against a basis expansion method in the case of nuclear deformation, the \texttt{HFBTHO} code was used \cite{MAREVIC2022108367}, it is a HFB solver which minimises the energy functional on a (transformed) harmonic oscillator basis. Since $^{24}$Mg is a light, bound nucleus, it still works well in this case.
All calculations were carried out using 12 oscillator shells and assuming a zero pairing interaction. 
Default parameters were adopted for the quadrupole constraints. 
Since the version of \texttt{HFBTHO} used in this work has been compiled with the $J^2$ terms disabled, we present the results from our code both with and without them. 
The results obtained without them serve as a benchmark for the code, 
while those including the $J^2$ contribution illustrate its impact on the ground state.
\subsubsection{Code parameters and axial constraint}
As for our code, calculations are performed on a box $[-10, 10]$ fm.
In the case of the ground state calculation, a step size of $0.33$ fm is used, with a starting guess of a deformed Woods-Saxon with $\beta_2=0.4$.

The calculation in the case of the deformation curve is carried out imposing the following constraints
\begin{align}
  \expval{\Re\mathcal Q_{22}} = \expval{\Im \mathcal Q_{22}} = 0,
  \\\expval{x} = \expval{y} = \expval{z} = 0,
  \\\expval{\mathcal Q_{20}} = q_{20}.
\end{align}
These constraints altogether impose an axial deformation on the system, the constraint on $\expval{\mathcal Q_{20}}$ alone is not sufficient because on an unconstrained mesh like in our case, the nucleus may deform on a different axis from the chosen one ($z$) or the $\mathcal Q_{20}$ moment may be subject to spurious contributions of a center of mass not centered in the origin; moreover, the axial symmetry of $\texttt{HFBTHO}$ doesn't allow broken axial symmetry configurations.

Regarding the stiffness $c$ and damping parameter $\mu$ of the ALM described in section \ref{sec:alm}, $c=0.005$ and $\mu=0.02$ were used. As for convergence criteria, a tolerance of $0.001$ on the value of $\beta_2 - \beta_{2, \text{target}}$ was used.
\subsubsection{Ground state}
Table \ref{tab:mg_table} reports data of the comparison for the ground state of $^{24}$Mg. In figure \ref{fig:mg_gs_density_axial} contours of the density at $y=0$ are shown, while in figure \ref{fig:mg_gs_top} the density viewed `from the top', ie $z=0$ is shown. As remarked by the density contours, the ground state is axially symmetric, quadrupole deformed, and parity-defined.

Charge radii for the two codes are displayed but not compared, due to different formulas used for their computation. $\expval{x^2}, \expval{y^2}$ and $\expval{z^2}$ are reported for our code but not for \texttt{HFBTHO} since it does not compute them.
\begin{table}[ht]
  \centering
  \begin{tabular}{lrccccc}
    \addlinespace[0.3em]
    \toprule
    && GCG & GCG no $J^2$ & \texttt{HFBTHO} & $\Delta$ & $\Delta\%$ \\
    \midrule
    $E_{\text{TOT}}$& [MeV]    & -195.854 & -197.219 &-197.030 & 0.189 & \num{9.52e-2} \\
    $\expval{ r^2_n}^{1/2}$    &[fm] & 3.0124 & 2.9998    & 2.9996 & 0.0002  & \num{6.67e-3}\\
    $\expval{ r^2_p}^{1/2}$    &[fm] & 3.0475 & 3.0346    & 3.0326  & 0.0020 & \num{6.59e-2}\\
    $\expval{ r^2_{ch}}^{1/2}$ &[fm] & 3.1364 & 3.1240    & 3.4614  & - & - \\
    $\expval{z^2}^{1/2}$ &[fm] & 2.145 &2.128 &- &-&-\\
    $\expval{x^2}^{1/2}$ &[fm] & 1.511 &1.511 & -&-&-\\
    $\expval{y^2}^{1/2}$ &[fm] & 1.514 &1.514 & -&-&-\\
    $\beta_2$ &[-] & 0.399 &0.390 & 0.390 & - & -  \\
    \bottomrule
  \end{tabular}
  \caption{Results for $^{24}$Mg ground state, no pairing interaction, box $[-10, 10]$ fm, step size 0.33 fm, SkM* parametrisation.}
  \label{tab:mg_table}
\end{table}

The comparison shows good agreement between the two codes, with the same $\beta_2$ minimum and similar ground state properties.
\begin{figure}[h]
  \centering
  \includegraphics[width=1.0\linewidth]{Images/mg_gs_density_axial}
  \caption{$^{24}$Mg ground state density $\rho(x, 0, z)$, calculation done on a box $[-10, 10]$ fm, step size 0.33 fm, SkM* parametrisation}
  \label{fig:mg_gs_density_axial}
\end{figure}
\begin{figure}[h]
  \centering
  \includegraphics[width=1.0\linewidth]{Images/mg_gs_density_top}
  \caption{$^{24}$Mg ground state density $\rho(x, y, 0)$, calculation done on a box $[-10, 10]$ fm, step size 0.33 fm, SkM* parametrisation.}
  \label{fig:mg_gs_top}
\end{figure}
\subsubsection{Deformation curve}
In figure \ref{fig:mg_no_pair_deformation}, the deformation curve is shown for $^{24}$Mg, without pairing. The quadrupole constraint is imposed for values of $\beta_2$ starting from $0.4$ fm and going up to $0.65$, since the HO expansion stops converging near the requested $\beta_2$ after this value, and using a discretisation of $0.02$.
\begin{figure}[h]
  \centering
  \includegraphics[width=0.8\linewidth]{Images/mg_nopair_curve.pdf}
  \caption{$^{24}$Mg deformation curve, no pairing interaction, calculation done on a box $[-10, 10]$ fm, step size 0.66 fm, SkM* parametrisation, neglecting $J^2$ terms.}
  \label{fig:mg_no_pair_deformation}
\end{figure}

Figure \ref{fig:mg_no_pair_deformation} shows the same trend for both codes, with a minimum of the energy in $\beta_2=0.390$, albeit a difference in the energies due to the coarse mesh, a gap which is shown in table \ref{tab:mg_table} to shrink when increasing the accuracy of the step size.
\subsection{\texttt{EV8} code and calculation details}
\label{sec:ev8}
$\texttt{HFBTHO}$ can be used to judge how well our implementation fares against other codes for deformed nuclei, but its numerical methodologies are profoundly different from the ones used in our implementation. 

An interesting alternative is the \texttt{EV8} code \cite{Ryssens2015_EV8}, which is a HF code that solves the problem on a 3D mesh. Unlike our case, the problem is reduced to a single octant of the box, which allows for quicker computational times, imposing plane reflection symmetry on the three planes $(x, y, 0),\ (x,0,z),\ (0,y,z)$, and the KS equations are discretised using Lagrange derivatives.

\subsubsection{\texttt{EV8} code comparison}
Figure \ref{fig:ev8_compare_nopair} shows the comparison between the deformation curves of the two codes for $^{24}$Mg. The parameters for our calculation are the same as those used in the \texttt{HFBTHO} comparison in section \ref{sec:hfbtho}, regarding the input parameters of $\texttt{EV8}$, a step size of $0.8$ fm, and a number of points on each axis of $32$ is used. The results are compared using the definition of $\beta_2$ in formula \eqref{eq:beta}, which uses a constant radius, hence the apparent wider range of $\beta_2$ values compared to the \texttt{HFBTHO} comparison in figure \ref{fig:mg_no_pair_deformation}.

\begin{figure}[htp]
  \centering
  \includegraphics[width=0.8\textwidth]{Images/ev8_compare_nopair}
  \caption{Comparison with the $\texttt{EV8}$ code for $^{24}$Mg, no pairing interaction, box $[-10, 10]$ fm, step size 0.6 fm, SLy4 parametrisation.}
  \label{fig:ev8_compare_nopair}
\end{figure}

As show by the comparison in figure \ref{fig:ev8_compare_nopair}, the results are much closer than the ones obtained with the \texttt{HFBTHO} code. A slight deviation is still present, likely due to the different numerical methods used by the two codes, both for the KS equations discretization and the minimisation of the energy functional. 
\subsection{Deformation curve for different functionals}
In figure \ref{fig:functionals_compare}, the deformation curves for $^{24}$Mg are compared for different Skyrme functionals.
\begin{figure}[htp]
  \centering
  \includegraphics[width=0.8\textwidth]{Images/functionals_compare}
  \caption{Comparison between different Skyrme functionals of the $^{24}$Mg deformation curves, box $[-10, 10]$ fm, step size $0.6$ fm.}
  \label{fig:functionals_compare}
\end{figure}
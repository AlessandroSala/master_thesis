\section{$^{24}$Mg Benchmarks}
In the following section, the results for $^{24}$Mg are presented and used as a benchmark for the performance of our code in a deformed case.
\subsection{Deformation parameters}
When dealing with deformed nuclei, mean square radii are not sufficient to characterize the nuclear density. The main parameter used is the quadrupole deformation parameter $\beta_2$, similar to the one encountered in section \ref{sec:deformations}, it can be computed through the actual mean square radius with formula \eqref{eq:beta_real_radius}
\begin{equation}
    \label{eq:beta_real_radius}
    \beta_2 = \frac{4\pi\expval{\mathcal Q_{2 0}}}{5A\expval{r^2}},
\end{equation}
where $\expval{r^2}$ is the total mean square radius of the nucleus 
\begin{equation}
    \expval{r^2} = \frac{\int (\rho_n + \rho_p )r^2 d\bm r}{\int (\rho_n + \rho_p )d\bm r} = \expval{x^2+y^2+z^2}
\end{equation}
or normalizing with respect to the parametetrised nuclear radius using formula \eqref{eq:nuc_radius}
\begin{equation}
  \label{eq:beta}
  \beta_{2} = \frac{4\pi\expval{\mathcal Q_{20}}}{3AR^2}
\end{equation}
where the relation 
\begin{equation}
  \expval{r^2} = \frac 3 5 \expval{|r|}^2
\end{equation}
is used, and the value for $r_0$ is taken to be $1.2$, in accordance with the one used in other codes \cite{Ryssens2015_EV8}.

Using either of the two formulae (\ref{eq:beta_real_radius}, \ref{eq:beta}), thanks to the normalization with respect to the total radius and mass, the $\beta_2$ parameter can be used to compare different nuclei across the chart.

The choice of $\beta_2$ used in the following deformation curves depends on the code which is used to benchmark the results. For comparisons with \texttt{HFBTHO}, formula \eqref{eq:beta_real_radius} is used, while for \texttt{EV8} and results in other sections, formula \eqref{eq:beta} is used. In figure \ref{fig:beta_compare}, a comparison between the two formulae is presented; as shown by the shape of the curves, it's possible to observe that $\beta_2$ has an asymptotic behaviour for large deformations, since for large $\expval{z^2}$, assuming axial symmetry and constant volume $V$
\begin{equation}
\beta_2 \propto \frac{2z^2-2V/z}{z^2 + 2V/z}\underset{{z\to\infty}}{\to} 2, 
\end{equation}
while $\beta_2'$ remains unbiased by the mean square radius, since the normalization is taken to be constant.
\begin{figure}
  \centering
  \includegraphics[width=0.8\textwidth]{Images/beta_compare.pdf}
  \caption{Comparison between the two formulae for $\beta_2$ in a deformation curve for $^{24}$Mg. $\beta_2'$ refers to the deformation parameter normalised by the parametrized radius, while $\beta_2$ refers to the deformation parameter normalised by the actual mean square radius.}
  \label{fig:beta_compare}
\end{figure}

\subsection{\texttt{HFBTHO} code and calculation details}
\label{sec:hfbtho}
\subsubsection{\texttt{HFBTHO}}
To benchmark the code in the case of nuclear deformation, the \texttt{HFBTHO} code was used \cite{MAREVIC2022108367}, it's a HFB code which minimizes the energy functional on a (Transformed) harmonic oscillator basis. Since $^{24}$Mg is a light nucleus, it still works well in this case.
All calculations were performed using 12 oscillator shells and assuming a zero pairing interaction. 
Default parameters were adopted for the quadrupole constraints. 
Since the version of \texttt{HFBTHO} used in this work has been compiled with the $J^2$ terms disabled, we present the results from our code both with and without them. 

The results obtained without them serve as a benchmark for the code, 
while those including the $J^2$ contribution illustrate its impact on the ground state.
\subsubsection{Code parameters and axial constraint}
As for our code, calculations are performed on a box $[-10, 10]$ fm.
In the case of the ground state calculation, a step size of $0.33$ fm is used, with a starting guess of a deformed Woods-Saxon with $\beta_2=0.4$.

The calculation in the case of the deformation curve is carried out imposing the following constraints
\begin{align}
  \expval{\Re\mathcal Q_{22}} = \expval{\Im \mathcal Q_{22}} = 0,
  \\\expval{x} = \expval{y} = \expval{z} = 0,
  \\\expval{\mathcal Q_{20}} = q_{20}.
\end{align}
These constraints altogether impose an axial deformation on the system, the constraint on $\mathcal Q_{20}$ alone is not sufficient because on an unconstrained mesh like in our case, the nucleus may deform on a different axis from the chosen one ($z$) or the $\mathcal Q_{20}$ moment may be subject to spurious contributions of a center of mass not centered in the origin; moreover, the axial symmetry of $\texttt{HFBTHO}$ doesn't allow broken axial symmetry configurations.

Regarding the stiffness $c$ and damping parameter $\mu$ of ALM described in section \ref{sec:alm}, $c=0.005$ and $\mu=0.02$ were used. As for convergence criteria, a tolerance of $0.001$ on the value of $\beta_2 - \beta_{2, \text{target}}$ was used.
\subsubsection{Ground state}
Table \ref{tab:mg_table} reports data of the comparison for the ground state of $^{24}$Mg. In figure \ref{fig:mg_gs_density_axial} a slice of the density at $y=0$ is shown, while in figure \ref{fig:mg_gs_top} the density viewed `from the top', ie $z=0$ is shown. 

Charge radii for the two codes are displayed but not compared, due to different formulas used for their computation. $\expval{x^2}, \expval{y^2}$ and $\expval{z^2}$ is reported for our code but not for \texttt{HFBTHO} since it doesn't compute them.
\begin{table}[ht]
  \centering
  \begin{tabular}{lrccccc}
    \addlinespace[0.3em]
    \toprule
    && GCG & GCG no $J^2$ & \texttt{HFBTHO} & $\Delta$ & $\Delta\%$ \\
    \midrule
    $E_{\text{TOT}}$& [MeV]    & -195.854 & -197.219 &-197.030 & 0.189 & \num{9.52e-2} \\
    $\expval{ r^2_n}^{1/2}$    &[fm] & 3.0124 & 2.9998    & 2.9996 & 0.0002  & \num{6.67e-3}\\
    $\expval{ r^2_p}^{1/2}$    &[fm] & 3.0475 & 3.0346    & 3.0326  & 0.0020 & \num{6.59e-2}\\
    $\expval{ r^2_{ch}}^{1/2}$ &[fm] & 3.1364 & 3.1240    & 3.4614  & - & - \\
    $\expval{z^2}^{1/2}$ &[fm] & 2.145 &2.128 &- &-&-\\
    $\expval{x^2}^{1/2}$ &[fm] & 1.511 &1.511 & -&-&-\\
    $\expval{y^2}^{1/2}$ &[fm] & 1.514 &1.514 & -&-&-\\
    $\beta_2$ &[-] & 0.399 &0.390 & 0.390 & - & -  \\
    \bottomrule
  \end{tabular}
  \caption{Results for $^{24}$Mg ground state, no pairing interaction, box $[-10, 10]$ fm, step size 0.33 fm, SkM* parametrization.}
  \label{tab:mg_table}
\end{table}

The comparison shows good agreement between the two codes, with the same $\beta_2$ minimum and similar ground state properties.
\begin{figure}[h]
  \centering
  \includegraphics[width=1.0\linewidth]{Images/mg_gs_density_axial.pdf}
  \caption{Magnesium ground state density $\rho(x, 0, z)$, calculation done on a box $[-10, 10]$ fm, step size 0.33 fm, SkM* parametrization}
  \label{fig:mg_gs_density_axial}
\end{figure}
\begin{figure}[h]
  \centering
  \includegraphics[width=1.0\linewidth]{Images/mg_gs_density_top}
  \caption{Magnesium ground state density $\rho(x, y, 0)$, calculation done on a box $[-10, 10]$ fm, step size 0.33 fm, SkM* parametrization.}
  \label{fig:mg_gs_top}
\end{figure}
\subsubsection{Deformation curve}
In figure \ref{fig:mg_no_pair_deformation}, the deformation curve is shown for $^{24}$Mg, without pairing. To counteract the sharp rise in CPU time, due to the high number of points in the curve, a coarser lattice than the one in the ground state calculation is used, hence the shift in energy of the curve.
\begin{figure}[h]
  \centering
  \includegraphics[width=0.8\linewidth]{Images/mg_nopair_curve.pdf}
  \caption{Magnesium deformation curve, no pairing interaction, calculation done on a box $[-10, 10]$ fm, step size 0.66 fm, SkM* parametrization, neglecting $J^2$ terms.}
  \label{fig:mg_no_pair_deformation}
\end{figure}

Figure \ref{fig:mg_no_pair_deformation} shows the same trend for both codes, with a minimum of the energy in $\beta_2=0.390$, albeit a difference in the energies due to the corase mesh, a gap which is shown in table \ref{tab:mg_table} to shrink when increasing the accuracy of the step size.
\subsection{\texttt{EV8} code and calculation details}
\label{sec:ev8}
$\texttt{HFBTHO}$ can be used to judge how well our implementation fares against other codes for deformed nuclei, but its numerical methodologies are profoundly different from the ones used in our implementation. 

An interesting alternative is the \texttt{EV8} code \cite{Ryssens2015_EV8}, which is a HF code that solves the problem on a 3D mesh. Unlike our case, quantities are represented only on a single octant of the box, which allows for quicker computational times, imposing plane reflection symmetry on the three planes, and the KS equations are solved using Fast Fourier Transforms (FFTs). .

\subsection{\texttt{EV8} code comparison}
Figure \ref{fig:ev8_compare_nopair} shows the comparison between the deformation curves of the two codes for $^{24}$Mg, with the $J^2$ terms included. The parameters for our calculation are the same as those used in the \texttt{HFBTHO} comparison in section \ref{sec:hfbtho}, regarding the input parameters of $\texttt{EV8}$, a step size of $0.8$ fm, and a number of points on each axis of $32$ is chosen.

\begin{figure}[htp]
  \centering
  \includegraphics[width=0.8\textwidth]{Images/ev8_compare_nopair.pdf}
  \caption{Comparison with the $\texttt{EV8}$ code for $^{24}$Mg, no pairing interaction, box $[-10, 10]$ fm, step size 0.6 fm, SLy4 parametrization.}
  \label{fig:ev8_compare_nopair}
\end{figure}

\section{Other results}
\subsection{Deformation curve for different functionals}
In figure \ref{fig:functionals_compare}, the deformation curves for $^{24}$Mg are compared for different Skyrme functionals.
\begin{figure}[htp]
  \centering
  \includegraphics[width=0.8\textwidth]{Images/functionals_compare}
  \caption{Comparison between different Skyrme functionals of the $^{24}$Mg deformation curves, box $[-10, 10]$ fm, step size $0.6$ fm.}
  \label{fig:functionals_compare}
\end{figure}